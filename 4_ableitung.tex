\chapter{Ableitung Modell}\label{sec:ableitung}

\section{Regelung}

Die Regelung umfasst Maßnahmen während des Serienanlaufs. Dazu gehören zum einen die Beherrschung von Komplexität und Unsicherheit und zum anderen Strategien zum optimalen Umgang mit Störungen. 

Zur Beherrschung von Unsicherheiten eignet sich die Abbildung komplexer Systeme in abstrahierten Modellen. Die Heuristik hilft bei Zeitdruck und mangelhafter Datenbasis gute Entscheidungen zu treffen. In komplexen Systemen können Muster erkannt und Regeln abgeleitet werden, ohne die Funktion oder die Ursachen zu verstehen. 

Zum optimalen Umgang mit Störungen soll im Vorfeld ein Methodenbaukasten entwickelt werden. Dazu gehört eine Auswahl an modernen Methoden und Werkzeugen sowie die Definition von Standards. Zur objektiven Bewertung von Situationen sollen quantifizierbare Anlaufindikatoren definiert werden. Werden sie mit Hilfe einer Scorecard laufend überwacht, kann eine schnelle und zielgerichtete Reaktion stattfinden. 

\section{Produktentwicklung}

Die Produktentwicklung soll Fehler im Produkt und Prozess frühzeitig vermeiden. Möglichst früh soll eine hohe Produkt- und Prozessreife erreicht werden. 

Beim Aufbau digitaler Produktdatenmodelle soll stets mit aktuellen, echten und plausiblen Daten gearbeitet werden. Produkt- und Prozessdaten sollen mit Hilfe sog. \gls{ipdm}-Systeme verknüpft werden. Somit bekommen Änderungen mehrdimensionalen Charakter. Die Änderungsfolgen können besser simuliert und bewertet werden. 

Das Set-based Engineering ermöglicht frühzeitiges Erreichen hoher Reifegrade. Es werden parallel verschiedene Lösungsentwürfe entwickelt. Das zu Beginn kostenintensive Verfahren zahlt sich später durch schnellere Marktreife und Kostenersparnisse in Serienanlauf und -produktion aus. 

\section{Wissen}

Nachhaltiges Wissensmanagement ermöglicht bessere Nutzung firmeninternen Wissens sowie den gezielten Einsatz von Lernprozessen. 

Wissenslücken, insbesondere mit Hinblick auf die Erfüllung der Kundenwünsche, müssen frühzeitig identifiziert werden. Dazu eignen sich Produkttests an Testpersonen. Geeignete Erhebungsinstrumente sind hierfür zu entwickeln. Die gewonnenen Erkenntnisse müssen entlang der gesamten Prozesskette zur Verfügung gestellt werden. Dazu eignet sich ein sog. E-Workflowsystem, welches eine Art \gls{erp}-System darstellt. 

Wissen kann auch gezielt durch Lernprozesse generiert werden. Die Prototypen- und Nullserienphase sollen als Versuchsfeld betrachtet werden. Geplante Versuche in diesen Phasen führen zu Erkenntisgewinn und letztlich zu Verkürzung der Anlaufzeit. 

\section{Qualität}

Gezielter Methodeneinsatz ermöglicht frühzeitige und effektive Analyse von Ursache-Wirkungs %TODO groß oder kleinschreibung? Siehe auch 3.6 QM : Zink2010  
Zusammenhängen bei Qualitätsproblemen. 

Das weit verbreitete Ishikawa-Diagramm wird an die Bedürfnisse des Serienanlaufs angepasst. Dazu wurde der sog. Hypothesen-Suchraum entwickelt. Dieser spannt sich über die bekannten und ggf. anzupassenden Ishikawa-Dimensionen (Methode, Material, Mensch, Messung, Milieu, Maschine) und weiterhin durch die relevanten Phasen im Produktlebenszyklus auf. Nachdem potentielle Ursachen  von Qualitätsproblemen identifiziert wurden, werden sie einzelnen Phasen zugeordnet. Im zweiten Schritt werden mit Hilfe des sog. Wirkgefüges Zusammenhänge zwischen den Einflussfaktoren erarbeitet. Dabei bildet das Wirkgefüge die inhaltlichen Wechselwirkungen zwischen den Ursachen übersichtlich ab und lässt Rückschlüsse auf die Hauptursachen zu.

\section{Risiken}

Mit der Adaption von klassischen Risikomanagement-Methoden auf den Serienanlauf können anlaufrobuste Produktionssysteme ermöglicht werden. 

Zur systematischen Erfassung von Risiken werden strukturierte Risikobereiche gebildet. Risiken lassen sich zu Produktionssystemelementen zuordnen. Anschließend werden sie unter dem Aspekt der Auswirkungen klassifiziert. Es folgt eine getrennte Betrachtung von Leistungs- und Kostenrisiken. 
Während die Kostenrisiken direkt quantifiziert werden können, müssen Leistungsrisiken im Rahmen der Szenarioberechnung bewertet werden (siehe Abschnitt \ref{sec:wildemannszenarien}).
Mit Hilfe der Szenarioberechnung wird ein Risikoportfolio erarbeitet. Vier Handlungsstrategien welche in Abschnitt  \ref{sec:wildemannszenarien} beschrieben wurden, lassen sich differenziert den einzelnen Risiken im Portfolio zuordnen.