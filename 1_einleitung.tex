% \section{Einführung}
\chapter{Einführung}\label{sec:einfuehrung}
\section{Motivation \& Problemstellung}
Die produzierende Industrie findet sich heutzutage in einem zunehmend dynamischen Wettbewerbsumfeld wieder, welches vielschichtige Herausforderungen mit sich bringt \cite{Renner2012}. Die hauptsächlichen Herausforderungen liegen in steigenden Innovationsgeschwindigkeiten, kürzeren Produktlebenszyklen und einer höheren Variantenvielfalt \cite{Kuhn2002,Stauder2016}. Um dem durch die Globalisierung verstärkten Wettbewerb standzuhalten, müssen produzierende Unternehmen innovative Produkte und Dienstleistungen anbieten und sich zunehmend kundenorientiert aufstellen \cite{Surbier2014}. 
Eine zentrale Rolle wird hier dem Anlauf von Serienprodukten zugeschrieben. Aufgrund immer kürzer werdender Produktlebenszyklen rücken Kosten und Zeitaufwand in den Vordergrund \cite{Winkler2007}. So hat der Anlauf einen signifikanten Einfluss auf den wirtschaftlichen Erfolg des Produkts und die Time-to-Volume \cite{Klocke16}. Selbst ein um wenige Monate verschobener Verkaufsstart kann über Erfolg oder Misserfolg des Produkts entscheidend sein \cite{Schuh2008a}. Die Bedeutung der Serienanläufe findet bisher in der Wissenschaft keine angemessene Aufarbeitung \cite{Dyckhoff2012}. 

\section{Fokus der Arbeit}
Der Trend zur Konzentration auf Kernkompetenzen sorgt dafür, dass in großen Unternehmen immer mehr Wertschöpfungsanteile an Zulieferer abgegeben werden  \cite{Hilmola2015, Wildemann2008}. Der Gesamtanlauf setzt sich fortan aus vielen lokalen Einzelanläufen zusammen \cite{Zimolong2006}. Daraus resultieren höhere Abhängigkeiten zwischen größeren Unternehmen und den Zulieferern, die meist mittelständische Unternehmen sind. 

Die Abschlussarbeit soll sich im Speziellen mit dem Serienanlauf im KMU und SME als Zulieferer für größere Unternehmen beschäftigen, da hier erhebliches Verbesserungspotential erkennbar ist \cite[S.18]{Dombrowski2009a}. So gibt es in KMU meist keine Anlaufprozesse. Da es in KMU oft keine Stabsstellen gibt, werden Anläufe von den Mitarbeitern oft zusätzlich zum Tagesgeschäft gesteuert \cite{Dombrowski2009}. %TODO kein Zugriff auf Primärquelle D.Spath!! 
Mangelnde finanzielle und zeitliche Kapazitäten sowie fehlendes Know-how verhindern eine nachvollziehbare Dokumentation sowie proaktive Maßnahmen \cite{Zimolong2006,Dombrowski2009a}. 

Weiterhin soll untersucht werden, wie der Auftraggeber den Anlaufprozess des Lieferanten unterstützen kann. Größere Unternehmen verfügen in der Regel über mehr Ressourcen und teilweise eigene Anlaufprozesse. Im Zuge der Verlagerung der Wertschöpfungsanteile, gewinnt die Innovationskraft von Modul- und Systemlieferanten zunehmend an Bedeutung für den Erfolg eines Produktes \cite{Kuhn2002}. Ein erfolgreiches und effizientes Anlaufmanagement in KMU ist im Sinne der Entwicklung einer nachhaltigen Partnerschaft für Auftraggeber und Lieferant von großer Bedeutung. \textit{Wildemann} erkennt hier das Potenzial von Einspareffekten sowie Nutzung erheblicher Wettbewerbsvorteile auf beiden Seiten \cite{Wildemann2008}.

\textit{Dyckhoff} und \textit{Scholz} sind zu der Erkenntniss gekommen, dass das Thema weder in Industrie noch in der Wissenschaft hinreichend Beachtung findet \cite{Dyckhoff2012, Scholz2010}, weshalb hier keine zufriedenstellenden Ergebnisse zu erwarten sind.
Ziel der Arbeit ist, einen Überblick über den Stand der Forschung zu geben und einen Entwurf für ein Anlaufmodell zu entwickeln. 

\section{Herangehensweise}
Die Abschlussarbeit wird eine Literaturarbeit. In der Einführung erfolgt eine knappe Darstellung der zu behandelnden Themen Lean Startup / KMU und Anlaufmanagement. Im Hauptteil wird zunächst der Stand der Wissenschaft zum Thema Lean Startup skizziert. Den größeren Teil bildet eine umfassende Literaturanalyse zum Stand der Wissenschaft des Anlaufmanagements. Die Literaturrecherche erfolgt nach fest definierten Kriterien. Für die Literaturanalyse werden mit Hilfe des Tools \textit{Atlas.ti} alle relevanten Textstellen gecoded, d.h. identifiziert und nachvollziehbar dokumentiert. Anhand der  Ergebnisse wird anhand von möglichst vielen Quellen der Stand der Wissenschaft dargestellt. Im nächsten Abschnitt werden für das Lean Startup nicht berücksichtigte Anforderungen an das Anlaufmanagement ermittelt und daraus eine Art Anlaufmodell abgeleitet. 

Die Validierung der Ergebnisse erfolgt durch Zitierung der Quellen. Auf eine Validierung durch Experten, Fragebögen oder empirische Untersuchungen wird aufgrund des großen Umfangs verzichtet.
%
\section{Kontext}

\subsection{Lean Start-up}
\subsubsection*{Einführung}
Das Lean Start-up ist eine Businessmethode für dynamische Unternehmen oder Projekte, die hohen Risiken und Unsicherheiten ausgesetzt sind. 
Hauptziele der Methode sind kürzere Entwicklungszeiten, Einsparung von Kosten in der Entwicklungsphase und frühzeitiges Erkennen der Kundenbedürfnisse. 
Sie ist eine Antwort auf hoch dynamische Märkte, unbekannte Problemstellungen und Lösungen und hohen Risiken. Die Ursprünge liegen in den Denkweisen von Taiichi Ōno, W. Edwards Deming und Peter Drucker. 
2008 übertrug Eric Ries Lean Produktions Methoden auf hochtechnologie Startups und veröffentlichte 2011 die erstmals "Lean Startup" genannte Methode in seinem Buch. %TODO 2011 or 2008? 

%\subsection*{Definitionen}

\subsubsection*{Bestandteile}

\textit{1. Entwickeln einer Vision}. Die Vision dient als Grundlage für alle weiteren Handlungen. Aus ihr werden im nächsten Schritt Hypothesen abgeleitet. Anstatt einen aufwändigen Businessplan zu erstellen wird die Vision in einem Business Model Canvas definiert \cite{Blank2013}. Die Vision eines Lean Start-up zeichnet sich durch viele Freiheitsgrade und Unsicherheiten aus. 

\textit{2. Überführen der Vision hin zu Hypothesen}. Für jedes Element der im Business Model Canvas beschriebenen Vision werden Hypothesen abgeleitet. Die Hypothesen bilden die Freiheitsgrade und Unsicherheiten des BMC ab. Ziel ist, die Risiken durch spätere Beantwortung der Hypothesen zu minimieren. Nach Möglichkeit sollen die Hypothesen so formuliert werden, dass sie quantitativ beurteilt werden können. Die Hypothesen müssen widerlegbar sein, um neue Erkenntnisse gewinnen zu können. 

\textit{3. Entwickeln von MVP Tests}. Ein minimal überlebensfähiges Produkt (\gls{mvp}, engl.: Minimum Viable Product) ist ein Werkzeug, mit dem man schnellstmöglich die Hypothesen am Kunden überprüfen kann \cite[93]{Ries2011}. Ziel ist zum einen den Build-Measure-Learn Zyklus zu beschleunigen, zum anderen die Lernrate in Bezug auf den Aufwand zu maximieren. So können frühzeitig nicht benötigte Funktionen und Produkteigenschaften erkannt und Zeit und Kosten gespart werden. Wenn die Entwicklung eines realen MVP zu aufwändig ist, kann ein Smoke Test eingesetzt werden. In einem Smoke Test wird das zukünftige Produkt in einem Video oder über eine Webseite vorgestellt.

\textit{4. Planung der Tests}. Bei der Durchführung der Tests kommt es darauf an, Kosten und Zeit zu minimieren. Daher werden zuerst Tests durchgeführt, die wenig kosten und hohe Risiken untersuchen. Beispielsweise ist eine Patentrecherche kostengünstig und kann frühzeitig sehr hohe Risiken aufdecken. Tests können nacheinander (seriell) oder gleichzeitig (parallel) durchgeführt werden. Bei parallelen Tests riskiert man, dass einzelne Tests überflüssig werden, profitiert jedoch von einem Zeitvorsprung gegenüber der seriellen Vorgehensweise. 

\textit{5. Interpretation der Ergebnisse}. Bei der Interpretation der Ergebnisse gibt es einige Fehlerquellen. Zum einen gibt es teilweise große Differenzen zwischen den geäußerten und reellen Kundenrückmeldungen. Zum anderen kann die Interpretation des Unternehmers durch eigene Wünsche oder Erwartungen verzerrt sein.

\textit{6. Reaktion}. Nach Auswertung der Ergebnisse sieht die LSU Methode eine Entscheidung zwischen drei Handlungsalternativen vor. \textit{Preserve}: Wenn die Tests die Hypothesen bestätigen wird die Strategie beibehalten. \textit{Pivot}: Wenn die Tests die Hypothesen widerlegen oder neue Chancen aufzeigen, wird die Strategie angepasst. \textit{Perish}: Wenn die Tests die Hypothesen widerlegen und der Unternehmer keine geeignete Strategie entwickeln kann, wird die Strategie verworfen. 

\textit{7. Skalierung und kontinuierliche Verbesserung}. Sobald alle relevanten Hypothesen bestätigt wurden, ist das Produkt auf den Markt abgestimmt. Jetzt kann massiv in Kundenakquise und Produktentwicklung investiert werden. Wichtig ist weiterhin, dass die Strategie fortwährend überprüft wird. Ein \textit{Pivot} ist auch nach der Skalierung bei größeren Änderungen sinnvoll. 

% \subsection*{Grenzen der Methodik}

\subsection{Anlaufmanagement}
Immer kürzere Produktlebenszyklen bei gleichzeitig höher werdenden Kundenwünschen und größerer Variantenvielfalt erhöhen die Komplexität und somit die Bedeutung des Serienanlaufs \cite{Kuhn2002,Schuh2004}. Die Risiken im Zusammenhang mit der Anlaufphase sind vielfältig. KUHN %TODO Biblatex Cite command for cap letter author in maintext
stellt fest, dass der Aufwand bis zum Erreichen einer stabilen Produktion oft unterschätzt wird. Infolgedessen kann es zum verspäteten Markteintritt sowie unzureichenden Kapazitäten und Qualitätsmängeln kommen \cite{Kuhn2002}. Um diesen Risiken entgegen zu wirken werden als übergeordnete Hauptziele für das Anlaufmanagement Beherrschung und zeitliche Verkürzung der Anlaufphase genannt \cite{Kuhn2002, Schmitt2015}. 

Produktionsanläufe stellen auch deshalb eine große Herausforderung für Unternehmen dar, da sie hochkomplex sind und sich durch viele parallele und sequenzielle Teilprozesse auszeichnen. Sie sorgen zudem für eine starke Vernetzung der beteiligten Abteilungen innerhalb und außerhalb des Unternehmens \cite{Schuh2004}.


\subsubsection*{Definition}
In der Literatur existiert keine Einheitliche Definition des Begriffs Anlaufmanagement \cite[4]{Bischoff2007}. Selbst SCHMITT %TODO 
bemängelte 2015 ein fehlendes einheitliches Verständnis der grundlegenden Begriffe des Produktionsanlaufs \cite[1]{Schmitt2015}. Vielmehr existieren unternehmensintern und teilweise auch projektspezifisch unterschiedliche Auffassungen über die Definition der Anlaufphase \cite[11]{Grosshenning2005}. KUHN %TODO
definierte das Anlaufmanagement wie folgt \cite[8]{Kuhn2002}: 
\begin{quotation}
Das Anlaufmanagement eines Serienproduktes umfasst alle Tätigkeiten und Maßnahmen zur Planung, Steuerung und Durchführung des Anlaufs mit den dazugehörigen Produktionssystemen, ab der Freigabe der Vorserie bis zum Erreichen einer geplanten Produktionsmenge, unter Einbeziehung vorgelagerter Prozesse und der nachgelagerten Prozesse im Sinne einer messbaren Eignung der Produkt- und Prozessreife.
\end{quotation}
SCHUH übernahm diese Auffassung \cite{Schuh08a} während RISSE und BISCHOFF den Beginn bereits nach der abgeschlossenen Produktentwicklung sehen \cite{Risse2002, Bischoff2007} (Freigabe Pflichtenheft).

Der Anwendungsbereich beschränkt sich nicht nur auf den Anlauf von neuen Produkten. Auch Modellderivate (Modellpflege), Varianten; neue Produktionssysteme, Fertigungsverfahren und Logistikprozesse stellen aus Perspektive des Managements ein Anlauf dar \cite[6]{Bischoff2007}. %TODO cite primary source LAICK/Warnecke/Aurich 2003

\subsubsection*{Strategie}
\textbf{Definition:}
Unter einer Strategie werden in der Wirtschaft die langfristig geplanten Aktivitäten sowie zur Erreichung der Unternehmensziele verstanden. %TODO cite Schuh08:12 but need to find a primary source
Eine Anlaufstrategie bezieht sich auf sämtliche Anläufe im Unternehmen und koordiniert die Aktivitäten zur Erreichung der Anlaufziele \cite[4]{Schuh2008}. Innerhalb des Unternehmens ist die Anlaufstrategie der Produktentwicklungs- und Produktionsstrategie untergeordnet und muss die Ziele beider Strategien aufgreifen und integrieren. %TODO cite primary source vonWagenheim1998, sec: Schuh08:12
Wie auch das Anlaufmanagement im Allgemeinen ist die Anlaufstrategie phasen- und funktionsübergreifend \cite{Pfohl2000}. %TODO check citation
Sie sollte in der frühen Phase des Produktentwicklungsprozesses definiert werden \cite{Schuh2004}. 
KUHN et al beschreibt als übergeordnete Ziele die Beherrschung der Qualität und die Reduzierung von Zeit und Kosten \cite[4]{Kuhn2002}. 

\textbf{Bestandteile:}
SCHUH beschreibt die Gestaltung der Strategie in den vier Dimensionen Management von Flexibilität, Komplexität, Qualität und Kosten \cite[13]{Schuh2008}. BISCHOFF nennt zudem die strategische Projektwahl, mit dem sich das Unternehmen auf strategisch wichtige Projekte Fokussieren und die Anzahl parallel abzuwickelnder Anläufe reduzieren kann \cite[43]{Bischoff2007}. 



\subsubsection*{Organisation}
Die Anlauforganisation bildet die zuvor definierte Strategie bzgl. der Serienanläufe in der Unternehmensstruktur ab. Hauptzweck ist, den gestiegenen Anforderungen in Form von zunehmender Dynamik, Abhängigkeiten und Interdisziplinarität mit der Gestaltung einer zweckmäßigen Unternehmensstruktur zu begegnen \cite[55]{Schuh2008}. 

Während die Anlauf-Aufbauorganisation involvierte Bereiche räumlich und formal strukturiert legt die Anlauf-Ablauforganisation die zeitlichen und logischen Beziehungen zueinander fest \cite[55]{Schuh2008}.
Für die Realisierung der Aufbauorganisation werden interdisziplinäre Stablinien- oder Matrixorganisationen eingesetzt \cite[77]{Bischoff2007}. Dabei wird die Matrixorganisation ggf. durch hochqualifizierte Expertenteams unterstützt \cite[4]{Schmitt2015}. %TODO cite primary source Schuh/Kampker/Franzkoch2005 s. 407
%TODO weitere Formen 4 Grundtypen von Anlauorganisation bzgl AUFBAU ORG s. Grafik \cite[58]{Schuh2008} -> ggf. in Anhang
Empfehlenswert ist auch der Einsatz eines Serienanlaufteams, dessen Funktionsweise und Einbindung in die Aufbauorganisation unterschiedlich ausgeprägt sein können \cite[79]{Bischoff2007}. 





\subsubsection*{Planung}

Planung und Simulation des Anlaufs

Definition eines Regelwerks zur überprüfung des Fortschritts z.B. KPI

Präventives RM

\subsubsection*{Steuerung}

Operative Steuerung des A.

Reaktion auf Unvorhersehbare Ergebnisse und Probleme

RM


\subsubsection*{Lieferanten}

Ziele Werte Verhaltensnormen für Zusammenarbeit mit Lieferanten werden gemäß der Vision definiert Schmitt2015

Harmonisierung der Schnittstellen innerhalb der SC mit transparenten unternehmensübergreifende Strukturen Bischoff2007

Gemeinsame Informationsstrategie  Kuhn02

Frühe Einbindung und Integration der Lieferanten bedeutend für reibungslosen Anlauf Bischoff2007 S.28, Kuhn2002 S. 26

Einheitliche Datenbasis für den Austausch von Informationen und Planungsdaten Kuhn02

Werkzeuge: 
  Lieferanten-Audits, KVP, PDCA Schuh08
  FMEA, QFD, Ishikawa, FTA Bischoff2007

\subsubsection*{Logistik}

Die Logistik beinhaltet die Koordinierung aller Material- und Informationsflüsse und Prozesse von Auftrag bis Auslieferung des Endprodukts. Die strategische Ebene beinhaltet die Entwicklung und Gestaltung der Wertschöpfungsnetzwerde und Prozesse nach logistischen Prinzipien. Die operative Ebene beinhaltet die Lenkung und Kontrolle der Material- und Informationsflüsse und der dazugehörigen Prozesse. 
Hauptziele der Logistik ist, durch Gestaltung und Lenkung der logistischen Prozesse die Kundenbedürfnisse in den ökologischen, ökonomischen und sozialen Dimensionen optimal zu erfüllen \cite[28]{Schmitt2015}. 

Die Bedeutung der Logistik für die Anlaufphase ist durch die Globalisierung der Märkte, Hit-Konzepte und Reduzierung der Wertschöpfungstiefe gestiegen. Die Logistik hat zwei spezielle Funktionen in der Anlaufphase. Zum einen muss sie den Materialfluss der ersten Produkte bewerkstelligen. Zum anderen erprobt sie bereits Logistikprozesse für die Serie.
Durch den Querschnittscharakter der Logistik ist eine Abstimmung mit anderen Funktionsbereichen und der Logistik anderer Unternehmen erforderlich \cite[1189]{Pfohl2000}.

\subsubsection*{Kooperationen}

\subsubsection*{Änderungen}

Definition: 
Technische Änderungen sind notwendige nachträgliche Anpassungen an bereits freigegebenen Entwicklungsständen \cite{Zanner2002}. Sie beinhalten immer eine Änderung der Dokumentation bzw. Datenbasis \cite[47]{Niemerg1997}. 	


%   Sekundärquelle: \cite[215]{Schuh2008}
%   cite Primärquelle Niemerg1997 - ZB Grimm-Zentrum 
%   Geschlossenes Außenmagazin 03a 
%   98 HA 8754 vorbestellt ins Campus Nord. 
Produktänderungen können in der Entwicklungs- und Konstruktionsphase bis zu 40\% der Gesamtressourcen beanspruchen \cite{Lindemann1998}
%TODO cite Lindemann1998 -> TUB QP624 77
Änderungsmanagement soll die Termintreue der Prozesse im Serienanlauf sicherstellen und die Durchlaufzeiten reduzieren \cite[216]{Schuh2008}. 

Ursachen: 
Auslöser für Änderungen können Gesetzesänderungen, interne Fehler, Qualitäts- und Sicherheitsprobleme, veränderte Kundenwünsche sowie eine veränderte Markt- und Wettbewerbssituation sein \cite{Zanner2002}. Auch treten Probleme oft erst dann in Erscheinung, wenn sie im Kontext der benachbarten Komponenten stehen \cite[24]{Kuhn2002}.

Konsequenzen: 
Änderungen bringen Konsequenzen mit sich. So führen sie zu steigendem Zeitdruck, einem erhöhten Personalaufwand in planerischen Abteilungen sowie können Kosten und Zeitverzögerungen aufgrund von Werkzeugänderungen entstehen \cite[24]{Kuhn2002}. 

Lösungsansatz und Bestandteile: 
Um den zeitlichen und finanziellen Aufwand gering zu halten, sollten Änderungen vermieden oder möglichst vorverlagert werden \cite{Schuh2008, Jania2004, Ass98}. 
% Schuh2008:215, Jania2004:69f, Ass98:107--131
%TODO Primärquelle Ass98 - TUB QP624 77
SCHUH teilt das Änderungsmanagement in Änderungsplanung, -ausführung und -absicherung ein \cite[217]{Schuh2008}. 
%TODO cite original author Florian Rösch et. al.
LINDEMANN hingegen unterteilt das Thema detaillierter in Vermeidung, Früherkennung, Problemanalyse, Lösungsfindung, Bewertung und Entscheidung. Die Erkenntnisse werden mit Hilfe einer sog. Lernorientierten Auswertung im Sinne eines KVP ausgewertet \cite{Lindemann1998}. 
%TODO cite Primärquelle
%TODO KVP glossary, ausgewertet synonym

Enabler: 
Als Schlüsselrolle für erfolgreiches Änderungsmanagement wird oft die Kommunikation von Problemen und Änderungen innerhalb und über Unternehmensgrenzen hinweg genannt \cite{Kuhn2002, Schuh2008}.
% Kuhn2002:28+24, Schuh08:219
ZANNER betont die Bedeutung der Vertrauensverhältnisses für den Informationsaustausch und schlägt informelle standortübergreifende Treffen der Entwickler vor. Die Zuordnung eines Verantwortlichen Mitarbeiters für die Abwicklung einer Änderung soll helfen, die Schnittstellenprobleme bei der Arbeitsteiligen Arbeitsweise zu überwinden  \cite[42]{Zanner2002}.
Weiterhin werden eine einheitliche Terminologie \cite{Zanner2002} und Datenbasis sowie ein durchgängiges Versionsmanagement \cite{Kuhn2002} als Erfolgsfaktoren genannt. 
% Kuhn: Datenbasis S.25, Versionsmanagement S. 25

\subsection{Wissen und Personalmanagement}

Definition: 
Die Themenkomplexe Wissensmanagement und Personalmanagement sind zusammenhängend zu betrachten. KUHN betont den Einfluss der am Anlauf beteiligten Mitarbeiter auf den Ablauf und Erfolg des Projekts. Die Erfolgsfaktoren setzen sich zum einen aus Mitarbeiterqualifikation und -motivation, und zum anderen aus Wissenserfassung, -visualisierung und -weitergabe zusammen \cite[31]{Kuhn2002}. 

Motivation: 

Bestandteile: 

Enabler: 

% \subsection{Produktion}
% 
% \subsection{Produktentwicklung}
% 
% \subsection{Kostenmanagement}
% 
% \subsection{Qualitätsmanagement}
% 
% \subsection{Risikomanagement}
% 
% \subsection{Vertrieb und Marketing}



% \cite*[prenote][postnote]{Schmitt2015}[extra] cite
% 
% \Cite*[prenote][postnote]{Schmitt2015}[extra] Cite
%  
% \footcite[prenote][postnote]{Schmitt2015}[extra] cite footcite
% 
% \Footcite[prenote][postnote]{Schmitt2015}[extra] cite Footcite
% 
% \textcite[prenote][postnote]{Schmitt2015}[extra] cite textcite
% 
% \citeauthor[prenote][postnote]{Schmitt2015}[extra] citeauthor
% 
% \citeauthor{Schmitt2015} Citeauthor
% 
% \citet{Schmitt2015} citet
% 
% \citep{Schmitt2015} citep
% 
% \autocites{Schmitt2015} autocites
% 
% \parencite{Schmitt2015} parencite