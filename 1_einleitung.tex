% \section{Einführung}
\chapter{Einführung}
\section{Motivation \& Problemstellung}
Die produzierende Industrie findet sich heutzutage in einem zunehmend dynamischen Wettbewerbsumfeld wieder, welches vielschichtige Herausforderungen mit sich bringt \cite{Renner2012}. Die hauptsächlichen Herausforderungen liegen in steigenden Innovationsgeschwindigkeiten, kürzeren Produktlebenszyklen und einer höheren Variantenvielfalt \cite{Kuhn2002,Stauder2016}. Um dem durch die Globalisierung verstärkten Wettbewerb standzuhalten, müssen produzierende Unternehmen innovative Produkte und Dienstleistungen anbieten und sich zunehmend kundenorientiert aufstellen \cite{Surbier2014}. 
Eine zentrale Rolle wird hier dem Anlauf von Serienprodukten zugeschrieben. Aufgrund immer kürzer werdender Produktlebenszyklen rücken Kosten und Zeitaufwand in den Vordergrund \cite{Winkler2007}. So hat der Anlauf einen signifikanten Einfluss auf den wirtschaftlichen Erfolg des Produkts und die Time-to-Volume \cite{Klocke16}. Selbst ein um wenige Monate verschobener Verkaufsstart kann über Erfolg oder Misserfolg des Produkts entscheidend sein \cite{Schuh2008a}. Die Bedeutung der Serienanläufe findet bisher in der Wissenschaft keine angemessene Aufarbeitung \cite{Dyckhoff2012}. 

\section{Fokus der Arbeit}
Der Trend zur Konzentration auf Kernkompetenzen sorgt dafür, dass in großen Unternehmen immer mehr Wertschöpfungsanteile an Zulieferer abgegeben werden  \cite{Hilmola2015, Wildemann2008}. Der Gesamtanlauf setzt sich fortan aus vielen lokalen Einzelanläufen zusammen \cite{Zimolong2006}. Daraus resultieren höhere Abhängigkeiten zwischen größeren Unternehmen und den Zulieferern, die meist mittelständische Unternehmen sind. 

Die Abschlussarbeit soll sich im Speziellen mit dem Serienanlauf im KMU und SME als Zulieferer für größere Unternehmen beschäftigen, da hier erhebliches Verbesserungspotential erkennbar ist \cite[S.18]{Dombrowski2009a}. So gibt es in KMU meist keine Anlaufprozesse. Da es in KMU oft keine Stabsstellen gibt, werden Anläufe von den Mitarbeitern oft zusätzlich zum Tagesgeschäft gesteuert \cite{Dombrowski2009}. %TODO kein Zugriff auf Primärquelle D.Spath!! 
Mangelnde finanzielle und zeitliche Kapazitäten sowie fehlendes Know-how verhindern eine nachvollziehbare Dokumentation sowie proaktive Maßnahmen \cite{Zimolong2006,Dombrowski2009a}. 

Weiterhin soll untersucht werden, wie der Auftraggeber den Anlaufprozess des Lieferanten unterstützen kann. Größere Unternehmen verfügen in der Regel über mehr Ressourcen und teilweise eigene Anlaufprozesse. Im Zuge der Verlagerung der Wertschöpfungsanteile, gewinnt die Innovationskraft von Modul- und Systemlieferanten zunehmend an Bedeutung für den Erfolg eines Produktes \cite{Kuhn2002}. Ein erfolgreiches und effizientes Anlaufmanagement in KMU ist im Sinne der Entwicklung einer nachhaltigen Partnerschaft für Auftraggeber und Lieferant von großer Bedeutung. \textit{Wildemann} erkennt hier das Potenzial von Einspareffekten sowie Nutzung erheblicher Wettbewerbsvorteile auf beiden Seiten \cite{Wildemann2008}.

Ziel der Arbeit ist, einen Überblick über den Stand der Forschung zu geben und mögliche Forschungsfelder aufzuzeigen. 

\section{Herangehensweise}
Die Abschlussarbeit wird eine Literaturarbeit. Zunächst wird nach festgelegten Kriterien Literatur zu folgenden Themenkomplexen gesucht: Anlaufmanagement in KMU / SME, Anforderungen an KMU / SME, Anlaufmanagement allgemein und Lean Management Methoden.

Es folgt eine Darstellung zum Stand der Wissenschaft. \textit{Dyckhoff} und \textit{Scholz} sind zu der Erkenntniss gekommen, dass das Thema weder in Industrie noch in der Wissenschaft hinreichend Beachtung findet \cite{Dyckhoff2012, Scholz2010}, weshalb hier keine zufriedenstellenden Ergebnisse zu erwarten sind. In einem weiteren Schritt werden die Theorie des Anlaufmanagements, Methonden der Lean Philosophie und die Anforderungen der KMU skizziert. 
Darauf aufbauend werden die zuvor dargestellten Themen mit den Anforderungen der KMU gegenübergestellt und mögliche Lösungsansätze entwickelt. In einer Diskussion werden die Lösungsansätze gegenübergestellt und zukünftige Forschungsfelder ermittelt. 

Die Validierung der Ergebnisse erfolgt durch Zitierung der Quellen. Auf eine Validierung durch Experten, Fragebögen oder empirische Untersuchungen wird aufgrund des großen Umfangs verzichtet. 
%
