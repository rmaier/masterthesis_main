% \section{Einführung}
\chapter{Einführung}\label{sec:einfuehrung}
Dieses Kapitel beginnt mit einer Herleitung von der Motivation bis hin zu der Entwicklung der Forschungsfragen. Nachdem der Aufbau der Arbeit kurz skizziert wurde, folgen die zum Verständnis und zur Beantwortung der Forschungsfragen notwendigen Definitionen. 

\section{Motivation \& Problemstellung}
Die produzierende Industrie findet sich heutzutage in einem zunehmend dynamischen Wettbewerbsumfeld wieder, welches vielschichtige Herausforderungen mit sich bringt \autocite{Renner2012}. Die hauptsächlichen Herausforderungen liegen in steigenden Innovationsgeschwindigkeiten, kürzeren Produktlebenszyklen und einer höheren Variantenvielfalt \autocite{Kuhn2002,Stauder2016}. Um dem durch die Globalisierung verstärkten Wettbewerb standzuhalten, müssen produzierende Unternehmen innovative Produkte und Dienstleistungen anbieten und sich zunehmend kundenorientiert aufstellen \autocite{Surbier2014}. 
Eine zentrale Rolle wird hier dem Anlauf von Serienprodukten zugeschrieben. Aufgrund immer kürzer werdender Produktlebenszyklen rücken Kosten und Zeitaufwand in den Vordergrund \autocite{Winkler2007}. So hat der Anlauf einen signifikanten Einfluss auf den wirtschaftlichen Erfolg des Produkts und die Time-to-volume \autocite{Klocke16}. Selbst ein um wenige Monate verschobener Verkaufsstart kann über Erfolg oder Misserfolg des Produkts entscheidend sein \autocite{Schuh2008a}. Die Bedeutung der Serienanläufe findet bisher in der Wissenschaft keine angemessene Aufarbeitung \autocite{Dyckhoff2012}. 

\section{Fokus der Arbeit}

Der Trend zur Konzentration auf Kernkompetenzen sorgt dafür, dass in großen produzierenden Unternehmen immer mehr Wertschöpfungsanteile an Zulieferer abgegeben werden  \autocite{Hilmola2015, Wildemann2008}. Der Gesamtanlauf setzt sich fortan aus vielen lokalen Einzelanläufen zusammen \autocite{Zimolong2006}, wodurch sich die Komplexität erhöht. 
Die höhere Variantenvielfalt sorgt bei gleichzeitig kürzeren Produktlebenszyklen für geringere Gesamtstückzahlen pro Produkt. Durch diesen Umstand versuchen nun auch kleinere Unternehmen in den Markt zu drängen.
% höhere Variantenvielfalt -> kleinere Produktionsgrößen -> kleine U. drängen in den Markt - LSU
An dieser Stelle setzt das Konzept des Lean Start-up (\gls{lsu}) an. Das Lean Start-up ist eine Businessmethode für dynamische Unternehmen oder Projekte, die hohen Risiken und Unsicherheiten ausgesetzt sind. 
%TODO cite!! possible sources: Edison2015a/2, Eisenmann2013/1 . -> search notes at home about 'Unternehmen und Projekte'
Im Jahre 2008 führte RIES zum ersten Mal den Begriff Lean Start-up ein \autocite{Ries2008}. 
Das Konzept des Anlaufmanagements befasst sich mit der Planung, Durchführung und Steuerung des Serienanlaufs \autocite[S.8]{Kuhn2002}. Hauptziele sind die Beherrschung und die zeitliche Verkürzung der Anlaufphase \autocite{Kuhn2002, Schmitt2015}.

Bei der Analyse der Literatur zu \gls{lsu} lässt sich feststellen, dass der Themenkomplex Anlaufmanagement bisher noch nicht abgebildet wird. Da jedoch eine Beherrschung reibungsloser Serienanläufe ein entscheidender Wettbewerbsvorteil ist, sind hier erhebliche Potentiale für das \gls{lsu} zu erwarten \autocite[S.XI]{Bischoff2007}. Darauf basierend lässt sich folgende Hypothese für die Arbeit ableiten: 

% \textbf{Hypothese}:
\begin{quotation}
Der Themenkomplex Anlaufmanagement findet in der Businessmethode Lean Start-up keine angemessene Beachtung. Die Beherrschung eines reibungslosen Serienanlaufs ist jedoch ein erheblicher Wettbewerbsvorteil. 
% DONE source Reinfelder2004 - discarded because of already cited Bischoff2007
\end{quotation}
Basierend auf der Hypothese werden folgende Forschungsfragen aufgestellt, die in der Abschlussarbeit beantwortet werden müssen: 

\begin{description}

\item[FF 1] Wie kann der Serienanlauf im \gls{lsu} gestaltet werden? 

\item[FF 1.1] Was zeichnet das \gls{lsu} mit Hinblick auf das Anlaufmanagement aus? Welche Anforderungen werden gestellt?

\item[FF 1.2] Welche Aspekte des Anlaufmanagements sind für das \gls{lsu} von Bedeutung? 

\item[FF 1.3] Wie könnte ein Anlaufmanagement-Ansatz für das \gls{lsu} auf Basis des Stands der Wissenschaft zum Thema Anlaufmanagement aussehen?  
\end{description}



\section{Herangehensweise}
Die Abschlussarbeit wird eine Literaturarbeit. In der Einführung erfolgt eine knappe Darstellung der zu behandelnden Themen Lean Start-up und Anlaufmanagement. Im Hauptteil wird zunächst der Stand der Wissenschaft zum Thema Lean Start-up skizziert. Den größeren Teil bildet eine umfassende Literaturanalyse zum Stand der Wissenschaft des Anlaufmanagements. Die Literaturrecherche erfolgt nach fest definierten Kriterien. Für die Literaturanalyse werden mit Hilfe des Tools \textit{Atlas.ti} alle relevanten Textstellen gecoded, d.h. identifiziert und nachvollziehbar dokumentiert. Anhand der  Ergebnisse wird anhand von 15-20 Quellen der Stand der Wissenschaft dargestellt. Im nächsten Abschnitt werden für das Lean Start-up nicht berücksichtigte Anforderungen an das Anlaufmanagement ermittelt und daraus eine Art Anlaufmodell abgeleitet. 
%
Die Validierung der Ergebnisse erfolgt durch Zitierung der Quellen. %Auf eine Validierung durch Experten, Fragebögen oder empirische Untersuchungen wird aufgrund des großen Umfangs verzichtet. 

\section{Inhaltlicher Aufbau der Arbeit}
Die Arbeit beginnt mit Kapitel \ref{sec:einfuehrung}, in dem Motivation und Zielsetzung der Arbeit dargelegt werden. Der Untersuchungsgegenstand der Arbeit wird aufgezeigt und die Forschungsfragen entwickelt. Weiterhin erfolgen zum Verständnis der Aufgabenstellung eine erste Definition zum Thema Anlaufmanagement sowie ein Überblick zu Lean Start-up.

In Kapitel \ref{sec:methodik} wird die methodische Herangehensweise der Arbeit unter Berücksichtigung von Suchstrategie und Forschungsmethodik dargelegt. 
% TODO Wirklich? 
Weiterhin werden spezifische Anforderungen des Lean Start-up an das Anlaufmanagementmodell entwickelt, welche bei der Ausarbeitung berücksichtigt werden sollen. Schließlich wird ein Grundgerüst für das zu entwickelnde Anlaufmanagementmodell erarbeitet, welches als struktureller und inhaltlicher roter Faden der Arbeit dient. 
%DONE Rechtschreibung roter Faden! Siehe auch weiter unten
Die Entwicklung des Grundgerüsts dient auch der Konsolidierung der verschiedenen Auffassungen des Themengebiets, da in der Wissenschaft bisher noch keine einheitliche Auffassung herrscht. 

In Kapitel \ref{sec:durchfuehrung} werden mittels qualitativer Literaturanalyse und anhand des zuvor entwickelten Grundgerüsts Lösungskonzepte zusammengetragen. 

Die Ergebnisse werden in Kapitel \ref{sec:ableitung} zusammengefasst. Darauf basierend wird ein Umsetzungsleitfaden entwickelt, der die Implementierung des Anlaufmanagementmodells anhand von spezifisch zu beantwortenden Fragen im Unternehmen unterstützen soll. 

Die Arbeit mündet in Kapitel \ref{sec:fazit} mit einem Fazit und Ausblick. Ein besonderes Augenmerk liegt auf der Bewertung der zur Beantwortung der Forschungsfragen angewandten Methoden. Im Ausblick werden Bereiche für weitere Forschung aufgezeigt. Schließlich werden kritische Faktoren für eine erfolgreiche Umsetzung im Unternehmen identifiziert. 

\section{Kontext}
In diesem Abschnitt werden die zum Verständnis und zur Beantwortung der Forschungsfragen notwendigen Definitionen dargestellt. 
\subsection{Lean Start-up}\label{sec:lsu}
\subsubsection*{Einführung}
Das Lean Start-up ist eine Businessmethode für dynamische Unternehmen oder Projekte, die hohen Risiken und Unsicherheiten ausgesetzt sind. 
Hauptziele der Methode sind kürzere Entwicklungszeiten, Einsparung von Kosten in der Entwicklungsphase und frühzeitiges Erkennen der Kundenbedürfnisse. 
Sie ist eine Antwort auf unbekannte Problemstellungen und Lösungen, hoch dynamische Märkte und hohe Risiken. Die Ursprünge liegen in den Denkweisen von Taiichi Ōno, W. Edwards Deming und Peter Drucker. 
2008 übertrug Eric Ries Lean Produktionsmethoden auf Hochtechnologie Start-ups und veröffentlichte 2011 die erstmals ``Lean Startup'' 
genannte Methode in seinem Buch.
%DONE 2011 or 2008? 
%DONE Leerzeichen Startup``genannte Merthode

%\subsection*{Definitionen}

\subsubsection*{Bestandteile}
\begin{enumerate}
 \item \textbf{Entwickeln einer Vision}. Die Vision dient als Grundlage für alle weiteren Handlungen. Aus ihr werden im nächsten Schritt Hypothesen abgeleitet. Anstatt einen aufwändigen Businessplan zu erstellen, wird die Vision in einem Business Model Canvas (\gls{bmc}) definiert \autocite{Blank2013}. Die Vision eines Lean Start-up zeichnet sich durch viele Freiheitsgrade und Unsicherheiten aus. 

\item \textbf{Überführen der Vision hin zu Hypothesen}. Für jedes Element der im Business Model Canvas beschriebenen Vision werden Hypothesen abgeleitet. Die Hypothesen bilden die Freiheitsgrade und Unsicherheiten des \gls{bmc} ab. Ziel ist, die Risiken durch spätere Beantwortung der Hypothesen zu minimieren. Nach Möglichkeit sollen die Hypothesen so formuliert werden, dass sie quantitativ beurteilt werden können. Um neue Erkenntnisse gewinnen zu können, müssen die Hypothesen widerlegbar sein. 

\item \textbf{Entwickeln von \gls{mvp} Tests}. Ein minimal überlebensfähiges Produkt (\gls{mvp}, engl.: Minimum Viable Product) ist ein Werkzeug, mit dem man schnellstmöglich die Hypothesen am Kunden überprüfen kann \autocite[93]{Ries2011}. Ziel ist zum einen den Build-Measure-Learn Zyklus zu beschleunigen, zum anderen die Lernrate in Bezug auf den Aufwand zu maximieren. So können frühzeitig nicht benötigte Funktionen und Produkteigenschaften erkannt und Zeit und Kosten gespart werden. Wenn die Entwicklung eines realen \gls{mvp} zu aufwändig ist, kann ein Smoke Test eingesetzt werden. In einem Smoke Test wird das zukünftige Produkt in einem Video oder über eine Webseite vorgestellt.

\item \textbf{Planung der Tests}. Bei der Durchführung der Tests kommt es darauf an, Kosten und Zeit zu minimieren. Daher werden zuerst Tests durchgeführt, die kostengünstig sind und hohe Risiken untersuchen. \Gls{bspw} ist eine Patentrecherche kostengünstig und kann frühzeitig sehr hohe Risiken aufdecken. Tests können nacheinander (seriell) oder gleichzeitig (parallel) durchgeführt werden. Bei parallelen Tests riskiert man im Gegensatz zu seriellen Tests, dass einzelne Tests überflüssig werden. Man profitiert jedoch von einem Zeitvorsprung gegenüber der seriellen Vorgehensweise. 

\item \textbf{Interpretation der Ergebnisse}. Bei der Interpretation der Ergebnisse gibt es einige Fehlerquellen. Zum einen gibt es teilweise große Differenzen zwischen den geäußerten und reellen Kundenrückmeldungen. Zum anderen kann die Interpretation des Unternehmers durch eigene Wünsche oder Erwartungen verzerrt sein.

\item \textbf{Reaktion}. Nach Auswertung der Ergebnisse sieht die \gls{lsu} Methode eine Entscheidung zwischen drei Handlungsalternativen vor. \textit{Preserve}: Wenn die Tests die Hypothesen bestätigen, wird die Strategie beibehalten. \textit{Pivot}: Wenn die Tests die Hypothesen widerlegen oder neue Chancen aufzeigen, wird die Strategie angepasst. \textit{Perish}: Wenn die Tests die Hypothesen widerlegen und der Unternehmer keine geeignete Strategie entwickeln kann, wird die Strategie verworfen. 

\item \textbf{Skalierung und kontinuierliche Verbesserung}. Sobald alle relevanten Hypothesen bestätigt wurden, ist das Produkt auf den Markt abgestimmt. Nun kann massiv in Kundenakquise und Produktentwicklung investiert werden. Wichtig ist weiterhin, dass die Strategie fortwährend überprüft wird. Ein \textit{Pivot} ist auch nach der Skalierung bei größeren Änderungen sinnvoll. 
\end{enumerate}


% \subsection*{Grenzen der Methodik}

\subsection{Anlaufmanagement}
Immer kürzere Produktlebenszyklen bei gleichzeitig wachsenden Kundenanforderungen und größerer Variantenvielfalt erhöhen die Komplexität und somit die Bedeutung des Serienanlaufs \autocite{Kuhn2002,Schuh2004}. Die Risiken im Zusammenhang mit der Anlaufphase sind vielfältig. KUHN 
%DONE Biblatex Cite command for cap letter author in maintext
stellt fest, dass der Aufwand bis zum Erreichen einer stabilen Produktion oft unterschätzt wird. Infolgedessen kann es zum verspäteten Markteintritt sowie zu unzureichenden Kapazitäten und Qualitätsmängeln kommen \autocite{Kuhn2002}. Um diesen Risiken entgegen zu wirken, werden als übergeordnete Hauptziele für das Anlaufmanagement Beherrschung und zeitliche Verkürzung der Anlaufphase genannt \autocite{Kuhn2002, Schmitt2015}. 

Produktionsanläufe stellen auch deshalb eine große Herausforderung für Unternehmen dar, da sie hochkomplex sind und sich durch viele parallele und sequenzielle Teilprozesse auszeichnen. Sie sorgen zudem für eine starke Vernetzung der beteiligten Abteilungen innerhalb und außerhalb des Unternehmens \autocite{Schuh2004}.


\subsubsection*{Definition}
In der Literatur existiert keine einheitliche Definition des Begriffs Anlaufmanagement \autocite[4]{Bischoff2007}. Selbst SCHMITT 
%DONE 
bemängelte 2015 ein fehlendes einheitliches Verständnis der grundlegenden Begriffe des Produktionsanlaufs \autocite[1]{Schmitt2015}. Vielmehr existieren unternehmensintern und teilweise auch projektspezifisch unterschiedliche Auffassungen über die Definition der Anlaufphase \autocite[11]{Grosshenning2005}. KUHN 
%DONE verbalization
definierte das Anlaufmanagement wie folgt \autocite[8]{Kuhn2002}: 
\begin{quotation}
Das Anlaufmanagement eines Serienproduktes umfasst alle Tätigkeiten und Maßnahmen zur Planung, Steuerung und Durchführung des Anlaufs mit den dazugehörigen Produktionssystemen, ab der Freigabe der Vorserie bis zum Erreichen einer geplanten Produktionsmenge, unter Einbeziehung vorgelagerter Prozesse und der nachgelagerten Prozesse im Sinne einer messbaren Eignung der Produkt- und Prozessreife.
\end{quotation}
SCHUH übernahm diese Auffassung \autocite{Schuh2008} während RISSE und BISCHOFF den Beginn bereits nach der abgeschlossenen Produktentwicklung sehen (Freigabe Pflichtenheft) \autocite{Risse2002, Bischoff2007}.

Der Anwendungsbereich beschränkt sich nicht nur auf den Anlauf von neuen Produkten. Auch Modellderivate (Modellpflege), Varianten, neue Produktionssysteme, Fertigungsverfahren und Logistikprozesse stellen aus Perspektive des Managements einen Anlauf dar \autocite[53]{Laick2003}. 
%
% \autocite[6]{Bischoff2007}. 
% DONE cite primary source LAICK/Warnecke/Aurich 2003 S. 53
% Laick, Thomas ; Warnecke, Günter ; Aurich, Jan C.: Hochlaufmanagement :
% Sicherer Produktionshochlauf durch zielorientierte Gestaltung und Lenkung des
% Produktionsprozesssystems. In: PPS Management Heft 2 (2003), S. 51-54. –
% ISSN 1434-2308
% (TU Magazin: 8TA2365) ??

