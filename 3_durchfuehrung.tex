\chapter{Durchführung}

\section{Reichwald-2004}
% Kommentar: 
% Der Ansatz des Feedback-Loops passt schon gut zum LSU Konzept. Hervorzuheben ist z.B. der zielgerichtete Einsatz von Software, auch wenn nicht Abteilungen oder Prozesse zu überwinden sind. Betont werden soll jedenfalls, dass der Einsatz von Software hier nur unterstützend ist und den Menschen nicht von seiner Verantwortung entbindet. 
REICHWALD et al. untersucht das Projektmanagement im Feldanlauf und fokussiert sich dabei auf ein nachhaltiges Wissensmanagement \cite{Reichwald2004}. Ein durchgängiges und effizientes Wissensmanagement ist ein wesentlicher Bestandteil erfolgreicher Anläufe \cite{Kuhn2002}. % TODO check primary source Kuhn2002

REICHWALD et al. unterteilt den Wissensmanagementprozess in folgende Bestandteile: Identifizierung der Wissenslücken, Wissenserwerb und Wissensentwicklung, Wissensverteilung und Wissensbewahrung. Die ersten vier Bestandteile werden in Folgendem kurz vorgestellt. 

\subsection{Identifizierung der Wissenslücken}
Bereits vor Markteinführung muss bekannt sein, inwieweit das Produkt den Kundenerwartungen entspricht. Besonders die Produktmerkmale, die unterhalb der Kundenerwartungen liegen müssen identifiziert werden. Dazu eignen sich realitätsnahe Produkttests mit einer dem zukünftigem Kundenkreis entsprechenden Gruppe. Es sind geeignete Erhebungsinstrumente auszuwählen und weiterzuentwickeln. Um möglichst viele Qualitätsaspekte zu berücksichtigen müssen die Erfahrungen der Testpersonen über ein breites Spektrum erfasst werden. Darüber hinaus müssen die Ergebnisse einfach auszuwerten sein um eine schnelle Berücksichtigung zu gewährleisten. Dazu eignen sich z.B. standardisierte Fragebögen oder kurze mündliche Befragungen. 

\subsection{Wissenserwerb, -entwicklung und -verteilung}
Nach erfolgter Produkttests werden die Ergebnisse ausgewertet. Die einzelnen Ergebnisse werden in ein sogenanntes E-Workflowsystem eingepflegt, welches eine Art \gls{erp} System darstellt.
Dieses System ist in der Lage das gesammelte Faktenwissen entlang der gesamten Prozesskette bereitzustellen. Die bei der Auswertung der Tests gewonnenen Schwerpunkte bilden die Handlungsfelder des Anlaufteams. Mit Hilfe des E-Workflowsystems werden den jeweiligen Handlungsfeldern Maßnahmen und Zuständigkeiten sowie Umsetzungstermine zugeteilt. 
Nach der Umsetzung der Maßnahmen muss die Wirksamkeit möglichst durch die gleichen Personen bestätigt werden, die im Vorfeld die Handlungsfelder aufgezeigt haben. Wird die Wirksamkeit bestätigt, ist das Handlungsfeld erfolgreich abgeschlossen. 
Sind Erkenntnisse des laufenden Projekts auch für zukünftige Projekte von Bedeutung, so sollten sie im E-Workflowsystem gesondert gekennzeichnet und in zukünftigen Entwicklungsprozessen eingegliedert werden. 


\subsection{Kurzzusammenfassung}

Ein durchgängiges und effizientes Wissensmanagement ist ein wesentlicher Bestandteil erfolgreicher Anläufe. Dabei werden frühzeitig Kundenrückmeldungen zur Produktverbesserung ausgewertet und die Arbeit mit Softwaresystemen unterstützt. 

\section{Harjes-2004}

HARJES et al. untersucht das Anlaufmanagement mit besonderer Berücksichtigung des Produktentstehungprozesses \cite{Harjes2004}. 

\subsection{Robuste Produktionssysteme}

Höhere Variantenvielfalt und Individualisierungswünsche der Kunden stellen hohe Anforderungen an Fertigungs- und Montagelinien. Zunächst ist eine Standardisierung erforderlich. Produktionssysteme sollten einfach und übertragbar gestaltet werden. Daraus erfolgt eine erhöhte Flexibilität bei Integration neuer Baureihen und Komponenten, Änderungen können somit reibungsloser implementiert werden. Um Auswirkungen vom Prozess oder Produkt auf das Produktionssystem frühzeitig bewerten zu können, sind Prozess- und Produktdaten standardisiert zu verknüpfen und stets aktuell zu halten. 

\subsection{Produktdatenmodell}
Die stetige Reduzierung der eigenen Wertschöpfungstiefe erfordert eine hohe Transparenz bzgl. der Produktdaten. Dies erfolgt mit dem Aufbau digitaler Produktdatenmodelle, welche stets einen echten, plausiblen und aktuellen Datenstand aufweisen müssen. 
Integrierte Produktdatenmodelle (\gls{ipdm})
verknüpfen Produkt- und Prozessdatenmodelle. Damit bekommen Änderungen mehrdimensionalen Charakter, betroffene Komponenten können identifiziert werden und die Folgen lassen sich simulieren und bewerten. 
Weiterführend wird die digitale Fabrik genannt, die die digitale Planung einer Fertigungsfabrik mit Integration aller Produkt- und Prozessdaten beschreibt. 

\subsection{Kurzzusammenfassung}
Robuste Profuktionssysteme reagieren agil auf Änderungen und können flexibel erweitert werden. Digitale Produktdatenmodelle sorgen Unternehmensübergreifend und -intern für erhöhte Transparenz und bessere Folgenabschätzung von Änderungen. 

\section{Straub-2006}

STRAUB et al. verfolgt die Vision der Umstellung der Produktion von \gls{sop} 
auf Kammlinie an einem Wochenende. Im Fokus seiner Arbeit steht die schnelle und richtige Reaktion auf ungeplante Störungen im Anlauf \cite{Straub2006}. Die bisher eingesetzte präventive Methode der digitalen Fabrik erhöht zwar signifikant den Reifegrad der Planung, bietet jedoch keine Antwort aud verbleibende ungeplante Störungen. STRAUB beschreibt drei Säulen zur Realisierung kürzerer Anläufe: Einsatz von Anlaufteams, die organisatorische Einbindung der Teams in die Organisation und der Einsatz eines Methodenbaukastens. Letzterer ist für das \gls{lsu} von Beteudung. 

\subsection{Methodenbaukasten}

Grundgedanke des Methodenbaukastens ist der Einsatz moderner Methoden, Werkzeuge und Standards.
Zum einen wird eine erhöhte Effizienz und Transparenz bewirkt. Zum anderen wird eine objektive Bewertung von Situationen und damit ein einheitliches Verständnis erreicht, was insbesondere die Zusammenarbeit mit jüngeren und unerfahrenen Mitarbeitern erleichert. 
Des weiteren wird der Einsatz einer Scorecard empfohlen. %TODO Bild 6 in Anhang oder hier. 
Zunächst werden quantifizierbare Anlaufindikatoren definiert. Mit Hilfe der Scorecard werden die wichtigsten Anlaufindikatoren kontinuierlich überwacht und Abweichungen vom Soll Wert werden schnell erkannt. Es folgt eine systematische Ursachenanalyse. So kann eine schnelle und zielgerichtete Reaktion gewährleistet werden. 

\section{Berg-2006}

\section{Quasdorff-2016 - Lean Management und Digitale Fabrik}

QUASDORFF et al. behandelt die Schnittmengen von Lean Management und der Digitalen Fabrik \cite{Quasdorff2016}. 

Die Digitale Fabrik umfasst die Abbildung und Simulation von Produkt, Prozess und Ressourcen in einem Informationssystem. Während für die Digitale Fabrik die Datenbasis für den Erfolg ausschlaggebend ist, muss die Lean Philosophie aktiv im Unternehmen gelebt werden. Beim gleichzeitigen Einsatz beider Methoden sind große Synergieeffekte zu erwarten. 

Die Digitale Fabrik unterstützt die Vermeidung von Muda (Verschwendung), Mura (Unausgeglichenheit) und Muri (Überbeanspruchung). Durch die zunehmende Digitalisierung (Industrie 4.0 bzw. \gls{iot}) wächst die Bedeutung von Quellen der Verschwendung im Bereich der Informationstechnik und der Datenverarbeitung. 

Bei der Gestaltung der Digitalen Fabrik müssen einige Aspekte beachtet werden. So ist die konsequente Anwendung von Lean Prinzipien Voraussetzung für die Digitale Fabrik. Schlanke Prozesse und deren Standardisierung sorgen dafür, dass die Komplexität der Modelle der Digitalen Fabrik beherrschbar wird. 
Verbesserungen an ineffizienten Prozessen sollten am Prozess als solchen ansetzen anstatt verbesserte Technologie einzusetzen. % TODO cite [8]

Erfolgsfaktoren sind eine hohe Detailtreue und Datenqualität. Das Modell sollte zu jedem Zeitpunkt der Realität entsprechen. Dennoch sollte vor einer Änderung der Ist-Zustand mit dem Dokumentationszustand verglichen werden. 

Abschließend ist zu bemerken, dass der Einsatz der Digitalen Fabrik den Gang in den Shop Floor nicht ersetzen sondern nur unterstützen kann. 

\textbf{Einordnung:} Der Artikel liefert Ansätze für die Gestaltung der DF im LSU. Dabei muss stets auf den Angemessenen Einsatz der Methoden geachtet werden. Ggf. sollten nur einige kritische Elemente Einzug in die DF erhalten. Dennoch muss das Modell zu jedem Zeitpunkt aktuell und plausibel sein. Auch die Detailtreue muss im Anfangsstadium nicht zwingend maximiert werden sondern den Zweck erfüllen einen hohen Reifegrad in der frühen Planungsphase zu erreichen und viele Entscheidungen möglichst früh treffen zu können. 

\section{Schwarz-2017 - Reifegradmodell für Lean Production}

SCHWARZ et al. entwickelt ein Reifegradmodell zur Bewertung des Implementierungsfortschritts von Lean Production im Unternehmen \cite{Schwarz2017}. Einfache Befragungen eignen sich aufgrund der Komplexität nicht zur Bewertung. 

\subsection{Bestandteile}
Das von SCHWARZ entwickelte Reifegradmodell erfasst den Fortschritt in den zwei Dimensionen Methodenkompetenz und Unternehmenskultur. 
Methodenkompetenz beschreibt die Fähigkeit eines produzierenden Unternehmens die Prinzipien der Lean Philosophie durch Anwendung spezifischer Methoden systematisch und gezielt im Produktionssystem umzusetzen. 
Die folgenden fünf Lean Prinzipien dienen als Grundlage für das Modell: Kundennutzen, Wertstrom, Fluss, Pull und Perfektion. 
Es existieren zahlreiche Methoden die jeweils ein oder mehrere Prinzipien umsetzen. 

Voraussetzung für den nachhaltigen Einsatz von Lean Prinzipien ist das aktive Leben der Ideen sowie die Verankerung in der Unternehmenskultur. % TODO Cite [7] Baumgärtner (nicht verfügbar), alt. Quelle suchen
Die Unternehmenskultur wird mithilfe folgender Aspekte beschrieben: 
\begin{itemize}
 \item Grundlegende Annahmen und Überzeugungen
\item Implizite und explizite Werte 
\item Mittel zur Verwirklichung dieser Werte
\item die Außenwirkung.
\end{itemize}
Die Ausprägungen der Aspekte sind Voraussetzungen für eine nachhaltige und langfristige Anwendung der Methoden durch die Mitarbeiter und somit für die Implementierung der Lean Prinzipien. 

\textbf{Gestaltung}

Die Bewertung der Reifegrade in den zwei Dimensionen erfolgt in sechs Stufen. % TODO siehe Grafik. 


Für die Dimension Methodenkompetenz erfolgt die Bewertung mit Hilfe von 13 Fragen. Abgefragt werden Eigenschaften, die auf den Implementierungsgrad abzielen. \Gls{bspw} wird die Qualifizierung der Mitarbeiter und Führungskräfte oder der Einfluss der Kundenforderungen auf die Produktion abgefragt. 
Auch wird der Einsatz bestimmter Methoden Reifegraden zugeordnet. So wird der Einsatz von \gls{smed} der Stufe 2 (``Wissend'') und der Einsatz von \gls{heijunka} der Stufe 4 (``Etabliert/Gesichert'') zugeordnet. 

Für die Dimension Unternehmenskultur wird anhand von sieben Fragen ermittelt, inwieweit die Lean Production in der Unternehmenskultur verankert ist. 

\Gls{bspw} wird abgefragt, inwieweit die 5 Lean Prinzipien in der Unternehmensphilosophie verankert, kommuniziert und verstanden ist (``Annahme und Überzeugung'') oder inwieweit die Implementierung von den Führungskräften unterstützt wird (``Werte''). 

\begin{figure}[!ht] 
    \begin{minipage}{0.3\linewidth} 
    \begin{center}
      \includegraphics[scale=.27]{./img/schwarz2017:rg.png}
 % schwarz2017:rg.png: 0x0 pixel, 300dpi, 0.00x0.00 cm, bb=
    \end{center}
      \caption{Die sechs Reifegradstufen \cite{Schwarz2017}}\label{fig:links} 
    \end{minipage} 
    \hfill 
    \begin{minipage}{0.6\linewidth} 
 \includegraphics[scale=.3]{./img/schwarz2017:portfolio.png}
 % schwarz2017:portfolio.png: 0x0 pixel, 300dpi, 0.00x0.00 cm, bb=
    \caption{Portfoliodarstellung mit zwei Dimensionen \cite{Schwarz2017}}\label{fig:rechts} 
    \end{minipage} 
  \end{figure} 

\subsection{Durchführung}
Die tatsächliche Durchführung teilt sich auf in: Befragung, Detaildarstellung der Ergebnisse aller 20 Themen, Analyse und Ableitung von Verbesserungspotentialen und Ausarbeitung eines Maßnahmeplans für die Realisierung. 
%
% \textbf{Befragung}
Zunächst erfolgt die Befragung bei der eine Einschätzung zu jedem der 20 Themen stattfindet. Dazu werden \gls{bspw} Führungskräfte und ggf. externe Personen befragt. 

% \textbf{Auswertung}
Die Auswertung erfolgt in drei Schritten. Zunächst werden die arithmetischen Mittel der Antworten für jedes Thema ermittelt. Große Abweichungen untereinander deuten auf eine unausgewogene Entwicklung hin und es besteht punktueller Nachholbedarf. Im nächsten Schritt wird der Mittelwert über alle 20 Themen ermittelt. Dieser stellt den aktuellen Reifegrad des Unternehmens in Bezug auf Lean Production insgesamt dar. Im dritten Schritt werden die Mittelwerte der zwei Dimensionen miteinander in Bezug gesetzt. % TODO Grafik

Eine Abweichung größer als eine Reifegradstufe wird als kritisch bewertet und deutet auf eine einseitige Implementierung hin. Abhilfe schafft hier die Anpassung des Ressourceneinsatzes. 

\textbf{Einordnung:} 
Die Überprüfung der Dimension Methodenkompetenz kann das Lean Start-up dabei unterstützen den Einsatz geeigneter Methoden zu steuern. 
Die Überprüfung der Dimension Unternehmenskultur hingegen ist in den frühen Phasen des Lean Start-up wenig sinnvoll. Lean Start-ups bestehen üblicherweise aus kleinen Teams mit flachen Hierarchien, und Identifikation und Motivation ist bei allen Mitarbeitern sehr ausgeprägt.



