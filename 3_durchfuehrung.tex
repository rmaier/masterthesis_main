\chapter{Durchführung}

\section{Reichwald-2004}
% Kommentar: 
% Der Ansatz des Feedback-Loops passt schon gut zum LSU Konzept. Hervorzuheben ist z.B. der zielgerichtete Einsatz von Software, auch wenn nicht Abteilungen oder Prozesse zu überwinden sind. Betont werden soll jedenfalls, dass der Einsatz von Software hier nur unterstützend ist und den Menschen nicht von seiner Verantwortung entbindet. 
REICHWALD et al. untersucht das Projektmanagement im Feldanlauf und fokussiert sich dabei auf ein nachhaltiges Wissensmanagement \cite{Reichwald2004}. Ein durchgängiges und effizientes Wissensmanagement ist ein wesentlicher Bestandteil erfolgreicher Anläufe \cite{Kuhn2002}. % TODO check primary source Kuhn2002

REICHWALD et al. unterteilt den Wissensmanagementprozess in folgende Bestandteile: Identifizierung der Wissenslücken, Wissenserwerb und Wissensentwicklung, Wissensverteilung und Wissensbewahrung. Die ersten vier Bestandteile werden in Folgendem kurz vorgestellt. 

\subsection{Identifizierung der Wissenslücken}
Bereits vor Markteinführung muss bekannt sein, inwieweit das Produkt den Kundenerwartungen entspricht. Besonders die Produktmerkmale, die unterhalb der Kundenerwartungen liegen müssen identifiziert werden. Dazu eignen sich realitätsnahe Produkttests mit einer dem zukünftigem Kundenkreis entsprechenden Gruppe. Es sind geeignete Erhebungsinstrumente auszuwählen und weiterzuentwickeln. Um möglichst viele Qualitätsaspekte zu berücksichtigen müssen die Erfahrungen der Testpersonen über ein breites Spektrum erfasst werden. Darüber hinaus müssen die Ergebnisse einfach auszuwerten sein um eine schnelle Berücksichtigung zu gewährleisten. Dazu eignen sich z.B. standardisierte Fragebögen oder kurze mündliche Befragungen. 

\subsection{Wissenserwerb, -entwicklung und -verteilung}
Nach erfolgter Produkttests werden die Ergebnisse ausgewertet. Die einzelnen Ergebnisse werden in ein sogenanntes E-Workflowsystem eingepflegt, welches eine Art ERP System darstellt. % TODO glossary ERP
Dieses System ist in der Lage das gesammelte Faktenwissen entlang der gesamten Prozesskette bereitzustellen. Die bei der Auswertung der Tests gewonnenen Schwerpunkte bilden die Handlungsfelder des Anlaufteams. Mit Hilfe des E-Workflowsystems werden den jeweiligen Handlungsfeldern Maßnahmen und Zuständigkeiten sowie Umsetzungstermine zugeteilt. 
Nach der Umsetzung der Maßnahmen muss die Wirksamkeit möglichst durch die gleichen Personen bestätigt werden, die im Vorfeld die Handlungsfelder aufgezeigt haben. Wird die Wirksamkeit bestätigt, ist das Handlungsfeld erfolgreich abgeschlossen. 
Sind Erkenntnisse des laufenden Projekts auch für zukünftige Projekte von Bedeutung, so sollten sie im E-Workflowsystem gesondert gekennzeichnet und in zukünftigen Entwicklungsprozessen eingegliedert werden. 


\subsection{Kurzzusammenfassung}

Ein durchgängiges und effizientes Wissensmanagement ist ein wesentlicher Bestandteil erfolgreicher Anläufe. Dabei werden frühzeitig Kundenrückmeldungen zur Produktverbesserung ausgewertet und die Arbeit mit Softwaresystemen unterstützt. 

\section{Harjes-2004}

HARJES et al. untersucht das Anlaufmanagement mit besonderer Berücksichtigung des Produktentstehungprozesses \cite{Harjes2004}. 

\subsection{Robuste Produktionssysteme}

Höhere Variantenvielfalt und Individualisierungswünsche der Kunden stellen hohe Anforderungen an Fertigungs- und Montagelinien. Zunächst ist eine Standardisierung erforderlich. Produktionssysteme sollten einfach und übertragbar gestaltet werden. Daraus erfolgt eine erhöhte Flexibilität bei Integration neuer Baureihen und Komponenten, Änderungen können somit reibungsloser implementiert werden. Um Auswirkungen vom Prozess oder Produkt auf das Produktionssystem frühzeitig bewerten zu können, sind Prozess- und Produktdaten standardisiert zu verknüpfen und stets aktuell zu halten. 

\subsection{Produktdatenmodell}
Die stetige Reduzierung der eigenen Wertschöpfungstiefe erfordert eine hohe Transparenz bzgl. der Produktdaten. Dies erfolgt mit dem Aufbau digitaler Produktdatenmodelle, welche stets einen echten, plausiblen und aktuellen Datenstand aufweisen müssen. 
Integrierte Produktdatenmodelle (IPDM) % TODO glossary
verknüpfen Produkt- und Prozessdatenmodelle. Damit bekommen Änderungen mehrdimensionalen Charakter, betroffene Komponenten können identifiziert werden und die Folgen lassen sich simulieren und bewerten. 
Weiterführend wird die digitale Fabrik genannt, die die digitale Planung einer Fertigungsfabrik mit Integration aller Produkt- und Prozessdaten beschreibt. 

\subsection{Kurzzusammenfassung}
Robuste Profuktionssysteme reagieren agil auf Änderungen und können flexibel erweitert werden. Digitale Produktdatenmodelle sorgen Unternehmensübergreifend und -intern für erhöhte Transparenz und bessere Folgenabschätzung von Änderungen. 

\section{Straub-2006}

STRAUB verfolgt die Vision der Umstellung der Produktion von SOP % TODO glossary
auf Kammlinie an einem Wochenende. Im Fokus seiner Arbeit steht die schnelle und richtige Reaktion auf ungeplante Störungen im Anlauf \cite{Straub2006}. Die bisher eingesetzte präventive Methode der digitalen Fabrik erhöht zwar signifikant den Reifegrad der Planung, bietet jedoch keine Antwort aud verbleibende ungeplante Störungen. STRAUB beschreibt drei Säulen zur Realisierung kürzerer Anläufe: Einsatz von Anlaufteams, die organisatorische Einbindung der Teams in die Organisation und der Einsatz eines Methodenbaukastens. Letzterer ist für das LSU % TODO glossary
von Beteudung. 

\subsection{Methodenbaukasten}

Grundgedanke des Methodenbaukastens ist der Einsatz moderner Methoden, Werkzeuge und Standards.
Zum einen wird eine erhöhte Effizienz und Transparenz bewirkt. Zum anderen wird eine objektive Bewertung von Situationen und damit ein einheitliches Verständnis erreicht, was insbesondere die Zusammenarbeit mit jüngeren und unerfahrenen Mitarbeitern erleichert. 
Des weiteren wird der Einsatz einer Scorecard empfohlen. %TODO Bild 6 in Anhang oder hier. 
Zunächst werden quantifizierbare Anlaufindikatoren definiert. Mit Hilfe der Scorecard werden die wichtigsten Anlaufindikatoren kontinuierlich überwacht und Abweichungen vom Soll Wert werden schnell erkannt. Es folgt eine systematische Ursachenanalyse. So kann eine schnelle und zielgerichtete Reaktion gewährleistet werden. 
