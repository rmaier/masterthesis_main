
\chapter{Durchführung}\label{sec:durchfuehrung}

\section{Strategie \& Organisation}

\subsection*{Dombrowski-2009 - Lean Ramp-up. Ein Organisationsmodell}\label{dom09}

DOMBROWSKI et al. entwickelt ein Organisationsmodell für das Lean Ramp-up \cite{Dombrowski2009}. Kerngedanke ist der effiziente Einsatz von Mitarbeitern durch zielgerichtete Verwendung von standardisierten Methoden und Werkzeugen. DOMBROWSKI et al. kritisiert zum Einen den isolierten Einsatz von Methoden und Werkzeugen sowie fehlende bereichsübergreifende Zusammenarbeit, zum Anderen den Einsatz von auf Großunternehmen zugeschnittenen Lösungen in \gls{kmu}. 
Grundlage für den hier entwickelten Ordnungsrahmen ist das Modell des Ganzheitlichen Produktionssystems (\gls{gps}). Dazu wurde das \gls{gps} entlang des \gls{pep} auf den Serienanlauf erweitert und der Lösungsansatz des Lean Ramp-up entwickelt. Die drei wesentlichen Gestaltungskomponenten werden im Folgenden erläutert. 

\subsubsection{Klassifizierung von Serienanläufen}
Serienanläufe werden bisweilen noch als aufwändig zu etablierendes Projekt angesehen \cite{Kuhn2002}. Jedoch lassen sich wiederkehrende Elemente im Serienanlauf in Form von Referenzprozessen abbilden. Hierzu werden Referenzprozesse für bestimmte Serienanlaufklassen definiert. Die Klassifizierung erfolgt sowohl nach Grad der Produkt- bzw. Prozessänderung \cite{Kuhn2002} als auch nach Komplexität%TODO cite [16] Hertrampf ZWF
.
Die Komplexität ergibt sich \gls{bspw} aus Fertigungstiefe, Variantenvielfalt, Stückzahl oder der Anzahl der Lieferanten.

\begin{figure}[ht]
 \centering
 \includegraphics[scale=.3]{./img/dom09:klassen.png}
 % dom09:klassen.png: 0x0 pixel, 0dpi, 0.00x0.00 cm, bb=
 \caption{Klassifizierung des Serienanlaufs \cite{Dombrowski2009}}
 \label{fig:anlaufklassen}
\end{figure}

Die Klassifizierung erfolgt in drei Stufen (Abb. \ref{fig:anlaufklassen}). \textit{Neuanlauf:} Einführung eines neuen Produkts oder Prozesses. \textit{Änderungsanlauf:} Ein Produkt oder Prozess mit wesentlichen Änderungen wird eingeführt. \textit{Wiederholungsanlauf:} Anlauf eines nahezu unveränderten Produkts mit standardisierten Prozessen nach einer längeren Pause. 
Es wird eine Unterteilung in weitere Unterklassen vorgeschlagen für die jeweils Referenzprozesse definiert werden. 
Durch Klassifizierung und den Einsatz der Referenzprozesse kann sich das Unternehmen auf kritische Anlaufprozesse konzentrieren und ein gleichzeitiges Agieren an allen Fronten wird vermieden.

\subsubsection{Organisatorische Verankerung des Ordnungsrahmens}
Erst durch die Zuordnung der Methoden und Werkzeuge zu den betreffenden Prozessen und Mitarbeitern im Ordnungsrahmen werden Aufwand und Komplexität im Serienanlauf reduziert. Dazu wurde der \gls{gps} Ordnungsrahmen für das Lean Ramp-up zu einer Matrixstruktur weiterentwickelt. Dieser integriert inhaltlich zusammengehörende Serienanlaufprozesse in sog. Handlungsfeldern. DOMBROWSKI et al. definiert ein Handlungsfeld als einen Rahmen zur inhaltlichen Strukturierung des Serienanlaufs, welches stets das beschreibt, was im Serienanlauf getan werden muss. Über die Verknüpfung zu den Gestaltungsfeldern wird definiert, wie die Dinge, d.h. mit welchen Werkzeugen sie getan werden müssen. Die Werkzeuge dienen in diesem Zusammenhang hauptsächlich der Umsetzung der Methoden. 

\subsubsection{Auswahl von Methoden und Werkzeugen mit Hilfe des \gls{qfd}}

\begin{figure}[ht]
 \centering
 \includegraphics[scale=.35]{./img/dom09:qfd.png}
 % dom09:qfd.png: 0x0 pixel, 0dpi, 0.00x0.00 cm, bb=
 \caption{Systematische Auswahl von Methoden und Werkzeugen mittels \gls{qfd} \cite{Dombrowski2009}}
 \label{fig:qfd}
\end{figure}

DOMBROWSKI et al. sieht einen entscheidenden Wettbewerbsvorteil im Einsatz perfekt aufeinander abgestimmten Methoden und Werkzeuge und deren perfekte Ausrichtung an die Kundenanforderungen. % TODO cite [10]
Das Quality Function Deployment (\gls{qfd}) eignet sich gut für die Analyse und Darstellung solch komplexer Korrelationen. In diesem Fall erfolgt die Auswahl von Methoden und Werkzeugen in einem zweistufigen \gls{qfd} (Abb. \ref{fig:qfd}). 

Zunächst werden die detaillierten Ziele mit den Handlungsfeldern, Prozessen und Ergebnissen in Beziehung gesetzt und anschließend gewichtet. Ggf. kann bestehendes Wissen aus bisherigen Serienanläufen bei der Bewertung hinzugezogen werden. 
Ziel ist die Identifikation und Priorisierung der kritischen Handlungsfelder, Prozesse und Ergebnisse.
Nach erfolgter Priorisierung werden Mess- und Kontrollpunkte sowie dazugehörige Soll-Werte definiert. 

Der zweite Schritt erfolgt analog zum Ersten, mit dem Unterschied, dass zusätzlich die Gestaltungsfelder, Methoden und Werkzeuge hinzugezogen, in Beziehung gesetzt und gewichtet werden. 
In (14) befinden sich dann schließlich die ausgewählten Gestaltungsfelder, Methoden und Werkzeuge für das Lean Ramp-up. 
%%

\subsubsection{Einordnung}
Die Klassifizierung der Serienanläufe sowie das Einfließen von Erfahrung aus vergangenen Serienanläufen ist in den frühen Phasen des \gls{lsu} nicht möglich bzw. nicht sinnvoll. Jedoch bietet das zweistufige \gls{qfd}-Verfahren eine objektive und systematische Herangehensweise die Aktivitäten im Serienanlauf auf die kritischen und wichtigen Prozesse zu konzentrieren. 

\subsection*{Dombrowski-2011a - Lean Ramp-up. Handlungs- und Gestaltungsfelder}\label{sec:dom11a}

DOMBROWSKI et al. entwickelt das Modell aus \ref{dom09} % TODO Abschnitt ? 
weiter und präzisiert die Definitionen für den Lean Ramp-up Ordnungsrahmen und für die Handlungs- und Gestaltungsfelder \cite{Dombrowski2011a}. 

\subsubsection{Der Lean Ramp-up Ordnungsrahmen}\label{sec:dom11a:ordnungsrahmen}
\begin{figure}[h]
 \centering
 \includegraphics[scale=.35,keepaspectratio=true]{./img/dom11a:ordnungsrahmen.png}
 % dom11a:ordnungsrahmen.png: 0x0 pixel, 0dpi, 0.00x0.00 cm, bb=
 \caption{Der Lean Ramp-up Ordnungsrahmen \cite{Dombrowski2011a}}
 \label{fig:dom11a:ordnungsrahmen}
\end{figure}

Der Ansatz des Lean Ramp-up verfolgt die Einführung eines Ganzheitlichen Produktionssystems (\gls{gps}) bereits während des Serienanlaufs. Dazu bedarf es jedoch eines eigenen Ordnungsrahmens. Der Lean Ramp-up Ordnungsrahmen wie auf Abb. \ref{fig:dom11a:ordnungsrahmen} zu sehen, hat die Aufgabe, die Gestaltung der Zielerreichung des Serienanlaufs auf verschiedenen Abstraktionsebenen darzustellen und zu gestalten. 

In der ersten Ebene wird das Zielsystem beschrieben. Dabei werden auch Teilziele definiert und zueinander in Beziehung gesetzt. Ein klar strukturiertes Zielsystem bildet die verbindlichen Forderungen ab, welche die Grundlage für die darunter liegenden Ebenen bilden. 

In der zweiten Ebene liegen die Handlungs- und Gestaltungsfelder. Die Handlungsfelder (s. Abb. \ref{fig:dom11a:hf}) beschreiben ``Was'' im Serienanlauf getan werden muss. Es beinhaltet die Aufgaben, welche das Erreichen der Teilziele unterstützen. Die Zuständigkeiten für die Erledigung der Aufgaben werden an die Mitarbeiter über sog. Rollen zugeordnet. Eine Rolle ist \gls{bspw} der Montageleiter. 
Die Gestaltungsfelder (s. Abb. \ref{fig:dom11a:gf}) beschreiben ``Wie'' die Dinge getan werden sollen. Sie bilden einen thematischen Rahmen für inhaltlich ähnliche Methoden und Werkzeuge, die die Teilziele unterstützen. Methoden sind planmäßige Vorgehensweisen wie z.B. der Problemlöseprozess, der mit dem Werkzeug \gls{fmea} % TODO cite [4]?! 
realisiert werden kann.  

In der dritten Ebene werden die Beziehungen zwischen Aufgaben, Mitarbeiter (Rollen), Methoden und Werkzeugen mittels der Ablauforganisation abgebildet. Dies gewährleistet eine systematische Prozessorientierung im Serienanlauf. 

In der vierten Ebene werden die Ebenen 1-3 in der Aufbauorganisation verankert. Sie regelt die Aufteilung der Verantwortlichkeiten, Aufgaben und Kompetenzen auf verschiedene Organisationseinheiten sowie deren Beziehungen untereinander. % TODO cite [12] 

\subsubsection{Die Handlungsfelder im Lean Ramp-up}\label{sec:dom11a:hf}
\begin{figure}[!ht] 
    \begin{minipage}{0.45\linewidth} 
    \begin{center}
 \includegraphics[scale=.4,keepaspectratio=true]{./img/dom11a:hf.png}
 \caption{Handlungsfelder \cite{Dombrowski2011a}}\label{fig:dom11a:hf}
 % dom11a:hf.png: 0x0 pixel, 0dpi, nanxnan cm, bb=
\end{center}
    \end{minipage} 
    \hfill 
    \begin{minipage}{0.45\linewidth} 
 \includegraphics[scale=.4,keepaspectratio=true]{./img/dom11a:gf.png}
 \caption{Gestaltungsfelder\cite{Dombrowski2011a}}\label{fig:dom11a:gf}
    \end{minipage} 
  \end{figure} 

Mit Hilfe der zehn Handlungsfelder (s. Abb. \ref{fig:dom11a:hf}) erfolgt die strukturelle Entwicklung und Realisierung der Teilsysteme des Produktionssystems. Ihre Aufgabe besteht darin, den Serienanlauf effektiv zu gestalten. Die Handlungsfelder unterstützen die Erreichung der Teilziele, die Forderungen an die Ergebnisse stellen, sog. Systemziele. Deren Erreichung hat maximale Priorität. 

Aufgrund der Interdependenzen zwischen den Handlungsfeldern ist eine ganzheitliche Betrachtungsweise bei der Umsetzung zwingend erforderlich. 
Eine detaillierte Beschreibung der zehn Handlungsfelder findet sich im Anhang \ref{appendix:dom11a:hf}. % TODO add and ref to annex

  
\subsubsection{Die Gestaltungsfelder im Lean Ramp-up}\label{sec:dom11a:gf}
Die zehn Gestaltungsfelder (s. Abb. \ref{fig:dom11a:gf}) unterstützen die Verbesserung des Serienanlaufs. Die davon abgeleiteten Methoden und Werkzeuge sorgen für Effizienz im Anlauf. Die Gestaltungsfelder unterstützen die Erreichung der Teilziele, die Forderungen an die Abwicklung stellen, sog. Vorgehensziele. Deren Erreichung ist im Vergleich zu den Systemzielen (aus \ref{sec:dom11a:hf}) sekundär. 

Auch hier stehen die Elemente in enger Beziehung zueinander. Einige Methoden oder Werkzeuge unterstützen mehrere Teilziele gleichzeitig. Dies reduziert die Komplexität der Elemente und vereinfacht die Anwendung. 
Eine detaillierte Beschreibung der zehn Gestaltungsfelder findet sich im Anhang \ref{appendix:dom11a:gf}. % TODO add and ref to annex

\subsubsection{Einordnung}
Selbst die teilweise Anwendung des Lean Ramp-up Ordnungsrahmens in frühen Phasen des \gls{lsu} hilft, die ``richtigen Dinge'' ``richtig zu tun''. Dabei wird im \gls{lsu} bereits in frühen Phasen eine Grundstruktur erarbeitet, auf die in den Wachstumsphasen sowohl der Detaillierungsgrad erhöht, als auch in den Abstraktionsebenen weiter voran geschritten werden kann. So ist \gls{bspw} eine Beschreibung der Aufbauorganisation in den frühen Phasen nicht zielführend. Auch führen die Rollenzuteilungen der Aufgaben in den frühen Phasen auf wenige Mitarbeiter. Da jedoch zu Beginn die Struktur der Rollenzuteilung steht, können die Aufgaben mit wenig Aufwand auf eine wachsende oder schwankende Mitarbeiterzahl zugeteilt werden. 

\subsection*{Dombrowski2011b - Lean Ramp-up: Schwerpunkte im Anlaufmanagement}

DOMBROWSKI et al. entwickelt hier Schwerpunkte für die Realisierung eines Lean Ramp-up \cite{Dombrowski2011b}. Grundlage bietet der zuvor in \ref{sec:dom11a} beschriebene Ordnungsrahmen mit den Handlungs- und Gestaltungsfeldern (s. Abb. \ref{fig:dom11a:ordnungsrahmen} sowie \ref{fig:dom11a:hf} und \ref{fig:dom11a:gf}). Es werden Schwerpunkte für verschiedene Timing-Strategien entwickelt, die auf die generischen Ziele Qualität, Kosten und Zeit aufbauen. Mit Hilfe der Schwerpunkte kann sich ein Unternehmen im Serienanlauf darauf konzentrieren die ``richtigen'' Dinge ``richtig'' zu tun. Sie lassen sich aufgrund der Orientierung an den generischen Zielen weitestgehend auf jeden Serienanlauf anwenden. 

\subsubsection{Schwerpunktbildung im Anlaufmanagement}

Mit Hilfe der Schwerpunkte kann sich das Anlaufmanagement von Beginn an auf die kritischen Themenstellungen effektiv und effizient konzentrieren. Ein gleichzeitiges Agieren auf allen Fronten wird somit vermieden. Als Schwerpunkte werden diejenigen Elemente bezeichnet, welche einen besonders hohen Beitrag zur Zielerreichung leisten. 

Die Auswahl der Schwerpunkte erfolgt in drei Schritten: 
\begin{enumerate}
 \item Festlegung und Priorisierung der strategischen Ziele
 \item Ausrichtung der Handlungs- und Gestaltungsfelder an den Zielen
 \item Priorisierung und Auswahl der Schwerpunkte
\end{enumerate}

\subsubsection{Ziele und Zieltypen im Produktionsanlauf}
Die Kategorisierung in Zieltypen erfolgt anhand des Markteintrittspunkts. Diese Unterteilung in sog. Timing-Strategien in Form von \textit{First-Mover} und \textit{Follower} ist in der Literatur bereits bekannt. Neu ist der Zieltyp \textit{Repeater}. Jeder Zieltyp wird dadurch charakterisiert, dass dieser die Zieldimensionen Qualität, Kosten und Zeit unterschiedlich hoch priorisiert. 

Der \textbf{First-Mover} priorisiert den Faktor Zeit. Er möchte so schnell wie möglich ein Produkt entwickeln und in den Markt drängen. Er strebt eine Vorreiterposition und langfristig die Monopolstellung an. Ferner möchte mit der Errichtung von Markteintrittsbarrieren wie \gls{bspw} Patenten, hohen Marktanteilen und einem Erfahrungsvorsprung potentielle Wettbewerber schwächen. 
Nachteile sind hohe Risiken, Unsicherheiten und Kosten für die Markterschließung. 

Der \textbf{Follower} priorisiert entweder den Faktor Qualität oder Kosten, je nach dem ob er die Qualitäts- oder Kostenführerschaft anstrebt. Der Markteintritt erfolgt nach erfolgreicher Markterschließung durch Wettbewerber. Somit können Anfängerfehler vermieden und Produkte schneller entwickelt werden. Als Nachteile werden teilw. hohe Markteintrittsbarrieren genannt. 

Der \textbf{Repeater} kann sich auf eines der drei generischen Zieldimensionen konzentrieren. Er zeichnet sich aus durch die zyklisch wiederholende Nachfrage eines bereits eingeführten Serienprodukts wie z.B. bei saisonalen Konsumgütern. In diesem Fall kann das Unternehmen bei jedem Neuanlauf die Marktposition weiter ausbauen und intern kontinuierliche Verbesserung der Prozesse und Produkte anstreben. Nachteilig kann sich die Strategie auf die Entwicklung neuer Produkte auswirken. 

\subsubsection{Schwerpunkte im Anlaufmanagement}
%	TODO wenigstens die Grafiken erläutern
% 	BEwertung v. HF / GF sowie Beziehungen untereinander 

%	FIRST MOVER
\begin{figure}[ht]
 \centering
 \includegraphics[scale=.5]{./img/dom11b:firstmover.png}
 % dom11b:firstmover.png: 0x0 pixel, 0dpi, 0.00x0.00 cm, bb=
 \caption{Schwerpunkte für Firstmover \cite{Dombrowski2011b}}
 \label{fig:dom11b:firstmover}
\end{figure}
\textbf{Schwerpunkte für First-Mover}: Der Fokus des First-Mover liegt in der Zieldimension Zeit. 
% 	IMAGE REF
Die Bewertung der Handlungs- und Gestaltungsfelder sowie die Auswahl der Schwerpunkte ist auf Abbildung \ref{fig:dom11b:firstmover} zu erkennen. 
%
Engpässe entstehen meist extern in der Beschaffung von Serienteilen, Fertigungs- und Montagemitteln. Die termingerechte Verfügbarkeit hat oberste Priorität. Da die Beschaffungszeiten sich oft nicht verkürzen lassen muss die Beschaffung vorgezogen oder abgesichert werden. 
Des weiteren ergibt sich eine stark erhöhte Komplexität durch den erstmaligen Praxiseinsatz der Wertschöpfungskette. Um die Komplexität zu beherrschen empfiehlt sich eine hohe Abstimmung mit der gesamten Wertschöpfungskette. 
Besonders wichtige Lieferanten sollten intensiv betreut und eine langfristige Partnerschaft angestrebt werden. 
Intern empfiehlt sich eine enge strategische und operative Zusammenarbeit von Einkauf, Disposition und Produktionsplanung. 

\begin{figure}[ht]
 \centering
 \includegraphics[scale=.5]{./img/dom11b:kostenfuehrer.png}
 % dom11b:firstmover.png: 0x0 pixel, 0dpi, 0.00x0.00 cm, bb=
 \caption{Schwerpunkte für Follower (Kosten) \cite{Dombrowski2011b}}
 \label{fig:dom11b:kostenfuehrer}
\end{figure}

% 	FOLLOWER - KOSTEN
\textbf{Schwerpunkte für Follower (Kosten):} Der Fokus des Follower (Kosten) liegt in der Zieldimension Kosten. 
% 	IMAGE REF
Die Bewertung der Handlungs- und Gestaltungsfelder sowie die Auswahl der Schwerpunkte ist auf Abbildung \ref{fig:dom11b:kostenfuehrer} zu erkennen. 
%
Die Produktentwicklung hat einen sehr hohen Einfluss auf spätere Herstellkosten. % TODO cite [8]
Kostentreiber sind insbesondere eine hohe Komplexität und Varianz in der Produktstruktur. Abhilfe schafft eine angemessene Standardisierung der Produkte und Komponenten. 
Auch die Qualitätssicherung und Instandhaltung bergen hohe Kosten durch unentdeckte Fehler und Risiken. Dem kann mit Hilfe von frühzeitiger Fehler- und Risikovermeidung begegnet werden. 
Weiteres Potential steckt in laufenden Kosten wie \gls{bspw} Personalkosten, Materialkosten und einmaligen Investitionen in Produktionsmittel. Einsparungen können bereits in der Anlaufphase durch konsequente Eliminierung von Verschwendung erreicht werden. 

\begin{figure}[ht]
 \centering
 \includegraphics[scale=.5]{./img/dom11b:qualitaetsfuehrer.png}
 % dom11b:firstmover.png: 0x0 pixel, 0dpi, 0.00x0.00 cm, bb=
 \caption{Schwerpunkte für Follower (Qualität) \cite{Dombrowski2011b}}
 \label{fig:dom11b:qualitaetsfuehrer}
\end{figure}

% 	FOLLOWER - QUALITÄT
\textbf{Schwerpunkte für Follower (Qualität):} Der Fokus des Follower (Qualität) liegt in der Zieldimension Qualität.
% 	IMAGE REF
Die Bewertung der Handlungs- und Gestaltungsfelder sowie die Auswahl der Schwerpunkte ist auf Abbildung \ref{fig:dom11b:qualitaetsfuehrer} zu erkennen. 
%
Die Produktqualität sollte in den frühen Phasen des \gls{pep} verbessert werden. Die Behebung der Ursachen von Qualitätsproblemen in der laufenden Produktion ist mit erhöhtem Aufwand und Kosten verbunden. Präventive Qualitätssicherung kann bereits im \gls{pep} Fehlerursachen mit niedrigem Aufwand beseitigen. 
In der reaktiven Qualitätssicherung kann \gls{bspw} ein standardisierter Problemlöseprozess mit zugeordneten Methoden und Werkzeugen effektiv und effizient Problemursachen in der laufenden Produktion beseitigen. 

Weiterhin können umfassende Pilotversuche frühzeitig Probleme erkennen. Weitere daraus gewonnenen Erfahrungen können in die spätere Produktion mit einfließen. 

\textbf{Schwerpunkte für Repeater:} Die Schwerpunkte für Repeater werden nicht gesondert erläutert da sie abhängig von der gewählten Zieldimension den bereits vorgestellten Herangehensweisen entsprechen. 
\subsubsection{Einordnung}
Die hier vorgestellte Systematisierung der Zieltypen und die davon abgeleiteten Schwerpunkte lassen sich gut im \gls{lsu} anwenden da auch das \gls{lsu} eines der genannten Ziele verfolgt. Dabei kann die Implementierung dynamisch und schrittweise erfolgen, was besonders mit Hinblick auf das Wachstum sinnvoll ist. So kann ein \gls{lsu} in der frühen Phase die richtigen Prioritäten setzen und damit die begrenzten Ressourcen optimal einsetzen. Die für das \gls{lsu} nicht relevanten Elemente wie \gls{bspw} Personalorganisation können zu Beginn mit Hinblick auf die Zukunft zwar beachtet werden, müssen jedoch nicht sofort ausgearbeitet und umgesetzt werden. Die Umsetzung kann sukzessive entsprechend des Wachstums erfolgen. 

\subsection*{Dombrowski2017a - Lean Ramp-up: Ein Organisationsmodell für das Anlaufmanagement}

DOMBROWSKI et al. integriert einen \gls{pdca}-Zyklus in den in \ref{sec:dom11a} entwickelten Lean Ramp-up Ordnungsrahmen und entwickelt ein Vorgehen in zehn Schritten zur unterenehmensspezifischen Gestaltung eines Lean Ramp-up Organisationsmodells \cite{Dombrowski2017a}. 

\subsubsection{Integration des \gls{pdca}-Zyklus in den Lean Ramp-up Ordnungsrahmen}
\begin{figure}[h]
 \centering
 \includegraphics[scale=.35,keepaspectratio=true]{./img/dom17a:ordnungsrahmen.png}
 % dom11a:ordnungsrahmen.png: 0x0 pixel, 0dpi, 0.00x0.00 cm, bb=
 \caption{Der Lean Ramp-up Ordnungsrahmen \cite{Dombrowski2017a}}
 \label{fig:dom17a:ordnungsrahmen}
\end{figure}

Der Lean Ramp-up Ordnungsrahmen wie in Abschnitt \ref{sec:dom11a} bereits vorgestellt, hat die Aufgabe, die Gestaltung der Zielerreichung des Serienanlaufs auf verschiedenen Abstraktionsebenen darzustellen und zu gestalten. 
An dieser Stelle wird der vorhandene Ordnungsrahmen (s. Abb. \ref{fig:dom11a:ordnungsrahmen} um eine Abstraktionsebene erweitert, mit dem Ziel, den \gls{pdca}-Zyklus nachhaltig zu integrieren (s. Abb. \ref{fig:dom17a:ordnungsrahmen}). 
Für eine Beschreibung der Abstraktionsebenen Zielsystem, Handlungs- und Gestaltungsfelder, Ablauforganisation und Aufbauorganisation wird auf Abschnitt \ref{sec:dom11a:ordnungsrahmen} verwiesen. 

Die neue, dritte Ebene wird an dieser Stelle neu eingeführt und hat die Aufgabe, die in den Handlungs- und Gestaltungsfeldern gebündelten Aufgaben in den \gls{pdca}-Zyklus zu integrieren. Dies erfolgt, in dem die Aufgaben den einzelnen \gls{pdca}-Phasen zugeordnet werden. Dadurch werden zum einen die Gestaltung von gleichförmigen Prozessen, zum anderen die Verknüpfung der Elemente untereinander vereinfacht. Beides wird durch die \gls{pdca}-Strukturierung vereinfacht, da gleichzeitig weniger Elemente miteinander in Beziehung stehen. Die Komplexität der Systemgestaltung wird reduziert. 

Für die Zuteilung der Mitarbeiter-Rollen ist keine \gls{pdca}-Einteilung vorgesehen, da dies bereits durch die \gls{pdca}-Einteilung der Aufgaben erfolgt. 


\subsubsection{Unternehmensspezifische Gestaltung des Lean Ramp-up Organisationsmodells in zehn Schritten}

\begin{figure}[ht]
 \centering
 \includegraphics[scale=.3,keepaspectratio=true]{./img/dom17a:qfd.png}
 % dom11a:ordnungsrahmen.png: 0x0 pixel, 0dpi, 0.00x0.00 cm, bb=
 \caption{Vorgehen zur Gestaltung eines Lean Ramp-up Organisationsmodells \cite{Dombrowski2017a}}
 \label{fig:dom17a:qfd}
\end{figure}

DOMBROWSKI et al. stellt eine auf der \gls{qfd}-Methode basierende Vorgehensweise vor, um die bisher vorgeschlagenen Konzepte im Unternehmen zu implementieren (s. Abb. \ref{fig:dom17a:qfd}). 
Während die Schritte vier und neun bei jedem Anlauf neu durchgeführt werden sollten, sind die restlichen nur bei Erstimplementierung oder im Rahmen einer Weiterentwicklung durchzuführen. 

\begin{enumerate}
 %	1
 \item Im ersten Schritt wird das Zielsystem definiert. Es besteht aus generellen Zielen und untergeordneten Teilzielen. Sie setzen die Vision, Mission und die Strategie des Unternehmens um. Die generellen Ziele und Teilziele werden zuerst definiert und anschließend priorisiert. Wie bereits in Abschnitt \ref{sec:dom11a:hf} und \ref{sec:dom11a:gf} beschrieben, wird zwischen System- und Vorgehenszielen unterschieden. 
 %	2
 \item Die Handlungsfelder werden gemäß der generellen Ziele und der Systemziele definiert. Hier setzt auch der \gls{pdca}-Ansatz an, der sich bis hin zum letzten Vorgehensschritt fortsetzt. Dies bedeutet, dass von nun an alle Elemente einer der \gls{pdca}-Phasen zugeordnet wird. 
 %	3
 \item Ein Anlaufprozess in Form eines Netzplans wird erstellt. Grundlage bietet die Analyse der Vorgänger-Nachfolger Beziehungen der Aufgaben.
 %	4
 \item  Der zuvor erstellte Anlaufprozess wird durch Prozess- und Kontrollvorgaben ergänzt und entsprechenden Mitarbeiter-Rollen zugeordnet. 
 An dieser Stelle sind bereits alle wesentlichen Elemente für den effektiven Produktionsanlauf definiert. 
 %	5
 \item Die Gestaltungsfelder werden gemäß der generellen Ziele und der Vorgehensziele definiert. Für jede \gls{pdca}-Phase werden relevante Methoden und Werkzeuge definiert und den entsprechenden Gestaltungsfeldern zugeordnet. 
 %	6
 \item Die Ablauforganisation wird definiert. Dies geschieht indem die Methoden und Werkzeuge mit dem Aufgaben in der Matrixstruktur verknüpft werden. Mit Hilfe von sog. \gls{pdca}-Matrizen werden die Korrelationen zwischen den Elementen analysiert und bewertet. 
 %	7
 \item Methoden und Werkzeuge werden durch Auswertung der Korrelationsergebnisse ausgewählt. Da die Elemente noch nicht aufeinander abgestimmt sind, stellt dies zunächst eine Vorauswahl dar. 
 %	8
 \item Der achte Schritt bildet mit dem vorherigen eine Optimierungsschleife. Synergien und Konflikte zwischen den Methoden und Werkzeugen werden in einer Beziehungsmatrix analysiert und bei Anpassungsbedarf wird Schritt sieben wiederholt. Sobald die Elemente aufeinander abgestimmt sind, ist die Auswahl nicht mehr vorläufig und gewährleisten einen effizienten Produktionsanlauf. 
 %	9
 \item Die Aufbauorganisation mit den dazugehörigen Verantwortungen, Aufgaben und Kompetenzen %(\gla{vak})
 wird nun definiert. Dieser Schritt bildet die Grundlage für die organisatorische Verankerung des Lean Ramp-up Organisationsmodells im Unternehmen. 
 %	10
 \item Die Gestaltung eines \gls{kvp} ist Voraussetzung für einen nachhaltigen und dauerhaften Erfolg. Dabei ist darauf zu achten, dass sich der \gls{kvp}-Prozess auf alle Ebenen erstreckt. 
\end{enumerate}

\subsubsection{Einordnung}
Die Bewertung des Lean Ramp-up Organisationsmodells im Zusammenhang mit dem \gls{lsu} erfolgte bereits. Es wird auf folgende Aspekte eingegangen: 
\begin{enumerate}
 \item \textbf{Integration des \gls{pdca}-Zyklus in den Lean Ramp-up Ordnungsrahmen: }

 
 Während sich der Grundgedanke der ständigen Verbesserung im \gls{lsu} auf das Produkt konzentriert, sorgt der um den \gls{pdca}-Zyklus erweiterte Lean Ramp-up Ordnungsrahmen für eine ganzheitliche und strukturelle Verankerung des \gls{kvp} im Unternehmen. 
 
 \item \textbf{Unternehmensspezifische Gestaltung des Lean Ramp-up Organisationsmodells in zehn Schritten: }
 
 Mit Hilfe der zehn Schritte lässt sich ein individuelles Organisationsmodell entwickeln, welches die Aspekte des Lean Ramp-up an die Bedürfnisse des \gls{lsu} anpasst. Dabei kann in den frühen Phasen des \gls{lsu} die Gestaltung der Gestaltungsfelder sowie der Rollenzuteilung vereinfacht und auf eine detaillierte Beschreibung der Aufbauorganisation zunächst verzichtet werden. Das Organisationsmodell kann während des Wachstums sukzessive ergänzt und überarbeitet werden, in dem die zehn Schritte wieder angewendet werden. Essentiell ist jedoch bereits in der frühen Phase des \gls{lsu} die Organisationsplanung mit Hinblick auf zukünftig auszubauende Elemente. 
 \end{enumerate}
 
\section{Planung}
\subsection*{Quasdorff-2016 - Lean Management und Digitale Fabrik}

QUASDORFF et al. behandelt die Schnittmengen von Lean Management und der Digitalen Fabrik \cite{Quasdorff2016}. 

Die Digitale Fabrik umfasst die Abbildung und Simulation von Produkt, Prozess und Ressourcen in einem Informationssystem. Während für die Digitale Fabrik die Datenbasis für den Erfolg ausschlaggebend ist, muss die Lean Philosophie aktiv im Unternehmen gelebt werden. Beim gleichzeitigen Einsatz beider Methoden sind große Synergieeffekte zu erwarten. 

Die Digitale Fabrik unterstützt die Vermeidung von Muda (Verschwendung), Mura (Unausgeglichenheit) und Muri (Überbeanspruchung). Durch die zunehmende Digitalisierung (Industrie 4.0 bzw. \gls{iot}) wächst die Bedeutung von Quellen der Verschwendung im Bereich der Informationstechnik und der Datenverarbeitung. 

Bei der Gestaltung der Digitalen Fabrik müssen einige Aspekte beachtet werden. So ist die konsequente Anwendung von Lean Prinzipien Voraussetzung für die Digitale Fabrik. Schlanke Prozesse und deren Standardisierung sorgen dafür, dass die Komplexität der Modelle der Digitalen Fabrik beherrschbar wird. 
Verbesserungen an ineffizienten Prozessen sollten am Prozess als solchen ansetzen anstatt verbesserte Technologie einzusetzen. % TODO cite [8]

Erfolgsfaktoren sind eine hohe Detailtreue und Datenqualität. Das Modell sollte zu jedem Zeitpunkt der Realität entsprechen. Dennoch sollte vor einer Änderung der Ist-Zustand mit dem Dokumentationszustand verglichen werden. 

Abschließend ist zu bemerken, dass der Einsatz der Digitalen Fabrik den Gang in den Shopfloor nicht ersetzen sondern nur unterstützen kann. 

\textbf{Einordnung:} Der Artikel liefert Ansätze für die Gestaltung der DF im LSU. Dabei muss stets auf den Angemessenen Einsatz der Methoden geachtet werden. Ggf. sollten nur einige kritische Elemente Einzug in die DF erhalten. Dennoch muss das Modell zu jedem Zeitpunkt aktuell und plausibel sein. Auch die Detailtreue muss im Anfangsstadium nicht zwingend maximiert werden sondern den Zweck erfüllen einen hohen Reifegrad in der frühen Planungsphase zu erreichen und viele Entscheidungen möglichst früh treffen zu können. 

\subsection*{Schwarz-2017 - Reifegradmodell für Lean Production}

SCHWARZ et al. entwickelt ein Reifegradmodell zur Bewertung des Implementierungsfortschritts von Lean Production im Unternehmen \cite{Schwarz2017}. Einfache Befragungen eignen sich aufgrund der Komplexität nicht zur Bewertung. 

\subsubsection{Bestandteile}
Das von SCHWARZ entwickelte Reifegradmodell erfasst den Fortschritt in den zwei Dimensionen Methodenkompetenz und Unternehmenskultur. 
Methodenkompetenz beschreibt die Fähigkeit eines produzierenden Unternehmens die Prinzipien der Lean Philosophie durch Anwendung spezifischer Methoden systematisch und gezielt im Produktionssystem umzusetzen. 
Die folgenden fünf Lean Prinzipien dienen als Grundlage für das Modell: Kundennutzen, Wertstrom, Fluss, Pull und Perfektion. 
Es existieren zahlreiche Methoden die jeweils ein oder mehrere Prinzipien umsetzen. 

Voraussetzung für den nachhaltigen Einsatz von Lean Prinzipien ist das aktive Leben der Ideen sowie die Verankerung in der Unternehmenskultur. % TODO Cite [7] Baumgärtner (nicht verfügbar), alt. Quelle suchen
Die Unternehmenskultur wird mit Hilfe folgender Aspekte beschrieben: 
\begin{itemize}
 \item Grundlegende Annahmen und Überzeugungen
\item Implizite und explizite Werte 
\item Mittel zur Verwirklichung dieser Werte
\item die Außenwirkung.
\end{itemize}
Die Ausprägungen der Aspekte sind Voraussetzungen für eine nachhaltige und langfristige Anwendung der Methoden durch die Mitarbeiter und somit für die Implementierung der Lean Prinzipien. 

\textbf{Gestaltung}

Die Bewertung der Reifegrade in den zwei Dimensionen erfolgt in sechs Stufen. % TODO siehe Grafik. 


Für die Dimension Methodenkompetenz erfolgt die Bewertung mit Hilfe von 13 Fragen. Abgefragt werden Eigenschaften, die auf den Implementierungsgrad abzielen. \Gls{bspw} wird die Qualifizierung der Mitarbeiter und Führungskräfte oder der Einfluss der Kundenforderungen auf die Produktion abgefragt. 
Auch wird der Einsatz bestimmter Methoden Reifegraden zugeordnet. So wird der Einsatz von \gls{smed} der Stufe 2 (``Wissend'') und der Einsatz von \gls{heijunka} der Stufe 4 (``Etabliert/Gesichert'') zugeordnet. 

Für die Dimension Unternehmenskultur wird anhand von sieben Fragen ermittelt, inwieweit die Lean Production in der Unternehmenskultur verankert ist. 

\Gls{bspw} wird abgefragt, inwieweit die 5 Lean Prinzipien in der Unternehmensphilosophie verankert, kommuniziert und verstanden ist (``Annahme und Überzeugung'') oder inwieweit die Implementierung von den Führungskräften unterstützt wird (``Werte''). 

\begin{figure}[!ht] 
    \begin{minipage}{0.3\linewidth} 
    \begin{center}
      \includegraphics[scale=.27]{./img/schwarz2017:rg.png}
 % schwarz2017:rg.png: 0x0 pixel, 300dpi, 0.00x0.00 cm, bb=
    \end{center}
      \caption{Die sechs Reifegradstufen \cite{Schwarz2017}}\label{fig:links} 
    \end{minipage} 
    \hfill 
    \begin{minipage}{0.6\linewidth} 
 \includegraphics[scale=.3]{./img/schwarz2017:portfolio.png}
 % schwarz2017:portfolio.png: 0x0 pixel, 300dpi, 0.00x0.00 cm, bb=
    \caption{Portfoliodarstellung mit zwei Dimensionen \cite{Schwarz2017}}\label{fig:rechts} 
    \end{minipage} 
  \end{figure} 

\subsubsection{Durchführung}
Die tatsächliche Durchführung teilt sich auf in: Befragung, Detaildarstellung der Ergebnisse aller 20 Themen, Analyse und Ableitung von Verbesserungspotentialen und Ausarbeitung eines Maßnahmeplans für die Realisierung. 
%
% \textbf{Befragung}
Zunächst erfolgt die Befragung bei der eine Einschätzung zu jedem der 20 Themen stattfindet. Dazu werden \gls{bspw} Führungskräfte und ggf. externe Personen befragt. 

% \textbf{Auswertung}
Die Auswertung erfolgt in drei Schritten. Zunächst werden die arithmetischen Mittel der Antworten für jedes Thema ermittelt. Große Abweichungen untereinander deuten auf eine unausgewogene Entwicklung hin und es besteht punktueller Nachholbedarf. Im nächsten Schritt wird der Mittelwert über alle 20 Themen ermittelt. Dieser stellt den aktuellen Reifegrad des Unternehmens in Bezug auf Lean Production insgesamt dar. Im dritten Schritt werden die Mittelwerte der zwei Dimensionen miteinander in Bezug gesetzt. % TODO Grafik

Eine Abweichung größer als eine Reifegradstufe wird als kritisch bewertet und deutet auf eine einseitige Implementierung hin. Abhilfe schafft hier die Anpassung des Ressourceneinsatzes. 

\textbf{Einordnung:} 
Die Überprüfung der Dimension Methodenkompetenz kann das Lean Start-up dabei unterstützen den Einsatz geeigneter Methoden zu steuern. 
Die Überprüfung der Dimension Unternehmenskultur hingegen ist in den frühen Phasen des Lean Start-up wenig sinnvoll. Lean Start-ups bestehen üblicherweise aus kleinen Teams mit flachen Hierarchien, und Identifikation und Motivation ist bei allen Mitarbeitern sehr ausgeprägt.

\section{Regelung}
\subsection*{Basse-2014 - Beherrschung von Unsicherheiten}

BASSE et al. entwickelt Prinzipien zur Beherrschung von Unsicherheiten im Anlauf \cite{Basse2014a}. Ansatz ist die Beherrschung der Komplexität. Für Komplexitätstreiber wurden drei Quellen identifiziert: Interdiszliplinarität, Interdependenzen und dynamisch verändernde innere und äußere Bedingungen \cite{Basse2014a, Gartzen2012, Schuh2008}. % TODO cite Franzkoch in Schuh2008 -> [11]. Gartzen2012 nicht geprüft
% \subsubsection*{Lösungsansatz}

\textbf{Modelldesign:} Modelle erleichtern die Darstellung und Analyse komplexer Systeme. Dies wird hauptsächlich durch Strukturierung und Abstraktion der Elemente erreicht. 
Insbesondere die Abbildung der Struktur und des Verhaltens sind für den Serienanlauf von großer Bedeutung. Zur weiteren Vereinfachung der Analyse kann das System modularisiert und bei Bedarf zu einem großen Modell zusammengefügt werden. 

\textbf{Regelung und Heuristik:} Aufgrund von Zeitdruck und mangelhafter Datenbasis können Entscheidungen nicht faktenbasiert getroffen werden. Hier eignet sich die Heuristik, die versucht innerhalb kurzer Zeit gute Entscheidungen mittels unvollständiger Datenbasis und Erfahrungswerte zu treffen. Zusätzlich können Methoden der Regelungstechnik ein Feedback geben, sodass Störungen erfolgreich abgestellt werden.

\textbf{Lösungsräume und Toleranzen:}
Die Anlaufplanung ist immer ein Kompromiss aus Detaillierungsgrad und Aufwand. Um die Systemziele zu erfüllen und gleichzeitig Rücksicht auf Zielkonflikte zu nehmen, hilft eine Herangehensweise die mögliche Lösungsansätze in Lösungsräumen darstellt und für die Ergebnisse Toleranzbereiche definiert. Lösungsräume beschreiben verschiedene Zusammensetzungen von Lösungsansätzen, die jeweils die Systemziele erfüllen. 

\textbf{Mustererkennung und Selbstoptimierung:}
Komplexe Systeme bestehen nicht gänzlich aus willkürlichen Strukturen. Auch in komplexen Systemen können Muster erkannt werden, von denen Regeln abgeleitet werden können. Sie helfen das System zu verstehen, ohne die Funktion oder die Ursachen zu erkennen. Darauf aubauend können Regelungsmechanismen angewendet werden. Mit Hilfe dieser Regelungsmechanismen können die Ziele und das Systemverhalten angepasst werden \cite{Frank2009}. %TODO nicht geprüft 

\textbf{Zusammenfassung:}
Die vorgeschlagenen Prinzipien sind Maßnahmen zur Verringerung der Komplexität eines Systems. Bei geringerer Komplexität sind Entscheidungen leichter zu fällen und die Konsequenzen besser abzuschätzen. 

\subsection*{Straub-2006 - Methodenbaukasten}

STRAUB et al. verfolgt die Vision der Umstellung der Produktion von \gls{sop} 
auf Kammlinie an einem Wochenende. Im Fokus seiner Arbeit steht die schnelle und richtige Reaktion auf ungeplante Störungen im Anlauf \cite{Straub2006}. Die bisher eingesetzte präventive Methode der digitalen Fabrik erhöht zwar signifikant den Reifegrad der Planung, bietet jedoch keine Antwort auf verbleibende ungeplante Störungen. STRAUB beschreibt drei Säulen zur Realisierung kürzerer Anläufe: Einsatz von Anlaufteams, die organisatorische Einbindung der Teams in die Organisation und der Einsatz eines Methodenbaukastens. Letzterer ist für das \gls{lsu} von Bedeutung. 

Grundgedanke des Methodenbaukastens ist der Einsatz moderner Methoden, Werkzeuge und Standards.
Zum einen wird eine erhöhte Effizienz und Transparenz bewirkt. Zum anderen wird eine objektive Bewertung von Situationen und damit ein einheitliches Verständnis erreicht, was insbesondere die Zusammenarbeit mit jüngeren und unerfahrenen Mitarbeitern erleichtert. 
Des weiteren wird der Einsatz einer Scorecard empfohlen. %TODO Bild 6 in Anhang oder hier. 
Zunächst werden quantifizierbare Anlaufindikatoren definiert. Mit Hilfe der Scorecard werden die wichtigsten Anlaufindikatoren kontinuierlich überwacht und Abweichungen vom Soll Wert werden schnell erkannt. Es folgt eine systematische Ursachenanalyse. So kann eine schnelle und zielgerichtete Reaktion gewährleistet werden. 


\section{Produktentwicklung}
\subsection*{Harjes-2004 - Produktdatenmodell}
HARJES et al. untersucht das Anlaufmanagement mit besonderer Berücksichtigung des Produktentstehungsprozesses \cite{Harjes2004}. 
Die stetige Reduzierung der eigenen Wertschöpfungstiefe erfordert eine hohe Transparenz bzgl. der Produktdaten. Dies erfolgt mit dem Aufbau digitaler Produktdatenmodelle, welche stets einen echten, plausiblen und aktuellen Datenstand aufweisen müssen. 
Integrierte Produktdatenmodelle (\gls{ipdm})
verknüpfen Produkt- und Prozessdatenmodelle. Damit bekommen Änderungen mehrdimensionalen Charakter, betroffene Komponenten können identifiziert werden und die Folgen lassen sich simulieren und bewerten. 
Weiterführend wird die digitale Fabrik genannt, die die digitale Planung einer Fertigungsfabrik mit Integration aller Produkt- und Prozessdaten beschreibt. 

Digitale Produktdatenmodelle sorgen unternehmensübergreifend und -intern für erhöhte Transparenz und bessere Folgenabschätzung von Änderungen. 

\subsection*{Christensen-2016 - Produktdaten}
CHRISTENSEN et al. untersucht, inwieweit Lean Prinzipien und Methoden auf den Produktionsanlauf übertragbar sind \cite{Christensen2016}. Schwerpunkte der Arbeit sind Qualität und Lernprozesse. Die Ergebnisse der Arbeit werden in einem Framework zusammengefasst. 
% \subsection{Produktdaten}
Detaillierte und transparente Produktdaten wirken Fehlern entgegen, die aus mangelhafter Dokumentation resultieren. 
Für Produktdaten und Arbeitsanweisungen sollten Standards erarbeitet und umgesetzt werden. Ein verankerter Lernprozess unterstützt den kontinuierlichen Verbesserungsprozess und verringert die Anzahl unvorhersehbarer Störungen. 

\section{Wissensmanagement}
\subsection*{Reichwald-2004}
% Kommentar: 
% Der Ansatz des Feedback-Loops passt schon gut zum LSU Konzept. Hervorzuheben ist z.B. der zielgerichtete Einsatz von Software, auch wenn nicht Abteilungen oder Prozesse zu überwinden sind. Betont werden soll jedenfalls, dass der Einsatz von Software hier nur unterstützend ist und den Menschen nicht von seiner Verantwortung entbindet. 
REICHWALD et al. untersucht das Projektmanagement im Feldanlauf und fokussiert sich dabei auf ein nachhaltiges Wissensmanagement \cite{Reichwald2004}. Ein durchgängiges und effizientes Wissensmanagement ist ein wesentlicher Bestandteil erfolgreicher Anläufe \cite{Kuhn2002}. % TODO check primary source Kuhn2002

REICHWALD et al. unterteilt den Wissensmanagementprozess in folgende Bestandteile: Identifizierung der Wissenslücken, Wissenserwerb und Wissensentwicklung, Wissensverteilung und Wissensbewahrung. Die ersten vier Bestandteile werden in Folgendem kurz vorgestellt. 

\subsubsection*{Identifizierung der Wissenslücken}
Bereits vor Markteinführung muss bekannt sein, inwieweit das Produkt den Kundenerwartungen entspricht. Besonders die Produktmerkmale, die unterhalb der Kundenerwartungen liegen müssen identifiziert werden. Dazu eignen sich realitätsnahe Produkttests mit einer dem zukünftigem Kundenkreis entsprechenden Gruppe. Es sind geeignete Erhebungsinstrumente auszuwählen und weiterzuentwickeln. Um möglichst viele Qualitätsaspekte zu berücksichtigen müssen die Erfahrungen der Testpersonen über ein breites Spektrum erfasst werden. Darüber hinaus müssen die Ergebnisse einfach auszuwerten sein um eine schnelle Berücksichtigung zu gewährleisten. Dazu eignen sich z.B. standardisierte Fragebögen oder kurze mündliche Befragungen. 

\subsubsection*{Wissenserwerb, -entwicklung und -verteilung}
Nach erfolgter Produkttests werden die Ergebnisse ausgewertet. Die einzelnen Ergebnisse werden in ein sogenanntes E-Workflowsystem eingepflegt, welches eine Art \gls{erp} System darstellt.
Dieses System ist in der Lage das gesammelte Faktenwissen entlang der gesamten Prozesskette bereitzustellen. Die bei der Auswertung der Tests gewonnenen Schwerpunkte bilden die Handlungsfelder des Anlaufteams. Mit Hilfe des E-Workflowsystems werden den jeweiligen Handlungsfeldern Maßnahmen und Zuständigkeiten sowie Umsetzungstermine zugeteilt. 
Nach der Umsetzung der Maßnahmen muss die Wirksamkeit möglichst durch die gleichen Personen bestätigt werden, die im Vorfeld die Handlungsfelder aufgezeigt haben. Wird die Wirksamkeit bestätigt, ist das Handlungsfeld erfolgreich abgeschlossen. 
Sind Erkenntnisse des laufenden Projekts auch für zukünftige Projekte von Bedeutung, so sollten sie im E-Workflowsystem gesondert gekennzeichnet und in zukünftigen Entwicklungsprozessen eingegliedert werden. 


\subsubsection*{Kurzzusammenfassung}

Ein durchgängiges und effizientes Wissensmanagement ist ein wesentlicher Bestandteil erfolgreicher Anläufe. Dabei werden frühzeitig Kundenrückmeldungen zur Produktverbesserung ausgewertet und die Arbeit mit Softwaresystemen unterstützt. 



\section{Qualitätsmanagement}

\section{Risikomanagement}

\subsection*{Wildemann-2004 - Präventive Handlungsstrategien für den Produktionsanlauf}

WILDEMANN adaptiert Prinzipien und Methoden des klassischen Risikomanagements für den Produktionsanlauf und leitet anschließend Handlungsempfehlungen ab \cite{Wildemann2004}. Die technischen und organisatorischen Risiken werden soweit minimiert, dass ein sog. anlaufrobustes Produktionssystem erreicht wird. 

\subsubsection*{Risikobetrachtung im Anlaufmanagement}
* noch nicht fertig gestellt * % TODO to be finished

\subsubsection*{Handlungsempfehlungen für die Risikohandhabung im Produktionsanlauf}

\textbf{Risikoidentifikation und Bewertung}
Risiken lassen sich zu Produktionssystemelementen zuordnen. Dazu gehören Personal, Material, Prozesse, Anlagen, IT und Infrastruktur. Anschließend werden die Risiken unter dem Aspekt der Auswirkungen klassifiziert. Leistungsrisiken sorgen für ein geringeres Leistungsvermögen des Produktionssystems:
% \cite{Wildemann2004}: 
\begin{itemize}
 \item Versorgungsengpässe bei Material,
Hilfs- und Betriebsstoffen,
\item Kapazitätsengpässe bei Personal und
Maschinen,
\item Instabilitäten und Ineffizienzen in
den Prozessen der Herstellung,
Logistik und Administration.
\end{itemize}
Kostenrisiken führen bei gleicher Leistung zu erhöhtem Aufwand und somit zu Mehrkosten: %TODO cite [4] Wiendahl, H.-P., Hegenscheidt, M.,
% Winkler, H.: Anlaufrobuste Produktions-
% systeme. In: wt werkstattstechnik 92
% (2002) 11/12, S. 650-655.
% keine Printexemplare ab 2000, kein Online Zugriff in Berlin, nur TH Wildau!!
\begin{itemize}
 \item Mehrkosten in der Bau- und Installationsphase,
\item Ausschuss- und Mehrarbeitskosten,
\item zusätzliche Logistikkosten aufgrund
eines höheren Handlingaufwands
und höherer Bestände,
\item zusätzliche Personalkosten durch
Überstunden und steigenden Koordinationsbedarf.
\end{itemize}

Um die Risiken zu bewerten, müssen sie quantifiziert werden. Während die Kostenrisiken direkt bewertet werden können, werden Leistungsrisiken im Rahmen der Szenarioberechnung bewertet. 

\textbf{Bildung risikobezogener Anlaufszenarien}
Bei der Bewertung risikobezogener Anlaufszenarien ist zu beachten, dass potentielle Risiken simultan auftreten und deren Wirkung sich addieren kann. Die Systemleistung ergibt sich aus dem schwächsten Glied, dem sogenannten dominierenden risikobedingten Engpass. Ziel ist zunächst, diese risikobedingten Engpässe mithilfe einer Berechnung von Anlaufszenarien zu identifizieren. %TODO cite [6] Fleischer, Spath. QSimulation im Serienanlauf 
Dabei werden Kapazitätsverfügbarkeit und -bedarf ermittelt und anschließend gegenübergestellt. Folgende Faktoren werden für die Kapazitätsverfügbarkeit hinzugezogen: Personal, Maschinen, Material, technische Anlageneffizienz, organisatorische Effizienz sowie die Lernkurve im Anlauf. % TODO ggf. Grafik 2 einfügen
Für die Ermittlung des Kapazitätsbedarfs werden Absatzmengen und der Produktmix hinzugezogen. 
% TODO Risikokritizität und -sensivität nicht beschrieben. 

\textbf{Ableitung von Handlungsstrategien und Gestaltungsregeln}

Im Folgenden werden vier Handlungsstrategien vorgestellt, die sich% TODO 

Handlungsstrategie 1: Systemrobustheit erhöhen: 

Handlungsstrategie 1 ist angezeigt, wenn eine geringe Veränderung der Eingangsparameter für erhebliche Schwankungen der Systemleistung sorgt. Da sie eine strukturelle Gestaltungskomponente ist, ist sie vor \gls{sop} anzuwenden. 
Robuste Systeme erbringen die gewünschte Leistung auch bei Schwankungen der Rahmenbedingungen. 
Erreicht wird dies durch gezielten Aufbau von Redundanzen und den Einsatz flexibler Anlagenkonzepte.
Bei der Beschaffung erhöhen \gls{bspw} eine hohe Informationstransparenz und Multi-Supplier-Konzepte die Systemrobustheit. 


\section{Änderungen}

\section{Kooperationen}

\section{Lieferanten}

\section{Logistik}

\section{Produktion}

\subsection*{Harjes-2004 - Robuste Produktionssysteme}
HARJES et al. untersucht das Anlaufmanagement mit besonderer Berücksichtigung des Produktentstehungsprozesses \cite{Harjes2004}. 

Höhere Variantenvielfalt und Individualisierungswünsche der Kunden stellen hohe Anforderungen an Fertigungs- und Montagelinien. Zunächst ist eine Standardisierung erforderlich. Produktionssysteme sollten einfach und übertragbar gestaltet werden. Daraus erfolgt eine erhöhte Flexibilität bei Integration neuer Baureihen und Komponenten, Änderungen können somit reibungsloser implementiert werden. Um Auswirkungen vom Prozess oder Produkt auf das Produktionssystem frühzeitig bewerten zu können, sind Prozess- und Produktdaten standardisiert zu verknüpfen und stets aktuell zu halten. 

Robuste Produktionssysteme reagieren agil auf Änderungen und können flexibel erweitert werden. 

\subsection*{Reinfelder-2004 - Planung anlaufrobuster Produktionssysteme}
% **STRUKTUR**
% Einführung
% Definition 
% Ziele
% Bestandteile
% Enabler

REINFELDER behandelt Aspekte für die Planung anlaufrobuster Produktionssysteme insbesondere während des Anlaufs mithilfe der Digitalen Fabrik \cite{Reinfelder2004}. 
Grundidee ist, Fertigungssysteme so auszulegen, dass sie ein Maximum an Flexibilität und Transparenz bieten. Mithilfe von Flexibilität können Defizite in noch nicht eingeschwungenen Fertigungssystemen ausgeglichen, sowie unkompliziert Änderungen vorgenommen werden. Transparenz dient hier dem Erreichen einer steilen Lernkurve. 

\subsubsection{Schnelle Erstellung von Planungsalternativen}

Starke Schwankungen der dynamischen Randbedingungen wie z.B. Verkaufszahlen machen Anpassungen des Produktionssystems in kurzen Abständen erforderlich. Dies gilt insbesondere für den Produktionsanlauf. Dabei entsteht ein Spannungsfeld zwischen Änderungsintervallen und Detaillierungsgrad der Planung. Um gleichzeitig in kürzeren Abständen und höherem Detaillierungsgrad verschiedene Planungsalternativen erstellen zu können, muss der Planungsaufwand beim Einsatz der Digitalen Fabrik minimiert werden. 
% Dabei haben sich folgende Maßnahmen als Sinnvoll herausgestellt: 
Dazu werden folgende Maßnahmen vorgeschlagen: 
Die Einführung und Nutzung von Standards verringert die Komplexität (Auswahl) und sorgt für eine Zeitersparnis. Bibliotheken für Maschinen und Anlagen sind eine Art der Standardisierung und verhindern Redundanzen. Zuletzt wird die Automatisierung von Routinetätigkeiten wie \gls{bspw} Auswertungen wie Kosten- oder Belegrechnung genannt.
Durch die zuvor genannten Maßnahmen sorgen dafür, dass die Mitarbeiter mehr Zeit in Kreativitäts- anstatt von Verwaltungsaufgaben einsetzen können. Sie können dadurch in kürzerer Zeit mehr und qualitativ bessere Planungsalternativen erstellen. 

\subsubsection{Erfolgsfaktoren}

Da die Werkzeuge der Digitalen Fabrik auf dem Markt frei verfügbar sind, sind Wettbewerbsvorteile nur noch durch den richtigen Einsatz im Unternehmen zu erreichen. 
REINFELDER betont die Verankerung im Unternehmen. Zum einen wird die Verankerung des Geschäftsprozesses für die Fertigungsplanung, also die Methoden und Werkzeuge, in den Unternehmensstrukturen genannt. Wichtiger jedoch ist die Verankerung in den Köpfen der Mitarbeiter. 
Weiterhin sollen Planungsinhalte in frühe Planungsphasen verlagert, Planungsleistung parallelisiert und Abteilungsübergreifendes Arbeiten forciert werden. Die Planungssoftware sollte auf die Bedürfnisse des Unternehmens angepasst, und in die Datenstruktur des Unternehmens eingebunden werden. 

\subsubsection{Einordnung}

Besonders in schnell wachsenden Unternehmen sind die dynamischen Randbedingungen volatiler. Schnelle und häufige Änderungen der Stückzahlen erfordern Änderungen des Produktionssystems in kurzen Intervallen. Häufige Produkt- und Prozessänderungen sind grundlegende Bestandteile des Lean Start-up und erfordern die gleichzeitige Generierung von mehreren Planungsalternativen. 

% \section{JoergHinrichs-2004 - Kollaboratives Anlaufmanagement}


% #####
% \chapter{Quellenzusammenfassung - temporär}
% 
% \section{Harjes-2004}
% 
% HARJES et al. untersucht das Anlaufmanagement mit besonderer Berücksichtigung des Produktentstehungsprozesses \cite{Harjes2004}. 
% 
% \subsection{Kurzzusammenfassung}
% 
% 
% \section{Berg-2006}
% 
% 
% 
% \section{Christensen-2016 - Lean Application to Manufacturing Ramp-up}
% 
% 
% \subsection{Qualität}
% Qualität ist ein wichtiger Indikator für die Marktreife des Produkts. Ein hohes Qualitätsniveau soll in kürzester Zeit erreicht werden, was bei immer kürzeren Produktlebenszyklen eine hohe Herausforderung darstellt. 
% 
% Die Mitarbeiter sollen dazu motiviert werden, mit Hilfe von Versuchen den kontinuierlichen Verbesserungsprozess zu unterstützen. Ferner sollen Qualitätsprobleme möglichst früh im Anlaufprozess identifiziert und beseitigt werden. 
% 
% \subsection{Zeit}
% Schnellere Produktionsprozesse und kürzere Taktzeiten erhöhen den Einfluss menschlicher und technischer Fehler. 
% 
% Vermeidung verschwenderischer Aktivitäten und Fokussierung auf Wertschöpfende Tätigkeiten ermöglichen eine höhere Qualität bei gleichzeitiger Zeitersparnis. 
% 
% \subsection{Kommunikation}
% Mangelnde Kommunikation stellt einen erheblichen Störfaktor im Serienanlauf dar. 
% 
% Standardisierte Kommunikation und Informationsflüsse in Kombination mit Lean Techniken wie z.B. \gls{obeya} Meetings überwinden das Abteilungsdenken. 
% 
% \subsection{Lieferanten}
% Die Leistung einer Lieferkette zeigt sich erst im Zusammenspiel mit allen Komponenten und Lieferanten. 
% Bevor Änderungen in der Lieferkette durchgeführt werden, müssen die verantwortlichen Mitarbeiter die konsequente Ausrichtung nach Lean Prinzipien gewährleisten. 
% 
% \subsection{Qualifizierung u. Personal}
% Eine feste Zuordnung von Verantwortlichkeiten kann die Geschwindigkeit und Qualität von Entscheidungen erhöhen. 
% Feste Zuordnung von Verantwortlichkeiten sollte bis in die unterste Ebene auf den Shopfloor reichen.
% 
% 
% \subsection{Engpässe}
% Engpässe beeinträchtigen die Anlaufperformance und sind schwer vorherzusagen. 
% 
% Mit Hilfe systematischer Identifikation und Beseitigung verschwenderischer Aktivitäten können Engpässe vermieden und die Produktionsleistung geglättet werden. 
% 
% 

