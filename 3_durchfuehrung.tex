\chapter{Durchführung}

\section{Reichwald-2004}
% Kommentar: 
% Der Ansatz des Feedback-Loops passt schon gut zum LSU Konzept. Hervorzuheben ist z.B. der zielgerichtete Einsatz von Software, auch wenn nicht Abteilungen oder Prozesse zu überwinden sind. Betont werden soll jedenfalls, dass der Einsatz von Software hier nur unterstützend ist und den Menschen nicht von seiner Verantwortung entbindet. 
REICHWALD et al. untersucht das Projektmanagement im Feldanlauf und fokussiert sich dabei auf ein nachhaltiges Wissensmanagement \cite{Reichwald2004}. Ein durchgängiges und effizientes Wissensmanagement ist ein wesentlicher Bestandteil erfolgreicher Anläufe \cite{Kuhn2002}. % TODO check primary source Kuhn2002

REICHWALD et al. unterteilt den Wissensmanagementprozess in folgende Bestandteile: Identifizierung der Wissenslücken, Wissenserwerb und Wissensentwicklung, Wissensverteilung und Wissensbewahrung. Die ersten vier Bestandteile werden in Folgendem kurz vorgestellt. 

\subsection{Identifizierung der Wissenslücken}
Bereits vor Markteinführung muss bekannt sein, inwieweit das Produkt den Kundenerwartungen entspricht. Besonders die Produktmerkmale, die unterhalb der Kundenerwartungen liegen müssen identifiziert werden. Dazu eignen sich realitätsnahe Produkttests mit einer dem zukünftigem Kundenkreis entsprechenden Gruppe. Es sind geeignete Erhebungsinstrumente auszuwählen und weiterzuentwickeln. Um möglichst viele Qualitätsaspekte zu berücksichtigen müssen die Erfahrungen der Testpersonen über ein breites Spektrum erfasst werden. Darüber hinaus müssen die Ergebnisse einfach auszuwerten sein um eine schnelle Berücksichtigung zu gewährleisten. Dazu eignen sich z.B. standardisierte Fragebögen oder kurze mündliche Befragungen. 

\subsection{Wissenserwerb, -entwicklung und -verteilung}
Nach erfolgter Produkttests werden die Ergebnisse ausgewertet. Die einzelnen Ergebnisse werden in ein sogenanntes E-Workflowsystem eingepflegt, welches eine Art \gls{erp} System darstellt.
Dieses System ist in der Lage das gesammelte Faktenwissen entlang der gesamten Prozesskette bereitzustellen. Die bei der Auswertung der Tests gewonnenen Schwerpunkte bilden die Handlungsfelder des Anlaufteams. Mit Hilfe des E-Workflowsystems werden den jeweiligen Handlungsfeldern Maßnahmen und Zuständigkeiten sowie Umsetzungstermine zugeteilt. 
Nach der Umsetzung der Maßnahmen muss die Wirksamkeit möglichst durch die gleichen Personen bestätigt werden, die im Vorfeld die Handlungsfelder aufgezeigt haben. Wird die Wirksamkeit bestätigt, ist das Handlungsfeld erfolgreich abgeschlossen. 
Sind Erkenntnisse des laufenden Projekts auch für zukünftige Projekte von Bedeutung, so sollten sie im E-Workflowsystem gesondert gekennzeichnet und in zukünftigen Entwicklungsprozessen eingegliedert werden. 


\subsection{Kurzzusammenfassung}

Ein durchgängiges und effizientes Wissensmanagement ist ein wesentlicher Bestandteil erfolgreicher Anläufe. Dabei werden frühzeitig Kundenrückmeldungen zur Produktverbesserung ausgewertet und die Arbeit mit Softwaresystemen unterstützt. 

\section{Harjes-2004}

HARJES et al. untersucht das Anlaufmanagement mit besonderer Berücksichtigung des Produktentstehungprozesses \cite{Harjes2004}. 

\subsection{Robuste Produktionssysteme}

Höhere Variantenvielfalt und Individualisierungswünsche der Kunden stellen hohe Anforderungen an Fertigungs- und Montagelinien. Zunächst ist eine Standardisierung erforderlich. Produktionssysteme sollten einfach und übertragbar gestaltet werden. Daraus erfolgt eine erhöhte Flexibilität bei Integration neuer Baureihen und Komponenten, Änderungen können somit reibungsloser implementiert werden. Um Auswirkungen vom Prozess oder Produkt auf das Produktionssystem frühzeitig bewerten zu können, sind Prozess- und Produktdaten standardisiert zu verknüpfen und stets aktuell zu halten. 

\subsection{Produktdatenmodell}
Die stetige Reduzierung der eigenen Wertschöpfungstiefe erfordert eine hohe Transparenz bzgl. der Produktdaten. Dies erfolgt mit dem Aufbau digitaler Produktdatenmodelle, welche stets einen echten, plausiblen und aktuellen Datenstand aufweisen müssen. 
Integrierte Produktdatenmodelle (\gls{ipdm})
verknüpfen Produkt- und Prozessdatenmodelle. Damit bekommen Änderungen mehrdimensionalen Charakter, betroffene Komponenten können identifiziert werden und die Folgen lassen sich simulieren und bewerten. 
Weiterführend wird die digitale Fabrik genannt, die die digitale Planung einer Fertigungsfabrik mit Integration aller Produkt- und Prozessdaten beschreibt. 

\subsection{Kurzzusammenfassung}
Robuste Produktionssysteme reagieren agil auf Änderungen und können flexibel erweitert werden. Digitale Produktdatenmodelle sorgen unternehmensübergreifend und -intern für erhöhte Transparenz und bessere Folgenabschätzung von Änderungen. 

\section{Straub-2006}

STRAUB et al. verfolgt die Vision der Umstellung der Produktion von \gls{sop} 
auf Kammlinie an einem Wochenende. Im Fokus seiner Arbeit steht die schnelle und richtige Reaktion auf ungeplante Störungen im Anlauf \cite{Straub2006}. Die bisher eingesetzte präventive Methode der digitalen Fabrik erhöht zwar signifikant den Reifegrad der Planung, bietet jedoch keine Antwort auf verbleibende ungeplante Störungen. STRAUB beschreibt drei Säulen zur Realisierung kürzerer Anläufe: Einsatz von Anlaufteams, die organisatorische Einbindung der Teams in die Organisation und der Einsatz eines Methodenbaukastens. Letzterer ist für das \gls{lsu} von Bedeutung. 

\subsection{Methodenbaukasten}

Grundgedanke des Methodenbaukastens ist der Einsatz moderner Methoden, Werkzeuge und Standards.
Zum einen wird eine erhöhte Effizienz und Transparenz bewirkt. Zum anderen wird eine objektive Bewertung von Situationen und damit ein einheitliches Verständnis erreicht, was insbesondere die Zusammenarbeit mit jüngeren und unerfahrenen Mitarbeitern erleichtert. 
Des weiteren wird der Einsatz einer Scorecard empfohlen. %TODO Bild 6 in Anhang oder hier. 
Zunächst werden quantifizierbare Anlaufindikatoren definiert. Mit Hilfe der Scorecard werden die wichtigsten Anlaufindikatoren kontinuierlich überwacht und Abweichungen vom Soll Wert werden schnell erkannt. Es folgt eine systematische Ursachenanalyse. So kann eine schnelle und zielgerichtete Reaktion gewährleistet werden. 

\section{Berg-2006}

\section{Quasdorff-2016 - Lean Management und Digitale Fabrik}

QUASDORFF et al. behandelt die Schnittmengen von Lean Management und der Digitalen Fabrik \cite{Quasdorff2016}. 

Die Digitale Fabrik umfasst die Abbildung und Simulation von Produkt, Prozess und Ressourcen in einem Informationssystem. Während für die Digitale Fabrik die Datenbasis für den Erfolg ausschlaggebend ist, muss die Lean Philosophie aktiv im Unternehmen gelebt werden. Beim gleichzeitigen Einsatz beider Methoden sind große Synergieeffekte zu erwarten. 

Die Digitale Fabrik unterstützt die Vermeidung von Muda (Verschwendung), Mura (Unausgeglichenheit) und Muri (Überbeanspruchung). Durch die zunehmende Digitalisierung (Industrie 4.0 bzw. \gls{iot}) wächst die Bedeutung von Quellen der Verschwendung im Bereich der Informationstechnik und der Datenverarbeitung. 

Bei der Gestaltung der Digitalen Fabrik müssen einige Aspekte beachtet werden. So ist die konsequente Anwendung von Lean Prinzipien Voraussetzung für die Digitale Fabrik. Schlanke Prozesse und deren Standardisierung sorgen dafür, dass die Komplexität der Modelle der Digitalen Fabrik beherrschbar wird. 
Verbesserungen an ineffizienten Prozessen sollten am Prozess als solchen ansetzen anstatt verbesserte Technologie einzusetzen. % TODO cite [8]

Erfolgsfaktoren sind eine hohe Detailtreue und Datenqualität. Das Modell sollte zu jedem Zeitpunkt der Realität entsprechen. Dennoch sollte vor einer Änderung der Ist-Zustand mit dem Dokumentationszustand verglichen werden. 

Abschließend ist zu bemerken, dass der Einsatz der Digitalen Fabrik den Gang in den Shopfloor nicht ersetzen sondern nur unterstützen kann. 

\textbf{Einordnung:} Der Artikel liefert Ansätze für die Gestaltung der DF im LSU. Dabei muss stets auf den Angemessenen Einsatz der Methoden geachtet werden. Ggf. sollten nur einige kritische Elemente Einzug in die DF erhalten. Dennoch muss das Modell zu jedem Zeitpunkt aktuell und plausibel sein. Auch die Detailtreue muss im Anfangsstadium nicht zwingend maximiert werden sondern den Zweck erfüllen einen hohen Reifegrad in der frühen Planungsphase zu erreichen und viele Entscheidungen möglichst früh treffen zu können. 

\section{Schwarz-2017 - Reifegradmodell für Lean Production}

SCHWARZ et al. entwickelt ein Reifegradmodell zur Bewertung des Implementierungsfortschritts von Lean Production im Unternehmen \cite{Schwarz2017}. Einfache Befragungen eignen sich aufgrund der Komplexität nicht zur Bewertung. 

\subsection{Bestandteile}
Das von SCHWARZ entwickelte Reifegradmodell erfasst den Fortschritt in den zwei Dimensionen Methodenkompetenz und Unternehmenskultur. 
Methodenkompetenz beschreibt die Fähigkeit eines produzierenden Unternehmens die Prinzipien der Lean Philosophie durch Anwendung spezifischer Methoden systematisch und gezielt im Produktionssystem umzusetzen. 
Die folgenden fünf Lean Prinzipien dienen als Grundlage für das Modell: Kundennutzen, Wertstrom, Fluss, Pull und Perfektion. 
Es existieren zahlreiche Methoden die jeweils ein oder mehrere Prinzipien umsetzen. 

Voraussetzung für den nachhaltigen Einsatz von Lean Prinzipien ist das aktive Leben der Ideen sowie die Verankerung in der Unternehmenskultur. % TODO Cite [7] Baumgärtner (nicht verfügbar), alt. Quelle suchen
Die Unternehmenskultur wird mit Hilfe folgender Aspekte beschrieben: 
\begin{itemize}
 \item Grundlegende Annahmen und Überzeugungen
\item Implizite und explizite Werte 
\item Mittel zur Verwirklichung dieser Werte
\item die Außenwirkung.
\end{itemize}
Die Ausprägungen der Aspekte sind Voraussetzungen für eine nachhaltige und langfristige Anwendung der Methoden durch die Mitarbeiter und somit für die Implementierung der Lean Prinzipien. 

\textbf{Gestaltung}

Die Bewertung der Reifegrade in den zwei Dimensionen erfolgt in sechs Stufen. % TODO siehe Grafik. 


Für die Dimension Methodenkompetenz erfolgt die Bewertung mit Hilfe von 13 Fragen. Abgefragt werden Eigenschaften, die auf den Implementierungsgrad abzielen. \Gls{bspw} wird die Qualifizierung der Mitarbeiter und Führungskräfte oder der Einfluss der Kundenforderungen auf die Produktion abgefragt. 
Auch wird der Einsatz bestimmter Methoden Reifegraden zugeordnet. So wird der Einsatz von \gls{smed} der Stufe 2 (``Wissend'') und der Einsatz von \gls{heijunka} der Stufe 4 (``Etabliert/Gesichert'') zugeordnet. 

Für die Dimension Unternehmenskultur wird anhand von sieben Fragen ermittelt, inwieweit die Lean Production in der Unternehmenskultur verankert ist. 

\Gls{bspw} wird abgefragt, inwieweit die 5 Lean Prinzipien in der Unternehmensphilosophie verankert, kommuniziert und verstanden ist (``Annahme und Überzeugung'') oder inwieweit die Implementierung von den Führungskräften unterstützt wird (``Werte''). 

\begin{figure}[!ht] 
    \begin{minipage}{0.3\linewidth} 
    \begin{center}
      \includegraphics[scale=.27]{./img/schwarz2017:rg.png}
 % schwarz2017:rg.png: 0x0 pixel, 300dpi, 0.00x0.00 cm, bb=
    \end{center}
      \caption{Die sechs Reifegradstufen \cite{Schwarz2017}}\label{fig:links} 
    \end{minipage} 
    \hfill 
    \begin{minipage}{0.6\linewidth} 
 \includegraphics[scale=.3]{./img/schwarz2017:portfolio.png}
 % schwarz2017:portfolio.png: 0x0 pixel, 300dpi, 0.00x0.00 cm, bb=
    \caption{Portfoliodarstellung mit zwei Dimensionen \cite{Schwarz2017}}\label{fig:rechts} 
    \end{minipage} 
  \end{figure} 

\subsection{Durchführung}
Die tatsächliche Durchführung teilt sich auf in: Befragung, Detaildarstellung der Ergebnisse aller 20 Themen, Analyse und Ableitung von Verbesserungspotentialen und Ausarbeitung eines Maßnahmeplans für die Realisierung. 
%
% \textbf{Befragung}
Zunächst erfolgt die Befragung bei der eine Einschätzung zu jedem der 20 Themen stattfindet. Dazu werden \gls{bspw} Führungskräfte und ggf. externe Personen befragt. 

% \textbf{Auswertung}
Die Auswertung erfolgt in drei Schritten. Zunächst werden die arithmetischen Mittel der Antworten für jedes Thema ermittelt. Große Abweichungen untereinander deuten auf eine unausgewogene Entwicklung hin und es besteht punktueller Nachholbedarf. Im nächsten Schritt wird der Mittelwert über alle 20 Themen ermittelt. Dieser stellt den aktuellen Reifegrad des Unternehmens in Bezug auf Lean Production insgesamt dar. Im dritten Schritt werden die Mittelwerte der zwei Dimensionen miteinander in Bezug gesetzt. % TODO Grafik

Eine Abweichung größer als eine Reifegradstufe wird als kritisch bewertet und deutet auf eine einseitige Implementierung hin. Abhilfe schafft hier die Anpassung des Ressourceneinsatzes. 

\textbf{Einordnung:} 
Die Überprüfung der Dimension Methodenkompetenz kann das Lean Start-up dabei unterstützen den Einsatz geeigneter Methoden zu steuern. 
Die Überprüfung der Dimension Unternehmenskultur hingegen ist in den frühen Phasen des Lean Start-up wenig sinnvoll. Lean Start-ups bestehen üblicherweise aus kleinen Teams mit flachen Hierarchien, und Identifikation und Motivation ist bei allen Mitarbeitern sehr ausgeprägt.


\section{Christensen-2016 - Lean Application to Manufacturing Ramp-up}

CHRISTENSEN et al. untersucht, inwieweit Lean Prinzipien und Methoden auf den Produktionsanlauf übertragbar sind \cite{Christensen2016}. Schwerpunkte der Arbeit sind Qualität und Lernprozesse. Die Ergebnisse der Arbeit werden in einem Framework zusammengefasst, der in Folgendem skizziert wird. 

\subsection{Qualität}
Qualität ist ein wichtiger Indikator für die Marktreife des Produkts. Ein hohes Qualitätsniveau soll in kürzester Zeit erreicht werden, was bei immer kürzeren Produktlebenszyklen eine hohe Herausforderung darstellt. 

Die Mitarbeiter sollen dazu motiviert werden, mit Hilfe von Versuchen den kontinuierlichen Verbesserungsprozess zu unterstützen. Ferner sollen Qualitätsprobleme möglichst früh im Anlaufprozess identifiziert und beseitigt werden. 

\subsection{Zeit}
Schnellere Produktionsprozesse und kürzere Taktzeiten erhöhen den Einfluss menschlicher und technischer Fehler. 

Vermeidung verschwenderischer Aktivitäten und Fokussierung auf Wertschöpfende Tätigkeiten ermöglichen eine höhere Qualität bei gleichzeitiger Zeitersparnis. 

\subsection{Kommunikation}
Mangelnde Kommunikation stellt einen erheblichen Störfaktor im Serienanlauf dar. 

Standardisierte Kommunikation und Informationsflüsse in Kombination mit Lean Techniken wie z.B. \gls{obeya} Meetings überwinden das Abteilungsdenken. 

\subsection{Lieferanten}
Die Leistung einer Lieferkette zeigt sich erst im Zusammenspiel mit allen Komponenten und Lieferanten. 
Bevor Änderungen in der Lieferkette durchgeführt werden, müssen die verantwortlichen Mitarbeiter die konsequente Ausrichtung nach Lean Prinzipien gewährleisten. 

\subsection{Qualifizierung u. Personal}
Eine feste Zuordnung von Verantwortlichkeiten kann die Geschwindigkeit und Qualität von Entscheidungen erhöhen. 
Feste Zuordnung von Verantwortlichkeiten sollte bis in die unterste Ebene auf den Shopfloor reichen.

\subsection{Produktdaten}
Detaillierte und transparente Produktdaten wirken Fehlern entgegen, die aus mangelhafter Dokumentation resultieren. 
Für Produktdaten und Arbeitsanweisungen sollten Standards erarbeitet und umgesetzt werden. Ein verankerter Lernprozess unterstützt den kontinuierlichen Verbesserungsprozess und verringert die Anzahl unvorhersehbarer Störungen. 

\subsection{Engpässe}
Engpässe beeinträchtigen die Anlaufperformance und sind schwer vorherzusagen. 

Mit Hilfe systematischer Identifikation und Beseitigung verschwenderischer Aktivitäten können Engpässe vermieden und die Produktionsleistung geglättet werden. 

\section{Wildemann-2004 - Präventive Handlungsstrategien für den Produktionsanlauf}

WILDEMANN adaptiert Prinzipien und Methoden des klassischen Risikomanagements für den Produktionsanlauf und leitet anschließend Handlungsempfehlungen ab \cite{Wildemann2004}. Die technischen und organisatorischen Risiken werden soweit minimiert, dass ein sog. anlaufrobustes Produktionssystem erreicht wird. 

\subsection{Risikobetrachtung im Anlaufmanagement}

\subsection{Handlungsempfehlungen für die Risikohandhabung im Produktionsanlauf}

\textbf{Risikoidentifikation und Bewertung}
Risiken lassen sich zu Produktionssystemelementen zuordnen. Dazu gehören Personal, Material, Prozesse, Anlagen, IT und Infrastruktur. Anschließend werden die Risiken unter dem Aspekt der Auswirkungen klassifiziert. Leistungsrisiken sorgen für ein geringeres Leistungsvermögen des Produktionssystems:
% \cite{Wildemann2004}: 
\begin{itemize}
 \item Versorgungsengpässe bei Material,
Hilfs- und Betriebsstoffen,
\item Kapazitätsengpässe bei Personal und
Maschinen,
\item Instabilitäten und Ineffizienzen in
den Prozessen der Herstellung,
Logistik und Administration.
\end{itemize}
Kostenrisiken führen bei gleicher Leistung zu erhöhtem Aufwand und somit zu Mehrkosten: %TODO cite [4] Wiendahl, H.-P., Hegenscheidt, M.,
% Winkler, H.: Anlaufrobuste Produktions-
% systeme. In: wt werkstattstechnik 92
% (2002) 11/12, S. 650-655.
% keine Printexemplare ab 2000, kein Online Zugriff in Berlin, nur TH Wildau!!
\begin{itemize}
 \item Mehrkosten in der Bau- und Installationsphase,
\item Ausschuss- und Mehrarbeitskosten,
\item zusätzliche Logistikkosten aufgrund
eines höheren Handlingaufwands
und höherer Bestände,
\item zusätzliche Personalkosten durch
Überstunden und steigenden Koordinationsbedarf.
\end{itemize}

Um die Risiken zu bewerten, müssen sie quantifiziert werden. Während die Kostenrisiken direkt bewertet werden können, werden Leistungsrisiken im Rahmen der Szenarioberechnung bewertet. 

\textbf{Bildung risikobezogener Anlaufszenarien}
Bei der Bewertung risikobezogener Anlaufszenarien ist zu beachten, dass potentielle Risiken simultan auftreten und deren Wirkung sich addieren kann. Die Systemleistung ergibt sich aus dem schwächsten Glied, dem sogenannten dominierenden risikobedinten Engpass. Ziel ist zunächst, diese risikobedingten Engpässe mithilfe einer Berechnung von Anlaufszenarien zu identifizieren. %TODO cite [6] Fleischer, Spath. QSimulation im Serienanlauf 
Dabei werden Kapazitätsverfügbarkeit und -bedarf ermittelt und anschließend gegenübergestellt. Folgende Faktoren werden für die Kapazitätsverfügbarkeit hinzugezogen: Personal, Maschinen, Material, technische Anlageneffizienz, organisatorische Effizienz sowie die Lernkurve im Anlauf. % TODO ggf. Grafik 2 einfügen
Für die Ermittlung des Kapazitätsbedarfs werden Absatzmengen und der Produktmix hinzugezogen. 
% TODO Risikokritizität und -sensivität nicht beschrieben. 

\textbf{Ableitung von Handlungsstrategien und Gestaltungsregeln}

Im Folgenden werden vier Handlungsstrategien vorgestellt, die sich% TODO 

Handlungsstrategie 1: Systemrobustheit erhöhen: 

Handlungsstrategie 1 ist angezeigt, wenn eine geringe Veränderung der Eingangsparameter für erhebliche Schwankungen der Systemleistung sorgt. Da sie eine strukturelle Gestaltungskomponente ist, ist sie vor \gls{sop} anzuwenden. 
Robuste Systeme erbringen die gewünschte Leistung auch bei Schwankungen der Rahmenbedingungen. 
Erreicht wird dies durch gezielten Aufbau von Redundanzen und den Einsatz flexibler Anlagenkonzepte.
Bei der Beschaffung erhöhen \gls{bspw} eine hohe Informationstransparenz und Multi-Supplier-Konzepte die Systemrobustheit. 

\section{Reinfelder-2004 - Planung anlaufrobuster Produktionssysteme}
% **STRUKTUR**
% Einführung
% Definition 
% Ziele
% Bestandteile
% Enabler

REINFELDER behandelt Aspekte für die Planung anlaufrobuster Produktionssysteme insbesondere während des Anlaufs mithilfe der Digitalen Fabrik \cite{Reinfelder2004}. 
Grundidee ist, Fertigungssysteme so auszulegen, dass sie ein Maximum an Flexibilität und Transparenz bieten. Mithilfe von Flexibilität können Defizite in noch nicht eingeschwungenen Fertigungssystemen ausgeglichen, sowie unkomliziert Änderungen vorgenommen werden. Transparenz dient hier dem Erreichen einer steilen Lernkurve. 

\subsection{Schnelle Erstellung von Planungsalternativen}

Starke Schwankungen der dynamischen Randbedingungen wie z.B. Verkaufszahlen machen Anpassungen des Produktionssystems in kurzen Abständen erforderlich. Dies gilt insbesondere für den Produktionsanlauf. Dabei entsteht ein Spannungsfeld zwischen Änderungsintervallen und Detaillierungsgrad der Planung. Um gleichzeitig in kürzeren Abständen und höherem Detaillierungsgrad verschiedene Planungsalternativen erstellen zu können, muss der Planungsaufwand beim Einsatz der Digitalen Fabrik minimiert werden. 
% Dabei haben sich folgende Maßnahmen als Sinnvoll herausgestellt: 
Dazu werden folgende Maßnahmen vorgeschlagen: 
Die Einführung und Nutzung von Standards verringert die Komplexität (Auswahl) und sorgt für eine Zeitersparnis. Bibliotheken für Maschinen und Anlagen sind eine Art der Standardisierung und verhindern Redundanzen. Zuletzt wird die Automatisierung von Routinetätigkeiten wie \gls{bspw} Auswertungen wie Kosten- oder Belegrechnung genannt.
Durch die zuvor genannten Maßnahmen sorgen dafür, dass die Mitarbeiter mehr Zeit in Kreativitäts- anstatt von Verwaltungsaufgaben einsetzen können. Sie können dadurch in kürzerer Zeit mehr und qualitativ bessere Planungsalternativen erstellen. 

\subsection{Erfolgsfaktoren}

Da die Werkzeuge der Digitalen Fabrik auf dem Markt frei verfügbar sind, sind Wettbewerbsvorteile nur noch durch den richtigen Einsatz im Unternehmen zu erreichen. 
REINFELDER betont die Verankerung im Unternehmen. Zum einen wird die Verankerung des Geschäftsprozesses für die Fertigungsplanung, also die Methoden und Werkzeuge, in den Unternehmensstrukturen genannt. Wichtiger jedoch ist die Verankerung in den Köpfen der Mitarbeiter. 
Weiterhin sollen Planungsinhalte in frühe Planungsphasen verlagert, Planungsleistung parallelisiert und Abteilungsübergreifendes Arbeiten forciert werden. Die Planungssoftware sollte auf die Bedürfnisse des Unternehmens angepasst, und in die Datenstruktur des Unternehmens eingebunden werden. 

\subsection{Einordnung}

Besonders in schnell wachsenden Unternehmen sind die dynamischen Randbedingungen volatiler. Schnelle und häufige Änderungen der Stückzahlen erfordern Änderungen des Produktionssystems in kurzen Intervallen. Häufige Produkt- und Prozessänderungen sind grundlegende Bestandteile des Lean Start-up und erfordern die Generierung von mehreren Planungsalternativen. 




