\section{Durchführung}

\subsection{Reichwald-2004}
REICHWALD et. al untersucht das Projektmanagement im Feldanlauf und fokussiert sich dabei auf ein nachhaltiges Wissensmanagement \cite{Reichwald2004}. Ein durchgängiges und effizientes Wissensmanagement ist ein wesentlicher Bestandteil erfolgreicher Anläufe \cite{Kuhn2002}. % TODO check primary source Kuhn2002

REICHWALD et. al unterteilt den Wissensmanagementprozess in folgende Bestandteile: Identifizierung der Wissenslücken, Wissenserwerb und Wissensentwicklung, Wissensverteilung und Wissensbewahrung. Die ersten vier Bestandteile werden in Folgendem kurz vorgestellt. 

\subsubsection{Identifizierung der Wissenslücken}
Bereits vor Markteinführung muss bekannt sein, inwieweit das Produkt den Kundenerwartungen entspricht. Besonders die Produktmerkmale, die unterhalb der Kundenerwartungen liegen müssen identifiziert werden. Dazu eignen sich realitätsnahe Produkttests mit einer dem zukünftigem Kundenkreis entsprechenden Gruppe. Es sind geeignete Erhebungsinstrumente auszuwählen und weiterzuentwickeln. Um möglichst viele Qualitätsaspekte zu berücksichtigen müssen die Erfahrungen der Testpersonen über ein breites Spektrum erfasst werden. Darüber hinaus müssen die Ergebnisse einfach auszuwerten sein um eine schnelle Berücksichtigung zu gewährleisten. Dazu eignen sich z.B. standardisierte Fragebögen oder kurze mündliche Befragungen. 

\subsubsection{Wissenserwerb, -entwicklung und -verteilung}
Nach erfolgter Produkttests werden die Ergebnisse ausgewertet. Die einzelnen Ergebnisse werden in ein sogenanntes E-Workflowsystem eingepflegt, welches eine Art ERP System darstellt. % TODO glossary ERP
Dieses System ist in der Lage das gesammelte Faktenwissen entlang der gesamten Prozesskette bereitzustellen. Die bei der Auswertung der Tests gewonnenen Schwerpunkte bilden die Handlungsfelder des Anlaufteams. Mit Hilfe des E-Workflowsystems werden den jeweiligen Handlungsfeldern Maßnahmen und Zuständigkeiten sowie Umsetzungstermine zugeteilt. 
Nach der Umsetzung der Maßnahmen muss die Wirksamkeit möglichst durch die gleichen Personen bestätigt werden, die im Vorfeld die Handlungsfelder aufgezeigt haben. Wird die Wirksamkeit bestätigt, ist das Handlungsfeld erfolgreich abgeschlossen. 
Sind Erkenntnisse des laufenden Projekts auch für zukünftige Projekte von Bedeutung, so sollten sie im E-Workflowsystem gesondert gekennzeichnet und in zukünftigen Entwicklungsprozessen eingegliedert werden. 


\subsubsection{Kurzzusammenfassung}

Ein durchgängiges und effizientes Wissensmanagement ist ein wesentlicher Bestandteil erfolgreicher Anläufe. Dabei werden frühzeitig Kundenrückmeldungen zur Produktverbesserung ausgewertet und die Arbeit mit Softwaresystemen unterstützt. 



