\chapter{Methodik}
\section{Anforderungen an das Anlaufmanagementmodell für das \gls{lsu}}
Damit das zu entwickelnde Modell den Serienanlauf im \gls{lsu} effektiv unterstützt, müssen zunächst einige Anforderungen formuliert werden. 

\textbf{Methodische Anforderungen}

Das zu entwickelnde Anlaufmanagementmodell muss sowohl horizontal als auch vertikal sinnvoll mit dem \gls{lsu} Ansatz kooperieren bzw. integriert werden. 
Die Beseitigung von Verschwendung sowie Anreize zur kontinuierlichen Verbesserung müssen strukturell im Modell verankert sein. 

Besondere Eigenschaften von (Lean-)Start-ups müssen berücksichtigt werden. Dazu zählen \gls{bspw} eine flache Hierarchie, eine kleine Anzahl an Mitarbeitern, Vorhandensein von Generalisten anstatt Spezialisten und Interdisziplinarität der Mitarbeiter und Aufgaben. Daraus werden folgende Forderungen abgeleitet: Eine kleine Anzahl an einfach anzuwendenden Methoden. Die Gestaltung von Ablauforganisation und Prozessen erfolgt mit geringem Detaillierungsgrad. An anderer Stelle soll jedoch mithilfe von Standardisierung die Komplexität der Lösungsalternativen beschränkt werden. Daraus folgt eine hohe Abstraktionsebene des Modells einerseits, andererseits jedoch ein hoher Detaillierungsgrad. 

Des weiteren wird eine Flexibilität des Modells gefordert. So muss das Modell, welches bereits in der Anfangsphase implementiert wird, bei schnellem Wachstum und stark veränderten Bedingungen weiterhin effektiv sein. Dazu zählen \gls{bspw} eine Skalierbarkeit der Methoden hinsichtlich Anzahl der Mitarbeiter sowie Mitarbeiterzuordnung von Kompetenzen und Aufgaben. 

\textbf{Technische Anforderungen}

Auch auf technischer Seite ist Flexibilität gefordert. Die Produktion bzw. der Anlauf müssen agil auf Stückzahlschwankungen reagieren können. Änderungen am Produkt oder die Einführung neuer Varianten müssen einfach und schnell mit hoher Qualität realisiert werden können. Analog dazu müssen Änderungen am Logistiksystem und Produktionslinie effizient durchgeführt werden können. 
Große Unsicherheiten sind ein inhärentes Merkmal des Serienanlaufs. Daher muss ein umfassendes Risikomanagement im Modell verankert sein. 


\section{Entwicklung des Grundgerüsts}
\textbf{Zielstellung}

Der Stand der Wissenschaft zum Thema Anlaufmanagement soll recherchiert und dargestellt werden. Dies erfolgt zunächst nur übergeordnet, indem ca. 10-20 Hauptaspekte identifiziert und priorisiert werden. Diese Hauptaspekte werden geordnet und in einem Grundgerüst abgebildet. Das Grundgerüst dient nun als struktureller und inhaltlicher roter Faden der Arbeit. 
Zunächst bildet er die Systematik für die Literaturrecherche. Dazu werden aus dem Grundgerüst Themengebiete und Stichworte für die Suche abgeleitet. 
Auch die Einordnung der Lösungskonzepte und Methoden erfolgt nach dem Grundgerüst. %TODO ref to chapter? 
Schließlich bildet es die Grundlage für das Ergebnis der Arbeit, das Anlaufmanagementmodell für das \gls{lsu}. % TODO ref to chapter? 

\textbf{Herangehensweise}

Nach kurzer Recherche ist festzustellen, dass bisher keine einheitliche Auffassung des Anlaufmanagements existiert. Vielmehr wurde das Themengebiet von einigen Autoren bisher nur aus individueller Perspektive behandelt. 
\Gls{bspw} hat SCHMITT im Jahre 2015 ein Glossar veröffentlicht, mit dem Ziel, ein einheitliches Verständnis sowie die Grundlage für die wissenschaftliche und praxisnahe Diskussion des Themengebiets zu schaffen \cite{Schmitt2015}. Diese Arbeit bestärkt die Einschätzung des Autors. %TODO Formulierung?? 

In der Konsequenz ist zunächst eine Konsolidierung anhand einschlägiger Literatur (Primär- und Sekundärliteratur) nötig, welche einen möglichst umfassenden Blickwinkel des Themengebiets behandelt. Dazu erfolgt im ersten Schritt eine Erstrecherche zum Thema Anlaufmanagement (engl. manufacturing ramp-up). Aus dem Ergebnis der Erstrecherche müssen nun die geforderten ``einschlägigen'' Quellen identifiziert werden. Dazu wurden zwei Kriterien definiert, welche und/oder verknüpft werden: 
\begin{enumerate}
 \item Der Autor der Quelle erhebt den Anspruch eines Glossars bzw. einer umfassenden Betrachtung wie \gls{bspw} SCHMITT \cite{Schmitt2015}. %TODO Alternativ die Enschätzung meinerseits als solches. 
 \item Eine häufige Zitierung der Quelle, insbesondere im Kontext der Einführung in das Anlaufmanagement. %TODO Daraus wird eine gewisse Reputation abgeleitet. 
\end{enumerate}

Für die weitere Vorgehensweise wurden folgende Quellen identifiziert: 

\begin{table}[h]
\begin{center}
\begin{tabular}{l l l r}
\textbf{Autor} & \textbf{Jahr} & \textbf{Titel} & \textbf{Ref.} \\ \hline
 Kuhn et al.  & 2002 & Fast Ramp-Up - Schneller Produktionsanlauf von Serienprodukten & \cite{Kuhn2002} \\
%  Schuh et a. & 2004 & Fast Ramp-Up. Anlaufstrategien, Deviationsmanagement und Wissensmanagement für den Anlauf & \cite{Schuh2004}  \\
 Bischoff & 2007 & Anlaufmanagement - Schnittstelle zwischen Projekt und Serie & \cite{Bischoff2007} \\
 Schuh et al. & 2008 & Grundlagen des Anlaufmanagements & \cite{Schuh2008} \\
 Schmitt & 2015 & Anlaufmanagement - Begriffe und Definitionen & \cite{Schmitt2015} 
 \end{tabular} 
 \end{center}
\caption{ Auswahl der Quellen für das Grundgerüst} \label{tab:quellengrundgeruest} 
\end{table}


Die in Tabelle \ref{tab:quellengrundgeruest} genannten Quellen wurden mit Hilfe des Tools \gls{atlas} % TODO italic? 
einer qualitativen Auswertung unterzogen. Dazu wurden sog. Codes definiert, welche jeweils einen thematischen Aspekt beschreiben. Für diese Codes wurden in den Quellen dazugehörige Textpassagen identifiziert. In einem weiteren Schritt wurden die Textpassagen zu den Themengebieten (Codes) gegenübergestellt und verglichen. Damit konnten Schnittmengen gefunden und Gruppen gebildet werden. Die Themengebiete wurden in zwei Abstraktionsebenen unterteilt: Konzeptionell und Ausführend.
Mit dem soeben beschriebenen Verfahren wurde die Codestruktur im Verlauf der Analyse angepasst und bildet das Grundgerüst, die Basis für die Abschlussarbeit. 

\section{Grundsätzliche Herangehensweise der Arbeit}


