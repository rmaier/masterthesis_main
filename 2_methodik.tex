\chapter{Methodik}
\section{Anforderungen an das Anlaufmanagementmodell für das \gls{lsu}}
Damit das zu entwickelnde Modell den Serienanlauf im \gls{lsu} effektiv unterstützt, müssen zunächst einige Anforderungen formuliert werden. 

\textbf{Methodische Anforderungen}

Das zu entwickelnde Anlaufmanagementmodell muss sowohl horizontal als auch vertikal sinnvoll mit dem \gls{lsu} Ansatz kooperieren bzw. integriert werden. 
Die Beseitigung von Verschwendung sowie Anreize zur kontinuierlichen Verbesserung müssen strukturell im Modell verankert sein. 

Besondere Eigenschaften von (Lean-)Start-ups müssen berücksichtigt werden. Dazu zählen \gls{bspw} eine flache Hierarchie, eine kleine Anzahl an Mitarbeitern, Vorhandensein von Generalisten anstatt Spezialisten und Interdisziplinarität der Mitarbeiter und Aufgaben. Daraus werden folgende Forderungen abgeleitet: Eine kleine Anzahl an einfach anzuwendenden Methoden. Die Gestaltung von Ablauforganisation und Prozessen erfolgt mit geringem Detaillierungsgrad. An anderer Stelle soll jedoch mithilfe von Standardisierung die Komplexität der Lösungsalternativen beschränkt werden. Daraus folgt eine hohe Abstraktionsebene des Modells einerseits, andererseits jedoch ein hoher Detaillierungsgrad. 

Des weiteren wird eine Flexibilität des Modells gefordert. So muss das Modell, welches bereits in der Anfangsphase implementiert wird, bei schnellem Wachstum und stark veränderten Bedingungen weiterhin effektiv sein. Dazu zählen \gls{bspw} eine Skalierbarkeit der Methoden hinsichtlich Anzahl der Mitarbeiter sowie Mitarbeiterzuordnung von Kompetenzen und Aufgaben. 

\textbf{Technische Anforderungen}

Auch auf technischer Seite ist Flexibilität gefordert. Die Produktion bzw. der Anlauf müssen agil auf Stückzahlschwankungen reagieren können. Änderungen am Produkt oder die Einführung neuer Varianten müssen einfach und schnell mit hoher Qualität realisiert werden können. Analog dazu müssen Änderungen am Logistiksystem und Produktionslinie effizient durchgeführt werden können. 
Große Unsicherheiten sind ein inhärentes Merkmal des Serienanlaufs. Daher muss ein umfassendes Risikomanagement im Modell verankert sein. 


\section{Entwicklung des Grundgerüsts}

\section{Grundsätzliche Herangehensweise der Arbeit}


