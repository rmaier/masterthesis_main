\chapter{Methodik}\label{sec:methodik}
In diesem Abschnitt wird die Methodik der Abschlussarbeit dargelegt. Zunächst erfolgt eine Beschreibung der allgemeinen Herangehensweise beginnend mit der Literaturrecherche. Anschließend wird der rote Faden der Arbeit entwickelt, der aus zwei Teilen besteht. Zuerst werden spezifische Anforderungen an das Anlaufmanagementmodell für das \gls{lsu} definiert. Im zweiten Schritt erfolgt die Entwicklung des Grundgerüsts, welches die weitere Bearbeitung mittels in Beziehung stehender Unteraspekte strukturiert. Während die Anforderungen als Erfolgsfaktoren zu sehen sind, gibt das Grundgerüst die inhaltliche Struktur vor.  

\section{Inhaltlicher Aufbau der Arbeit}
Das Kapitel \ref{sec:einfuehrung} legt die Motivation und Zielsetzung der Arbeit dar. Weiterhin erfolgt zum Verständnis der Aufgabenstellung eine erste Definition zum Thema Anlaufmanagement sowie ein Überblick zu Lean Start-up. 

Kapitel \ref{sec:methodik} beschreibt zunächst den inhaltlichen Aufbau der Arbeit. Die methodische Herangehensweise wird unter Berücksichtigung von Suchstrategie und Forschungsmethodik dargelegt. In einem weiteren Schritt werden spezifische Anforderungen des Lean Start-up an das Anlaufmanagementmodell beschrieben, welche bei der Ausarbeitung berücksichtigt werden sollen. Schließlich wird ein Grundgerüst für das zu entwickelnde Anlaufmanagementmodell entwickelt, welches als struktureller und inhaltlicher roter Faden der Arbeit dient. %TODO Rechtschreibung Roter Faden? Siehe auch weiter unten
Die Entwicklung des Grundgerüsts dient auch der Konsolidierung der verschiedenen Auffassungen des Themengebiets. 

In Kapitel \ref{sec:durchfuehrung} werden mittels qualitativer Literaturanalyse und anhand des zuvor entwickelten Grundgerüsts Lösungskonzepte zusammengetragen. 

Diese werden anschließend in Kapitel \ref{sec:ableitung} zueinander in Beziehung gesetzt und sollen als Gesamtbild in das Lean Start-up integriert werden. 

Kapitel \ref{sec:diskussion}...

Die Arbeit mündet in Kapitel \ref{sec:fazit} mit einem Fazit und Ausblick. Es werden die Grenzen der angewandten Methodik sowie des entwickelten Modells aufgezeigt. Schließlich werden weitere Entwicklungsschritte vorgeschlagen und der Forschungsbedarf identifiziert. 

Weiterhin wird weiterer Forschungsbedarf identifiziert. 

\section{Methodische Herangehensweise der Arbeit}

Zu Beginn der Arbeit erfolgte eine erste Recherche zu den beiden Themenfeldern Anlaufmanagement und Lean Start-up. Die verwendeten Suchbegriffe sind auf Tabelle \ref{tab:algorythm} aufgeführt. Dabei wurden in folgender Rangfolge Suchmaschinen benutzt: 
1. google.com, 2. scholar.google.com, 3. sciencedirect.com, 4. rd.springer.com. Die erste Recherche zu \gls{lsu} genügte den Anforderungen. Die Recherche zu Anlaufmanagement genügte lediglich für die Bildung eines Grundgerüsts, dessen Herangehensweise ausführlich in Abschnitt \ref{sec:grundgeruest} beschrieben wird. Darauf aufbauend erfolgte eine Vertiefung der Literaturrecherche mittels Schneeballsystem. Dazu wurden Literaturlisten der Quellen, die Namen der Verfasser und \gls{bspw} das Graduiertenkolleg Anlaufmanagement (1491-2) der RWTH-Aachen zu Grunde gelegt. 
\begin{table}[h]
\begin{center}
\begin{tabular}{l l}
\textbf{Themenfeld} & \textbf{Algorithmus }\\ \hline
Anlaufmanagement & ('Ramp-up' OR 'Manufacturing' AND 'Ramp-up' OR 'Production' \\ 
& AND 'Ramp-up' OR 'Anlaufmanagement' OR 'Produktion' \\
& AND 'KMU' OR 'Manufacturing' AND 'SME') \\
Lean Start-up & ('Lean' AND 'Startup' OR 'Lean' AND 'Start-up')
 \end{tabular} 
 \end{center}
\caption{Suchalgorithmen für die Literaturrecherche} \label{tab:algorythm} 
\end{table}

In Kapitel \ref{sec:durchfuehrung} erfolgt die Erfassung von Lösungskonzepten mittels qualitativer Literaturanalyse. Die Auswahl der Quellen erfolgt nach folgenden Kriterien: Beiträge aus Fachzeitschriften oder Konferenzen, bei denen die Qualitätssicherung mittels Peer-Review erfolgt. Weiterhin erlaubt sind einzelne Kapitel aus wissenschaftlichen Monographien. Die Anzahl der auszuwertenden Quellen wird auf 20 begrenzt. Für die qualitative Literaturanalyse bildet das Grundgerüst aus Abschnitt \ref{sec:grundgeruest} die Struktur. Zur Präzisierung der Perspektive und Verringerung der Subjektivität werden in folgendem Abschnitt \ref{sec:anforderungen} Anforderungen an das Modell beschrieben. 



\section{Anforderungen an das Anlaufmanagementmodell für das \gls{lsu}}\label{sec:anforderungen}
Damit das zu entwickelnde Modell den Serienanlauf im \gls{lsu} effektiv unterstützt, müssen zunächst einige Anforderungen formuliert werden. 

\subsection*{Methodische Anforderungen}
%TODO falls methodik und technik aufgelöst wird, im weiteren Text Passagen ersetzen, insbesondere im Ausblick!!


Auf Tabelle \ref{tab:eigenschaften_lsu} sind Eigenschaften von \gls{lsu} (und \gls{kmu}) zusammengefasst. Diese bilden die Basis für die Ableitung der methodischen Anforderungen an das Anlaufmanagementmodell. 
\begin{table}[h]
\begin{center}
\begin{tabular}{l l}
\textbf{Eigentschaft} & \textbf{Quelle }\\ \hline
Flache Hierarchie & \cite[10]{Dombrowski2009a} \\ % TODO cite doerler1988
Kleine Anzahl von Beschäftigten & \cite[9]{Dombrowski2009a} \\
Generalisten statt Spezialisten & \cite[9]{Dombrowski2009a}\\
Technisch orientierte Ausbildung der Eigentümer & \cite[47]{Dombrowski2009a}\\
Kaum nachvollziehbare Dokumentation & \cite[1]{Zimolong2006} \\
Mangelnde Führungskenntnisse & \cite[47]{Dombrowski2009a}\\
Mangelnde strategische Orientierung & \cite[48]{Dombrowski2009a}\\
Sehr hohe Risiken & \cite[9]{Ries2011}\\
 \end{tabular} 
 \end{center}
\caption{Eigenschaften von \gls{lsu} und \gls{kmu}} \label{tab:eigenschaften_lsu} 
\end{table}

Aufgrund der flachen Hierarchie und der kleinen Anzahl von Mitarbeitern ist ein niedriger Detaillierungsgrad bei der Abbildung von Ablauforganisation und Prozessen ausreichend \cite[151]{Dombrowski2009a}. 
Die Methoden und Informationssysteme müssen schlank, einfach zu benutzen und zu implementieren sein \cite[4]{Zimolong2007}. % TODO check Mail from Zimolong
Das Anlaufmanagementmodell soll eine ganzheitliche Perspektive und die Gestaltung und Verfolgung von (Modernisierungs-) Prozessen unterstützen. Aufgrund der mangelnden strategischen Orientierung der Unternehmen soll die Strategieformulierung systematisch unterstützt werden \cite[47-48]{Dombrowski2009a}.
Des weiteren wird eine Flexibilität des Modells gefordert. So muss das Modell, welches bereits in der Anfangsphase implementiert wird, bei schnellem Wachstum und stark veränderten Bedingungen weiterhin effektiv sein. Dazu zählen \gls{bspw} eine Skalierbarkeit der Methoden hinsichtlich Anzahl der Mitarbeiter sowie Mitarbeiterzuordnung von Kompetenzen und Aufgaben. 
Das durch hohe Risiken geprägte Umfeld erfordert ein Risikomanagement, das auf die Bedürfnisse des Serienanlaufs angepasst ist. 


\subsection*{Technische Anforderungen}

Auch auf technischer Seite ist Flexibilität gefordert, die in drei Dimensionen kurz skizziert wird. 
Die Produktion bzw. der Anlauf müssen agil auf Stückzahlschwankungen reagieren können (Volumenflexibilität). Die schnelle Integration innovativer Technologien muss sichergestellt werden (Technologieflexibilität). Schließlich erfordert die zunehmende Variantenvielfalt, dass mehrere Varianten auf einer Linie gefertigt werden können (Variantenflexibilität) \cite[22]{Scholz2010}.
Änderungen am Produkt oder die Einführung neuer Varianten müssen einfach und schnell mit hoher Qualität realisiert werden können. Analog dazu müssen Änderungen am Logistiksystem und Produktionslinie effizient durchgeführt werden können.  

\section{Entwicklung des Grundgerüsts}\label{sec:grundgeruest}
\subsection*{Zielstellung}

Der Stand der Wissenschaft zum Thema Anlaufmanagement soll recherchiert und dargestellt werden. Dies erfolgt zunächst nur übergeordnet, indem ca. 10-20 Hauptaspekte identifiziert und priorisiert werden. Diese Hauptaspekte werden geordnet und in einem Grundgerüst abgebildet. Das Grundgerüst dient nun als struktureller und inhaltlicher roter Faden der Arbeit. 
Zunächst bildet er die Systematik für die Literaturrecherche. Dazu werden aus dem Grundgerüst Themengebiete und Stichworte für die Suche abgeleitet. 
Auch die Einordnung der Lösungskonzepte und Methoden erfolgt nach dem Grundgerüst. %TODO ref to chapter? 
Schließlich bildet es die Grundlage für das Ergebnis der Arbeit, das Anlaufmanagementmodell für das \gls{lsu}. % TODO ref to chapter? 

\subsection*{Herangehensweise}\label{sec:herangehensweise_gg}

Nach kurzer Recherche ist festzustellen, dass bisher keine einheitliche Auffassung des Anlaufmanagements existiert. Vielmehr wurde das Themengebiet von einigen Autoren bisher nur aus individueller Perspektive behandelt. 
\Gls{bspw} hat SCHMITT im Jahre 2015 ein Glossar veröffentlicht, mit dem Ziel, ein einheitliches Verständnis sowie die Grundlage für die wissenschaftliche und praxisnahe Diskussion des Themengebiets zu schaffen \cite{Schmitt2015}. Diese Arbeit bestärkt die Einschätzung des Autors. %TODO Formulierung?? 

In der Konsequenz ist zunächst eine Konsolidierung anhand einschlägiger Literatur (Primär- und Sekundärliteratur) nötig, welche einen möglichst umfassenden Blickwinkel des Themengebiets behandelt. Dazu erfolgt im ersten Schritt eine Erstrecherche zum Thema Anlaufmanagement (engl. manufacturing ramp-up). Aus dem Ergebnis der Erstrecherche müssen nun die geforderten ``einschlägigen'' Quellen identifiziert werden. Dazu wurden zwei Kriterien definiert, welche und/oder verknüpft werden: 
\begin{enumerate}
 \item Der Autor der Quelle erhebt den Anspruch eines Glossars bzw. einer umfassenden Betrachtung des eigenen Werks wie \gls{bspw} SCHMITT \cite{Schmitt2015}. %TODO Alternativ die Enschätzung meinerseits als solches. 
 \item Eine häufige Zitierung der Quelle in anderen Werken, insbesondere im Kontext der Einführung in das Anlaufmanagement. %TODO Daraus wird eine gewisse Reputation abgeleitet. 
\end{enumerate}

Für die weitere Vorgehensweise wurden die Quellen in Tabelle \ref{tab:quellengrundgeruest} identifiziert. 
% 
\begin{table}[h]
\begin{center}
\begin{tabular}{l l l r}
\textbf{Autor} & \textbf{Jahr} & \textbf{Titel} & \textbf{Ref.} \\ \hline
 Kuhn et al.  & 2002 & Fast Ramp-Up - Schneller Produktionsanlauf von Serienprodukten & \cite{Kuhn2002} \\
%  Schuh et a. & 2004 & Fast Ramp-Up. Anlaufstrategien, Deviationsmanagement und Wissensmanagement für den Anlauf & \cite{Schuh2004}  \\
 Bischoff & 2007 & Anlaufmanagement - Schnittstelle zwischen Projekt und Serie & \cite{Bischoff2007} \\
 Schuh et al. & 2008 & Grundlagen des Anlaufmanagements & \cite{Schuh2008} \\
 Schmitt & 2015 & Anlaufmanagement - Begriffe und Definitionen & \cite{Schmitt2015} 
 \end{tabular} 
 \end{center}
\caption{ Auswahl der Quellen für das Grundgerüst} \label{tab:quellengrundgeruest} 
\end{table}
% 
% 
Die in Tabelle \ref{tab:quellengrundgeruest} genannten Quellen wurden mit Hilfe des Tools \gls{atlas} % TODO italic? 
einer qualitativen Auswertung unterzogen. Dazu wurden sog. Codes definiert, welche jeweils einen thematischen Aspekt beschreiben. Für diese Codes wurden in den Quellen dazugehörige Textpassagen identifiziert. In einem weiteren Schritt wurden die Textpassagen zu den Themengebieten (Codes) gegenübergestellt und verglichen. Damit konnten Schnittmengen gefunden und Gruppen gebildet werden. Die Themengebiete wurden in zwei Abstraktionsebenen unterteilt: Konzeptionell und Ausführend.
Mit dem soeben beschriebenen Verfahren wurde die Codestruktur im Verlauf der Analyse angepasst und bildet das Grundgerüst, die Basis für die Abschlussarbeit. 

\subsection*{Kurzbeschreibung des Grundgerüsts}

%       Herangehensweise: 
% 
%       1. Definition 
% 	   (Ursachen, Konsequenzen)
%       2. Bestandteile, Lösungsansatz
%       3. Enabler, Erfolgsfaktoren
%       4. Herausforderungen

\subsubsection*{A - Strategie}
% \textbf{Definition:}
Unter einer Strategie werden in der Wirtschaft die langfristig geplanten Aktivitäten zur Erreichung der Unternehmensziele verstanden \cite[12]{Schuh2008}. % cite Schuh08:12 but need to find a primary source
% 
Eine Anlaufstrategie bezieht sich auf sämtliche Anläufe im Unternehmen und koordiniert die Aktivitäten zur Erreichung der Anlaufziele \cite[4]{Schuh2008}. Innerhalb des Unternehmens ist die Anlaufstrategie der Produktentwicklungs- und Produktionsstrategie untergeordnet und muss die Ziele beider Strategien aufgreifen und integrieren. %TODO cite primary source vonWagenheim1998, sec: Schuh08:12
	
% Horváth, Péter
% Integrationsmanagement für neue Produkte / Péter Horváth ... (Hrsg.)
% STABI Potsdamer Str. 1 A 339422

Wie auch das Anlaufmanagement im Allgemeinen ist die Anlaufstrategie phasen- und funktionsübergreifend \cite{Pfohl2000}. %TODO check citation
Sie sollte in der frühen Phase des Produktentwicklungsprozesses definiert werden \cite{Schuh2004}. 
KUHN et al beschreibt als übergeordnete Ziele die Beherrschung der Qualität und die Reduzierung von Zeit und Kosten \cite[4]{Kuhn2002}. 

% \textbf{Bestandteile:}
SCHUH beschreibt die Gestaltung der Strategie in den vier Dimensionen Management von Flexibilität, Komplexität, Qualität und Kosten \cite[13]{Schuh2008}. BISCHOFF nennt zudem die strategische Projektwahl, mit dem sich das Unternehmen auf strategisch wichtige Projekte Fokussieren und die Anzahl parallel abzuwickelnder Anläufe reduzieren kann \cite[43]{Bischoff2007}. 

\subsubsection*{B - Organisation}
Die Anlauforganisation bildet die zuvor definierte Strategie bzgl. der Serienanläufe in der Unternehmensstruktur ab. Hauptzweck ist, den gestiegenen Anforderungen in Form von zunehmender Dynamik, Abhängigkeiten und Interdisziplinarität mit der Gestaltung einer zweckmäßigen Unternehmensstruktur zu begegnen \cite[55]{Schuh2008}. 

Während die Anlauf-Aufbauorganisation involvierte Bereiche räumlich und formal strukturiert, legt die Anlauf-Ablauforganisation die zeitlichen und logischen Beziehungen zueinander fest \cite[55]{Schuh2008}.
Für die Realisierung der Aufbauorganisation werden interdisziplinäre Stablinien- oder Matrixorganisationen eingesetzt \cite[77]{Bischoff2007}. Dabei wird die Matrixorganisation ggf. durch hochqualifizierte Expertenteams unterstützt \cite[4]{Schmitt2015}. %TODO cite primary source Schuh/Kampker/Franzkoch2005 s. 407
%TODO weitere Formen 4 Grundtypen von Anlauorganisation bzgl AUFBAU ORG s. Grafik \cite[58]{Schuh2008} -> ggf. in Anhang
Empfehlenswert ist auch der Einsatz eines Serienanlaufteams, dessen Funktionsweise und Einbindung in die Aufbauorganisation unterschiedlich ausgeprägt sein können \cite[79]{Bischoff2007}. 

\subsubsection*{C - Planung}

% Definition: 
Die Anlaufplanung umfasst zum einen die Entwicklung eines technischen Konzepts für das Produktionssystem und zum anderen die Planung des organisatorischen Ablaufs \cite{Risse2002}. % TODO check primary source: referred to Risse 2003 but I do have Risse 2002
KUHN et al. stellt fest, dass viele Verzögerungen und Änderungen während der Anlaufphase direkt auf mangelhafte Planung zurück zu führen sind \cite[19]{Kuhn2002}. 
Ziel ist, mit Hilfe von Erfahrungswissen mögliche Probleme und Entscheidungen in die Planungsphase zu vorzuverlegen und somit Zeit und Kosten in der Anlaufphase zu sparen \cite{Risse2002}.  % TODO check primary source: referred to Risse 2003 but I do have Risse 2002

% Bestandteile: 
Die Anlaufplanung bedient sich proaktiver Methoden und Werkzeuge. KUHN et al. nennt \gls{bspw} die Integration von Standards, Quality Gates und Meilensteindefinitionen \cite[19]{Kuhn2002}. 
Diese Methoden, die zusammen ein Reifegradcontrolling ergeben, basieren auf Ermittlung und Kontrolle erreichter Produkt- und Prozessreifegrade. Dazu werden quantifizierbare und messbare Reifegradindikatoren und Zielgrößen definiert, deren Zielerreichung mit Hilfe objektiver Mittel gemessen und bewertet wird \cite[62--63]{Schuh2008}. 

\subsubsection*{D - (Steuerung bzw.) Regelung}

% Die Definition steht gewissermaßen im Widerspruch mit der Auffassung von Schmitt-2015, er fasst die Aktivitäten unter Steuerung zusammen. 
Unter Regelung wird in der Systemtheorie ein System verstanden, das fortwährend derart in das System eingreift, sodass die Differenz zwischen IST und SOLL Werten minimiert wird \cite[136]{diniec60050}. Somit bildet die Regelung die operative Umsetzung zur Einhaltung der in der Planung definierten Zielgrößen in Form von messbaren Kennzahlen. 
Des Weiteren nennt KUHN et al. sog. Controllingmodelle, welche Probleme möglichst früh erkennen und geeignete Reaktionsstrategien auswählen \cite{Kuhn2002}. 

\subsubsection*{1 - Produktentwicklung}
Nach KRISHNAN et al. beschreibt die Produktentwicklung die Transformation einer Marktchance in Verbindung mit Annahmen über eine Produkttechnologie hin zu einem käuflich erwerbbarem Produkt \cite{Krishnan2001}. Für die Praxis bedeutet es die Umsetzung der technischen Kundenwünsche und Vorgaben der Geschäftsführung in realisierbare und effiziente Lösungen \cite[9]{Scholz2010}. 

\subsubsection*{2 - Wissensmanagement}

% Definition: 
% Die Themenkomplexe Wissensmanagement und Personalmanagement sind zusammenhängend zu betrachten. KUHN betont den Einfluss der am Anlauf beteiligten Mitarbeiter auf den Ablauf und Erfolg des Projekts. Die Erfolgsfaktoren setzen sich zum einen aus Mitarbeiterqualifikation und -motivation, und zum anderen aus Wissenserfassung, -visualisierung und -weitergabe zusammen \cite[31]{Kuhn2002}. 
Unter Wissensmanagement werden Tätigkeiten verstanden, die dem organisierten, systematischen und kontrollierten Umgang mit Unternehmenswissen dienen \cite{Disterer2000}. Durch effektive Ausnutzung von im Unternehmen erlangtem Wissen, können Wettbewerbsvorteile erzielt werden \cite{Bischoff2007}. 
DISTERER sieht eine große Herausforderung in der Sicherung von in Projekten erlangtem Wissen, was sich auf Anlaufprojekte übertragen lässt \cite{Disterer2000}. %cited in Bischoff2007
Die Schwierigkeit besteht darin, dass nach Projektabschluss keine festen Ansprechpartner als Wissensträger zur Verfügung stehen. Eine weitere Schwierigkeit besteht darin, implizites in explizites Wissen umzuwandeln. % TODO cite Houssein: Find source. Cited in Bischoff2007

% Bestandteile: 
KUHN et al. schlägt die Entwicklung eines anlaufspezifischen, abteilungs- und unternehmensübergreifenden Wissensmanagement vor, mit Fokus auf eine menschengerechte Bereitstellung der Daten \cite{Kuhn2002}. 
BISCHOFF et al. nennt die Wahrung der Datenkonsistenz als Erfolgsfaktor, insbesondere in mehrstufigen Lieferantennetzwerken \cite{Bischoff2007}. 

\subsubsection*{3 - Qualitätsmanagement}
\begin{quotation}
 Eine auf Qualität ausgerichtete Organisation fördert eine Kultur, die zu Verhaltensweisen, Einstellungen, Tätigkeiten und Prozessen führt, die Wert schaffen, indem sie die Erfordernisse und Erwartungen von Kunden und anderen relevanten interessierten Parteien erfüllen \cite[10]{ISO9000}. % cite ISO9000:2015 p. 10 from Schmitt2015 p.30
\end{quotation}

Nach DIN ISO 9000:2015 ist Qualität definiert als: ``Grad, in dem ein Satz inhärenter Merkmale eines Objekts Anforderungen erfüllt.'' \cite[39]{ISO9000}. % cite 9000:2015 p. 39 from Schmitt2015 p.30
SEGHEZZI et al. unterteilt das Qualitätsmanagement in strategisch-normative und operative Tätigkeiten \cite{Seghezzi2013}. 
% Cited from Schmitt2015 p. 30, Seghezzi2013 pp. 79
Dabei umfassen die strategisch-normativen Tätigkeiten \gls{bspw} die Definition und Umsetzung einer Qualitätsstrategie und -politik. Qualitätsmanagementsysteme verankern und standardisieren die Tätigkeiten nachhaltig im Unternehmen. 
Risikomanagement, Qualitätsplanung und -lenkung werden den operativen Tätigkeiten zugeordnet \cite{Seghezzi2013}. 
%  cite seghezzi from Schmitt2015 p.30
Zur strukturellen Bewertung vorgeschlagener Qualitätsverbesserungen werden z.B. der \textit{kontinuierliche Verbesserungsprozess} (\gls{kvp}) sowie der \textit{``Plan-do-check-act''}(\gls{pdca})-Zyklus verwendet \cite[17]{Schuh2008}.
% 
SCHUH et al. nennt folgende Ansatzpunkte für Anlaufstrategien \cite[17]{Schuh2008}: 
\begin{itemize}
 \item Strukturierte Planung des Qualitätsniveaus und der Prüfparameter anhand von konkreten Kundenanforderungen
 \item Inhaltliche und prozessuale Abstimmung der Qualitätsplanung mit Supply-Chain-Partnern mittels standardisierter Techniken (\gls{qfd}, Reifegrad- und Meilenstein-Controlling)
 \item Frühzeitige Identifizierung potenzieller Schwachstellen (\gls{fmea} Lieferantenbewertung)
 \item Nutzung von Industriestandards (z.B. ISO9001)
 \item Implementierung von \gls{kvp}- und Lieferantenentwicklungsprozessen
\end{itemize}

\subsubsection*{4 - Risikomanagement}
Unter Risikomanagement werden systematische Herangehensweisen zusammengefasst, die der Identifizierung und Bewertung potenzieller Risiken und anschließender Ableitung geeigneter Maßnahmen wie \gls{bspw} Risikoverhütung oder -minderung dienen. %\cite[302]{Burghardt2006}.
Unternehmen sind gesetzlich dazu verpflichtet, ein Überwachungssysten (Risikomanagement) zur Früherkennung gefährdeter Entwicklungen einzusetzen \cite[302]{Burghardt2006}.
WILDEMANN misst dem Risikomanagement eine hohe Bedeutung für den Serienanlauf bei \cite{Wildemann2004}. 

ZÄH et al. beschreibt ein Risikomanagement-Prozess, der sich in vier Phasen unterteilt: Risikoidentifikation, -bewertung, -steuerung und -überwachung \cite{Zaeh2004}. 
Zur strukturellen Analyse von Risiken und Risikofolgen eignen sich  Methoden wie \gls{bspw} \gls{fmea} und \gls{fta}. Sie ermöglichen die Eliminierung potenzieller Fehler, Schwachstellen und Risiken im Vorfeld.
Die Identifizierung potentieller Risiken kann durch den Einsatz von Kreativitätstechniken wie \gls{bspw} Brainstorming oder \gls{ishikawa} (Ursache-Wirkungs-Diagramm) unterstützt werden \cite[41]{Bischoff2007}. 

\subsubsection*{5 - Änderungen}

% Definition: 
Technische Änderungen sind notwendige nachträgliche Anpassungen an bereits freigegebenen Entwicklungsständen \cite{Zanner2002}. Sie beinhalten immer eine Änderung der Dokumentation bzw. Datenbasis \cite[47]{Niemerg1997}. 	

%   Sekundärquelle: \cite[215]{Schuh2008}
%   cite Primärquelle Niemerg1997 - ZB Grimm-Zentrum 
%   Geschlossenes Außenmagazin 03a 
%   98 HA 8754 vorbestellt ins Campus Nord. 
Produktänderungen können in der Entwicklungs- und Konstruktionsphase bis zu 40\% der Gesamtressourcen beanspruchen \cite{Lindemann1998}.
%TODO cite Lindemann1998 -> TUB QP624 77
Änderungsmanagement soll die Termintreue der Prozesse im Serienanlauf sicherstellen und die Durchlaufzeiten reduzieren \cite[216]{Schuh2008}. 

% Ursachen: 
Auslöser für Änderungen können Gesetzesänderungen, interne Fehler, Qualitäts- und Sicherheitsprobleme, veränderte Kundenwünsche sowie eine veränderte Markt- und Wettbewerbssituation sein \cite{Zanner2002}. Auch treten Probleme oft erst dann in Erscheinung, wenn sie im Kontext der benachbarten Komponenten stehen \cite[24]{Kuhn2002}.

% Konsequenzen: 
Änderungen bringen Konsequenzen mit sich. So führen sie zu steigendem Zeitdruck, einem erhöhten Personalaufwand in planerischen Abteilungen sowie können Kosten und Zeitverzögerungen aufgrund von Werkzeugänderungen entstehen \cite[24]{Kuhn2002}. 

% Lösungsansatz und Bestandteile: 
Um den zeitlichen und finanziellen Aufwand gering zu halten, sollten Änderungen vermieden oder möglichst vorverlagert werden \cite{Schuh2008, Jania2004, Ass98}. 
% Schuh2008:215, Jania2004:69f, Ass98:107--131
%TODO Primärquelle Ass98 - TUB QP624 77
SCHUH teilt das Änderungsmanagement in Änderungsplanung, -ausführung und -absicherung ein \cite[217]{Schuh2008}. 
%TODO cite original author Florian Rösch et. al.
LINDEMANN hingegen unterteilt das Thema detaillierter in Vermeidung, Früherkennung, Problemanalyse, Lösungsfindung, Bewertung und Entscheidung. Die Erkenntnisse werden mit Hilfe einer sog. Lernorientierten Auswertung im Sinne eines \gls{kvp} ausgewertet \cite{Lindemann1998}. 
% TODO Lernorientierten Name oder klein geschrieben? 
%TODO cite Primärquelle
%TODO KVP glossary, ausgewertet synonym

% Enabler: 
Als Schlüsselrolle für erfolgreiches Änderungsmanagement wird oft die Kommunikation von Problemen und Änderungen innerhalb und über Unternehmensgrenzen hinweg genannt \cite{Kuhn2002, Schuh2008}.
% Kuhn2002:28+24, Schuh08:219
ZANNER betont die Bedeutung der Vertrauensverhältnisses für den Informationsaustausch und schlägt informelle standortübergreifende Treffen der Entwickler vor. Die Zuordnung eines Verantwortlichen Mitarbeiters für die Abwicklung einer Änderung soll helfen, die Schnittstellenprobleme bei der Arbeitsteiligen Arbeitsweise zu überwinden  \cite[42]{Zanner2002}.
Weiterhin werden eine einheitliche Terminologie \cite{Zanner2002} und Datenbasis sowie ein durchgängiges Versionsmanagement \cite{Kuhn2002} als Erfolgsfaktoren genannt. 
% Kuhn: Datenbasis S.25, Versionsmanagement S. 25
\subsubsection*{6 - Kooperationen}
% Definition: 
Unter dem Stichwort Kooperationen werden Maßnahmen zusammengefasst, die die Unternehmensinterne und -übergreifende Zusammenarbeit verbessern. Gegenstand der Untersuchungen sind meist Informationsflüsse. 
So verspricht KUHN et al. einerseits verbesserte horizontale sowie vertikale Integration der Zulieferer, andererseits eine verbesserte Synchronisation der Anlaufaktivitäten. Dadurch können Fehlerquellen, Änderungsaufwände, Kosten und Zeitbedarfe von Anläufen erheblich reduziert werden \cite[26]{Kuhn2002}. 

% Lösungskonzepte:
Eingesetzt werden ganzheitliche Ansätze wie \gls{bspw} übergreifende Transparenz von Daten und Prozessen sowie eine bedarfsgerechte Gestaltung der Schnittstellen \cite{Kuhn2002}. 

\subsubsection*{7 - Lieferanten}
% Definition: 
Aufgabe des Lieferantenmanagements ist, die Ziele, Werte und Verhaltensnormen bzgl. der Zusammenarbeit mit Lieferanten festzulegen \cite{Schuh2008}. % TODO cite chapter Druml2008, cited in Schmitt2015
Die Bedeutung des Lieferantenmanagements nimmt in Folge sinkender Wertschöpfungstiefe stetig zu. 

% Bestandteile: 
Kernaufgabe ist die Auswahl und Integration der richtigen Lieferanten. Dabei sollte sich ein Unternehmen möglichst früh auf optimale Lieferanten konzentrieren und diese gezielt in den Anlauf integrieren. \cite{Schuh2008}.  %TODO cite Peters et al. from Schmitt2015
Um eine reibungslose Zusammenarbeit zu ermöglichen, schlägt FALZMANN eine regelmäßige Bewertung der Lieferanten vor \cite{Falzmann2007}. Damit möglichst schnell Änderungen durchgeführt werden können, sollten die Ergebnisse der Bewertungen zeitnah an die Lieferanten kommuniziert werden \cite{Hofbauer2012}. % TODO get source and cite. Cited in Schmitt2015

\subsubsection*{8 - Logistik}

Die Logistik beinhaltet die Koordinierung aller Material- und Informationsflüsse und Prozesse von Auftrag bis Auslieferung des Endprodukts. Die strategische Ebene beinhaltet die Entwicklung und Gestaltung der Wertschöpfungsnetzwerde und Prozesse nach logistischen Prinzipien. Die operative Ebene beinhaltet die Lenkung und Kontrolle der Material- und Informationsflüsse und der dazugehörigen Prozesse. 
Hauptziel der Logistik ist, durch Gestaltung und Lenkung der logistischen Prozesse die Kundenbedürfnisse in den ökologischen, ökonomischen und sozialen Dimensionen optimal zu erfüllen \cite[28]{Schmitt2015}. 

Die Bedeutung der Logistik für die Anlaufphase ist durch die Globalisierung der Märkte, Hit-Konzepte und Reduzierung der 
% TODO HIT ?? -> glossary und Zeilenumbruch!! 
Wertschöpfungstiefe gestiegen. Die Logistik hat zwei spezielle Funktionen in der Anlaufphase. Zum einen muss sie den Materialfluss der ersten Produkte bewerkstelligen. Zum anderen erprobt sie bereits Logistikprozesse für die Serie.
Durch den Querschnittscharakter der Logistik ist eine Abstimmung mit anderen Funktionsbereichen und der Logistik anderer Unternehmen erforderlich \cite[1189]{Pfohl2000}.

\subsubsection*{9 - Produktion}
% Definition: 
Unter Produktionsmanagement werden alle Tätigkeiten verstanden, die die physische Herstellung der Produkte ermöglichen. Dazu gehören nach KUHN et al. die Subsysteme Fertigung, Montage und Logistik \cite{Kuhn2002}. Das Subsystem Logistik wird im Rahmen dieser Arbeit jedoch einzeln betrachtet. Weiterhin führt KUHN et al. den Begriff \textit{Anlaufrobuste Produktionssysteme} ein. \textit{Anlaufrobuste Produktionssysteme} zeichnen sich dadurch aus, dass sie agil auf späte Änderungen und Stückzahlschwankungen reagieren \cite[20]{Bischoff2007}. 

% Bestandteile: 
SCHUH et al. unterteilt das Produktionsmanagement in drei Teilaspekte: Werksstruktur- und Betriebsmittelplanung, Standardisierung in der Produktion und Mitarbeiterqualifikation und {-befähigung} \cite[177]{Schuh2008}. %TODO cite chapter 
BISCHOFF et al. nennt die Ermittlung von Reifegraden für Prozesse als zentralen Bestandteil des Produktionsmanagements \cite[20]{Bischoff2007}. 

Die Tatsache, dass im Serienanlauf noch keine Serienbedingungen herrschen stellt eine enorme Herausforderung dar. \Gls{bspw} stammen die Zuliefererteile aus Vorserienwerkzeugen oder entsprechen einem veralteten Entwicklungsstand \cite[21]{Kuhn2002}. 