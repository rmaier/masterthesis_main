\chapter{Methodik}\label{sec:methodik}
In diesem Abschnitt wird die Methodik der Abschlussarbeit dargelegt. Zunächst erfolgt eine Beschreibung der allgemeinen Herangehensweise beginnend mit der Literaturrecherche. Anschließend wird der rote Faden der Arbeit entwickelt, der aus zwei Teilen besteht. Zuerst werden spezifische Anforderungen an das Anlaufmanagementmodell für das \gls{lsu} definiert. Im zweiten Schritt erfolgt die Entwicklung des Grundgerüsts, welches die weitere Bearbeitung mittels in Beziehung stehender Unteraspekte strukturiert. Während die Anforderungen als Erfolgsfaktoren zu sehen sind, gibt das Grundgerüst die inhaltliche Struktur vor.  

\section{Inhaltlicher Aufbau der Arbeit}
Das Kapitel \ref{sec:einfuehrung} legt die Motivation und Zielsetzung der Arbeit dar. Weiterhin erfolgt zum Verständnis der Aufgabenstellung eine erste Definition zum Thema Anlaufmanagement sowie ein Überblick zu Lean Start-up. 

Kapitel \ref{sec:methodik} beschreibt zunächst den inhaltlichen Aufbau der Arbeit. Die methodische Herangehensweise wird unter Berücksichtigung von Suchstrategie und Forschungsmethodik dargelegt. In einem weiteren Schritt werden spezifische Anforderungen des Lean Start-up an das Anlaufmanagementmodell beschrieben, welche bei der Ausarbeitung berücksichtigt werden sollen. Schließlich wird ein Grundgerüst für das zu entwickelnde Anlaufmanagementmodell entwickelt, welches als struktureller und inhaltlicher roter Faden der Arbeit dient. %TODO Rechtschreibung Roter Faden? Siehe auch weiter unten
Die Entwicklung des Grundgerüsts dient auch der Konsolidierung der verschiedenen Auffassungen des Themengebiets. 

In Kapitel \ref{sec:durchfuehrung} werden mittels qualitativer Literaturanalyse und anhand des zuvor entwickelten Grundgerüsts Lösungskonzepte zusammengetragen. 

Diese werden anschließend in Kapitel \ref{sec:ableitung} zueinander in Beziehung gesetzt und sollen als Gesamtbild in das Lean Start-up integriert werden. 

Kapitel \ref{sec:diskussion}...

Die Arbeit mündet in Kapitel \ref{sec:fazit} mit einem Fazit und Ausblick. Es werden die Grenzen der angewandten Methodik sowie des entwickelten Modells aufgezeigt. Schließlich werden weitere Entwicklungsschritte vorgeschlagen und der Forschungsbedarf identifiziert. 

Weiterhin wird weiterer Forschungsbedarf identifiziert. 

\section{Methodische Herangehensweise der Arbeit}

Zu Beginn der Arbeit erfolgte eine erste Recherche zu den beiden Themenfeldern Anlaufmanagement und Lean Start-up. Die verwendeten Suchbegriffe sind auf Tabelle \ref{tab:algorythm} aufgeführt. Dabei wurden in folgender Rangfolge Suchmaschinen benutzt: 
1. google.com, 2. scholar.google.com, 3. sciencedirect.com, 4. rd.springer.com. Die erste Recherche zu \gls{lsu} genügte den Anforderungen. Die Recherche zu Anlaufmanagement genügte lediglich für die Bildung eines Grundgerüsts, dessen Herangehensweise ausführlich in Abschnitt \ref{sec:grundgeruest} beschrieben wird. Darauf aufbauend erfolgte eine Vertiefung der Literaturrecherche mittels Schneeballsystem. Dazu wurden Literaturlisten der Quellen, die Namen der Verfasser und \gls{bspw} das Graduiertenkolleg Anlaufmanagement (1491-2) der RWTH-Aachen zu Grunde gelegt. 
\begin{table}[h]
\begin{center}
\begin{tabular}{l l}
\textbf{Themenfeld} & \textbf{Algorithmus }\\ \hline
Anlaufmanagement & ('Ramp-up' OR 'Manufacturing' AND 'Ramp-up' OR 'Production' \\ 
& AND 'Ramp-up' OR 'Anlaufmanagement' OR 'Produktion' \\
& AND 'KMU' OR 'Manufacturing' AND 'SME') \\
Lean Start-up & ('Lean' AND 'Startup' OR 'Lean' AND 'Start-up')
 \end{tabular} 
 \end{center}
\caption{Suchalgorithmen für die Literaturrecherche} \label{tab:algorythm} 
\end{table}

In Kapitel \ref{sec:durchfuehrung} erfolgt die Erfassung von Lösungskonzepten mittels qualitativer Literaturanalyse. Die Auswahl der Quellen erfolgt nach folgenden Kriterien: Beiträge aus Fachzeitschriften oder Konferenzen, bei denen die Qualitätssicherung mittels Peer-Review erfolgt. Weiterhin erlaubt sind einzelne Kapitel aus wissenschaftlichen Monographien. Die Anzahl der auszuwertenden Quellen wird auf 20 begrenzt. Für die qualitative Literaturanalyse bildet das Grundgerüst aus Abschnitt \ref{sec:grundgeruest} die Struktur. Zur Präzisierung der Perspektive und Verringerung der Subjektivität werden in folgendem Abschnitt \ref{sec:anforderungen} Anforderungen an das Modell beschrieben. 



\section{Anforderungen an das Anlaufmanagementmodell für das \gls{lsu}}\label{sec:anforderungen}
Damit das zu entwickelnde Modell den Serienanlauf im \gls{lsu} effektiv unterstützt, müssen zunächst einige Anforderungen formuliert werden. 

\subsection*{Methodische Anforderungen}

Das zu entwickelnde Anlaufmanagementmodell muss sowohl horizontal als auch vertikal sinnvoll mit dem \gls{lsu} Ansatz kooperieren bzw. integriert werden. 
Die Beseitigung von Verschwendung sowie Anreize zur kontinuierlichen Verbesserung müssen strukturell im Modell verankert sein. 

Besondere Eigenschaften von (Lean-)Start-ups müssen berücksichtigt werden. Dazu zählen \gls{bspw} eine flache Hierarchie, eine kleine Anzahl an Mitarbeitern, Vorhandensein von Generalisten anstatt Spezialisten und Interdisziplinarität der Mitarbeiter und Aufgaben. Daraus werden folgende Forderungen abgeleitet: Eine kleine Anzahl an einfach anzuwendenden Methoden. Die Gestaltung von Ablauforganisation und Prozessen erfolgt mit geringem Detaillierungsgrad. An anderer Stelle soll jedoch mithilfe von Standardisierung die Komplexität der Lösungsalternativen beschränkt werden. Daraus folgt eine hohe Abstraktionsebene des Modells einerseits, andererseits jedoch ein hoher Detaillierungsgrad. 

Des weiteren wird eine Flexibilität des Modells gefordert. So muss das Modell, welches bereits in der Anfangsphase implementiert wird, bei schnellem Wachstum und stark veränderten Bedingungen weiterhin effektiv sein. Dazu zählen \gls{bspw} eine Skalierbarkeit der Methoden hinsichtlich Anzahl der Mitarbeiter sowie Mitarbeiterzuordnung von Kompetenzen und Aufgaben. 

\subsection*{Technische Anforderungen}

Auch auf technischer Seite ist Flexibilität gefordert. Die Produktion bzw. der Anlauf müssen agil auf Stückzahlschwankungen reagieren können. Änderungen am Produkt oder die Einführung neuer Varianten müssen einfach und schnell mit hoher Qualität realisiert werden können. Analog dazu müssen Änderungen am Logistiksystem und Produktionslinie effizient durchgeführt werden können. 
Große Unsicherheiten sind ein inhärentes Merkmal des Serienanlaufs. Daher muss ein umfassendes Risikomanagement im Modell verankert sein. 

\section{Entwicklung des Grundgerüsts}\label{sec:grundgeruest}
\subsection*{Zielstellung}

Der Stand der Wissenschaft zum Thema Anlaufmanagement soll recherchiert und dargestellt werden. Dies erfolgt zunächst nur übergeordnet, indem ca. 10-20 Hauptaspekte identifiziert und priorisiert werden. Diese Hauptaspekte werden geordnet und in einem Grundgerüst abgebildet. Das Grundgerüst dient nun als struktureller und inhaltlicher roter Faden der Arbeit. 
Zunächst bildet er die Systematik für die Literaturrecherche. Dazu werden aus dem Grundgerüst Themengebiete und Stichworte für die Suche abgeleitet. 
Auch die Einordnung der Lösungskonzepte und Methoden erfolgt nach dem Grundgerüst. %TODO ref to chapter? 
Schließlich bildet es die Grundlage für das Ergebnis der Arbeit, das Anlaufmanagementmodell für das \gls{lsu}. % TODO ref to chapter? 

\subsection*{Herangehensweise}

Nach kurzer Recherche ist festzustellen, dass bisher keine einheitliche Auffassung des Anlaufmanagements existiert. Vielmehr wurde das Themengebiet von einigen Autoren bisher nur aus individueller Perspektive behandelt. 
\Gls{bspw} hat SCHMITT im Jahre 2015 ein Glossar veröffentlicht, mit dem Ziel, ein einheitliches Verständnis sowie die Grundlage für die wissenschaftliche und praxisnahe Diskussion des Themengebiets zu schaffen \cite{Schmitt2015}. Diese Arbeit bestärkt die Einschätzung des Autors. %TODO Formulierung?? 

In der Konsequenz ist zunächst eine Konsolidierung anhand einschlägiger Literatur (Primär- und Sekundärliteratur) nötig, welche einen möglichst umfassenden Blickwinkel des Themengebiets behandelt. Dazu erfolgt im ersten Schritt eine Erstrecherche zum Thema Anlaufmanagement (engl. manufacturing ramp-up). Aus dem Ergebnis der Erstrecherche müssen nun die geforderten ``einschlägigen'' Quellen identifiziert werden. Dazu wurden zwei Kriterien definiert, welche und/oder verknüpft werden: 
\begin{enumerate}
 \item Der Autor der Quelle erhebt den Anspruch eines Glossars bzw. einer umfassenden Betrachtung des eigenen Werks wie \gls{bspw} SCHMITT \cite{Schmitt2015}. %TODO Alternativ die Enschätzung meinerseits als solches. 
 \item Eine häufige Zitierung der Quelle in anderen Werken, insbesondere im Kontext der Einführung in das Anlaufmanagement. %TODO Daraus wird eine gewisse Reputation abgeleitet. 
\end{enumerate}

Für die weitere Vorgehensweise wurden die Quellen in Tabelle \ref{tab:quellengrundgeruest} identifiziert. 
% 
\begin{table}[h]
\begin{center}
\begin{tabular}{l l l r}
\textbf{Autor} & \textbf{Jahr} & \textbf{Titel} & \textbf{Ref.} \\ \hline
 Kuhn et al.  & 2002 & Fast Ramp-Up - Schneller Produktionsanlauf von Serienprodukten & \cite{Kuhn2002} \\
%  Schuh et a. & 2004 & Fast Ramp-Up. Anlaufstrategien, Deviationsmanagement und Wissensmanagement für den Anlauf & \cite{Schuh2004}  \\
 Bischoff & 2007 & Anlaufmanagement - Schnittstelle zwischen Projekt und Serie & \cite{Bischoff2007} \\
 Schuh et al. & 2008 & Grundlagen des Anlaufmanagements & \cite{Schuh2008} \\
 Schmitt & 2015 & Anlaufmanagement - Begriffe und Definitionen & \cite{Schmitt2015} 
 \end{tabular} 
 \end{center}
\caption{ Auswahl der Quellen für das Grundgerüst} \label{tab:quellengrundgeruest} 
\end{table}
% 
% 
Die in Tabelle \ref{tab:quellengrundgeruest} genannten Quellen wurden mit Hilfe des Tools \gls{atlas} % TODO italic? 
einer qualitativen Auswertung unterzogen. Dazu wurden sog. Codes definiert, welche jeweils einen thematischen Aspekt beschreiben. Für diese Codes wurden in den Quellen dazugehörige Textpassagen identifiziert. In einem weiteren Schritt wurden die Textpassagen zu den Themengebieten (Codes) gegenübergestellt und verglichen. Damit konnten Schnittmengen gefunden und Gruppen gebildet werden. Die Themengebiete wurden in zwei Abstraktionsebenen unterteilt: Konzeptionell und Ausführend.
Mit dem soeben beschriebenen Verfahren wurde die Codestruktur im Verlauf der Analyse angepasst und bildet das Grundgerüst, die Basis für die Abschlussarbeit. 

\subsection*{Kurzbeschreibung des Grundgerüsts}



