% \section{Methodik}
\chapter{Methodik}

\section{Lean Start-up}
\subsection*{Einführung}
Das Lean Start-up ist eine Businessmethode für dynamische Unternehmen oder Projekte, die hohen Risiken und Unsicherheiten ausgesetzt sind. 
Hauptziele der Methode sind kürzere Entwicklungszeiten, Einsparung von Kosten in der Entwicklungsphase und frühzeitiges Erkennen der Kundenbedürfnisse. 
Sie ist eine Antwort auf hoch dynamische Märkte, unbekannte Problemstellungen und Lösungen und hohen Risiken. Die Ursprünge liegen in den Denkweisen von Taiichi Ōno, W. Edwards Deming und Peter Drucker. 
2008 übertrug Eric Ries Lean Produktions Methoden auf hochtechnologie Startups und veröffentlichte 2011 die erstmals "Lean Startup" genannte Methode in seinem Buch. 

\subsection*{Definitionen}

\subsection*{Elemente}

\textit{1. Entwickeln einer Vision}. Die Vision dient als Grundlage für alle weiteren Handlungen. Aus ihr werden im nächsten Schritt Hypothesen abgeleitet. Anstatt einen aufwändigen Businessplan zu erstellen wird die Vision in einem Business Model Canvas definiert \cite{Blank2013}. Die Vision eines Lean Start-up zeichnet sich durch viele Freiheitsgrade und Unsicherheiten aus. 

\textit{2. Überführen der Vision hin zu Hypothesen}. Für jedes Element der im Business Model Canvas beschriebenen Vision werden Hypothesen abgeleitet. Die Hypothesen bilden die Freiheitsgrade und Unsicherheiten des BMC ab. Ziel ist, die Risiken durch spätere Beantwortung der Hypothesen zu minimieren. Nach Möglichkeit sollen die Hypothesn so formuliert werden, dass sie quantitativ beurteilt werden können. Die Hypothesen müssen wiederlegbar sein, um neue Erkenntnisse gewinnen zu können. 

\textit{3. Entwickeln von MVP Tests}. Ein minimal überlebensfähiges Produkt (MVP, engl.: Minimum Viable Product) ist ein Werkzeug, mit dem man schnellstmöglich die Hypothesen am Kunden überprüfen kann \cite{Ries2013}. Ziel ist zum einen die den Build-Measure-Learn Zyklus zu beschleunigen, zum anderen die Lernrate in Bezug auf den Aufwand zu maximieren. So können frühzeitig nicht benötigte Funktionen und Produkteigenschaften erkannt und Zeit und Kosten gespart werden. Wenn die Entwicklung eines realen MVP zu aufwändig ist, kann ein Smoke Test eingesetzt werden. In einem Smoke Test wird das zukünftige Produkt in einem Video oder über eine Webseite vorgestellt.  

\textit{4. Planung der Tests}

\textit{5. Interpretation der Ergebnisse}

\textit{6. Reaktion}

\textit{7. Skalierung und kontinuierliche Verbesserung}

\subsection*{Grenzen der Methodik}

\section{Anlaufmanagement}


