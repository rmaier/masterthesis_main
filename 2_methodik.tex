% \section{Methodik}
\chapter{Methodik}

\section{Lean Start-up}

Das Lean Start-up ist eine Businessmethode für dynamische Unternehmen oder Projekte, die hohen Risiken und Unsicherheiten ausgesetzt sind. Hauptziele der Methode sind kürzere Entwicklungszeiten, Einsparung von Kosten in der Entwicklungsphase und frühzeitiges Erkennen der Kundenbedürfnisse. Sie ist eine Antwort auf hoch dynamische Märkte, unbekannte Problemstellungen und Lösungen und hohen Risiken. Die Ursprünge liegen in den Denkweisen von Taiichi Ōno, W. Edwards Deming und Peter Drucker. 2008 übertrug Eric Ries Lean Produktions Methoden auf hochtechnologie Startups und veröffentlichte 2011 die erstmals "Lean Startup" genannte Methode in seinem Buch. 



\section{Anlaufmanagement}


