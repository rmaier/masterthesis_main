\chapter*{Kurzzusammenfassung}
% %  Ziele / Problemstellung
% Steigende Innovationszyklen, kürzere Produktlebenszyklen und eine höhere Variantenvielfalt bilden heute die größten Herausforderungen für die produzierende Industrie. Somit bekommen Serienanläufe einen wachsenden Einfluss auf den wirtschaftlichen Erfolg des Produkts. 
% Durch die kontinuierlich sinkende Wertschöpfungstiefe setzt sich der Gesamtanlauf aus vielen Einzelanläufen zusammen. Dadurch erhöht sich die Komplexität des Gesamtanlaufs. 
Serienanläufe stellen aufgrund ihrer zunehmenden Komplexität immer größere Herausforderungen an produzierende Unternehmen. Dies gilt insbesondere für Lean-Startups (\gls{lsu}), die oft nicht über ein systematisches Anlaufmanagement verfügen.
Gegenstand dieser Arbeit ist die Entwicklung eines Leitfadens zur optimalen Umsetzung einer Best-Practice für den Serienanlauf im \gls{lsu}.

% % Lösung: 
% GG
Da bisher in der Literatur keine einheitliche Auffassung über Anlaufmanagement im allgemeinen herrscht, wurden zunächst fünf wichtige Quellen identifiziert. Mit Hilfe des Tools \gls{atlas} wurden die Themenkomplexe strukturiert und konsolidiert in einem Grundgerüst abgebildet. Dieses setzt sich aus strategisch konzeptionellen und operativen Aspekten zusammen und bildet die Struktur für die weitere Arbeit. 
%  Literaturrecherche
Anhand des Grundgerüsts erfolgte eine Literaturrecherche, die den Stand der Wissenschaft bzgl. der Themenkomplexe abbildet. Es wurden geeignete Methoden und Gestaltungsempfehlungen identifiziert und in den Kontext eingeordnet. 
% Leitfaden
Schließlich wurde ein Leitfaden nach Vorbild des Business Model Canvas erstellt. Dieses soll einem Unternehmer anhand von spezifisch zu beantwortenden Fragen helfen, zielgerichtet fundierte Entscheidungen hinsichtlich der Gestaltung eines Serienanlaufs zu treffen. 
% 

% % Impact
% Wissenschaft
Das hier entwickelte Grundgerüst sowie der Leitfaden können als Grundlage für weitere Forschung dienen. Diese kann sowohl hinsichtlich Quantität (\gls{bspw} weitere Literaturrecherche) als auch hinsichtlich Qualität (Erweiterung des Grundgerüsts) variiert werden. Denkbar ist auch eine Änderung der Betrachtungsebene. Für die wissenschaftliche Validierung der Erkenntnisse bietet sich empirische Forschung oder Ac­tion-Re­search an. 
Weiterhin ist zu beachten, dass der Leitfaden lediglich zu gestaltende Aspekte abbildet. Im Hauptteil sind jedoch viele allgemeine Handlungsempfehlungen identifiziert worden, welche im Leitfaden nicht darstellbar sind. Diese sind jedoch von großem Interesse uns sollten \gls{bspw} in einem \textit{Handbuch Anlaufmanagement für das \gls{lsu}} zusammengefasst werden.
% 
% Wirtschaft
In der Praxis ermöglicht der Leitfaden kaufmännisch geprägten Gründern fundierte Entscheidungen zu treffen. Voraussetzungen dafür sind eine hohe Motivation sowie vorhandene Ressourcen. Ebenfalls ist ein Verständnis der Grundbegriffe aus Produktion und Qualitätsmanagement erforderlich. 
%  2061 characters