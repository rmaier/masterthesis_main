\chapter*{Zusammenfassung}
% %  Ziele / Problemstellung
% Steigende Innovationszyklen, kürzere Produktlebenszyklen und eine höhere Variantenvielfalt bilden heute die größten Herausforderungen für die produzierende Industrie. Somit bekommen Serienanläufe einen wachsenden Einfluss auf den wirtschaftlichen Erfolg des Produkts. 
% Durch die kontinuierlich sinkende Wertschöpfungstiefe setzt sich der Gesamtanlauf aus vielen Einzelanläufen zusammen. Dadurch erhöht sich die Komplexität des Gesamtanlaufs. 
Serienanläufe stellen aufgrund ihrer zunehmenden Komplexität immer größere Herausforderungen an produzierende Unternehmen. Dies gilt insbesondere für Lean Start-ups (\gls{lsu}), die oft nicht über ein systematisches Anlaufmanagement verfügen.
Gegenstand dieser Arbeit ist die Entwicklung eines Leitfadens zur optimalen Umsetzung einer Best-Practice für den Serienanlauf im \gls{lsu}.

% % Lösung: 
% GG
Da bisher in der Literatur keine einheitliche Auffassung über Anlaufmanagement im Allgemeinen herrscht, wurden zunächst fünf wichtige Quellen identifiziert. Mit Hilfe des Tools \textit{\gls{atlas}} wurden die Themenkomplexe strukturiert und konsolidiert in einem Grundgerüst abgebildet. Dieses setzt sich aus strategisch konzeptionellen und operativen Aspekten zusammen und bildet die Struktur für die weitere Arbeit. 
%  Literaturrecherche
Anhand des Grundgerüsts erfolgte eine Literaturrecherche, die den Stand der Wissenschaft bzgl. der Themenkomplexe abbildet. Es wurden geeignete Methoden und Gestaltungsempfehlungen identifiziert und in den Kontext eingeordnet. 
% Leitfaden
Schließlich wurde ein Leitfaden nach Vorbild des Business Model Canvas erstellt. Dieses soll einem Unternehmer anhand von spezifisch zu beantwortenden Fragen helfen, zielgerichtet fundierte Entscheidungen hinsichtlich der Gestaltung eines Serienanlaufs zu treffen. 
% 

% % Impact
% Wissenschaft
Das hier entwickelte Grundgerüst sowie der Leitfaden können als Grundlage für weitere Forschung dienen. Diese kann sowohl hinsichtlich Quantität (\gls{bspw} weitere Literaturrecherche) als auch hinsichtlich Qualität (Erweiterung des Grundgerüsts) variiert werden. Denkbar ist auch eine Änderung der Betrachtungsebene. Für die wissenschaftliche Validierung der Erkenntnisse bietet sich empirische Forschung oder Ac­tion-Re­search an. 
% Bemerkung zu allg. Handlungsempfehlungen. Ggf. zu viel für das Abstract??
Weiterhin ist zu beachten, dass der Leitfaden lediglich zu gestaltende Aspekte abbildet. Im Hauptteil sind jedoch viele allgemeine Handlungsempfehlungen identifiziert worden, welche im Leitfaden nicht darstellbar sind. Diese sind jedoch von großem Interesse uns sollten \gls{bspw} in einem \textit{Handbuch Anlaufmanagement für das \gls{lsu}} zusammengefasst werden.
% 
% Wirtschaft
In der Praxis ermöglicht der Leitfaden kaufmännisch geprägten Gründern fundierte Entscheidungen zu treffen. Voraussetzungen dafür sind eine hohe Motivation sowie vorhandene Ressourcen. Ebenfalls ist ein Verständnis der Grundbegriffe aus Produktion und Qualitätsmanagement erforderlich. 
%  2061 characters

\chapter*{Abstract}
Manufacturing ramp-ups challenge the manufacturing industry due to growing complexity. This affects especially the Lean Startup (\gls{lsu}), since it doesn't have a systematically set up ramp-up management. The purpose of this Thesis is to develop a guideline which supports the setup of a proper ramp-up management in \gls{lsu}s. 

Since there is no uniform interpretation of ramp-up management in literature, five relevant sources were identified. Supported with the tool \textit{\gls{atlas}}, these five interpretations were structured and unified into a basic framework. It consists of strategic-conceptual and operational elements and provides the structure for the further work of the thesis. The literature research was based on this framework, and represents the current state of science and technology. Suitable methods and design recommendations were identified and put in context. Finally, the results were formulated into a guideline, which follows the design of the \textit{Business Model Canvas}. This guideline contains specific questions, which lead to profound decisions regarding the design of the ramp-up, when answered by the entrepreneur. 

Both the developed framework and the guideline may also provide a frame for further research. % As well...as
The research may be continued in quantitative (e.g. further research depth) as well as qualitative (expanding the framework) aspects. Additionally, the level of abstraction may be changed. Both empirical research and action research are applicable for validation of the results proposed in this Thesis. 
On the other hand, the guideline allows commercial oriented entrepreneurs to take profound decisions in their industial practice. A high level of motivation and available ressources are requirements that need to be met. In addition, the entrepreneur needs to have a basic knowledge of quality management and manufacturing terms. 

% 1870 characters