\chapter*{Kurzzusammenfassung}

Die Ernährung der Weltbevölkerung ist eine der zentralen Herausforderungen der Zukunft. Der Bevölkerungszuwachs und damit auch der Bedarf an Nahrungsmitteln wird bis zum Jahre 2030 voraussichtlich 40 Prozent betragen. 

Um diesen Problemen entgegenzuwirken, suchen Wissenschaftler nach Wegen, die vorhandenen Ressourcen optimal auszunutzen. Insbesondere in Südeuropa sorgt die großzügige Bewässerung für ein Sinken des Grundwasserspiegels. %, den Flächenertrag von Landwirtschaftlichen Böden zu maximieren. Weiteres Ziel, ist eine optimale Ausnutzung der Ressourcen. Um den Wasser- und Düngemittelbedarf abzuschätzen, müssen entsprechende Bodenparameter permanent gemessen werden. 
Aktuelle Erdfeuchtemessgeräte eignen sich nicht für die automatisierte Messdatenerfassung. 


Mit Hilfe eines engmaschigen Sensorsystems, welches permanent Bodenparameter erfasst und aufzeichnet, kann der Ressourcenbedarf ermittelt werden. Dabei stellt die Energieversorgung der einzelnen Bodensensoren eine konstruktive Herausforderung dar.  
Ziel der Arbeit ist, ein Konzept für eine unabhängige Energieversorgung zu entwickeln. Dabei werden Temperaturdifferenzen im Erdboden mittels Peltier-Elementen in elektrischen Strom umgewandelt. 

Um geeignete Bodentiefen für die Wärmeentnahme zu ermitteln, wurde eine Temperaturmessung im Erdboden durchgeführt. Es wurden Peltier-Elemente ausgewählt und ein Konzept für ihre thermische Anbindung entwickelt. 

Ergebnis ist eine Konstruktion, die Temperaturgradienten von drei verschiedenen Bodentiefen nutzt. Dabei werden zwei Peltier-Elemente eingesetzt. Die berechneten thermischen Verluste der Temperaturgradienten betragen 4\%. Die Höhendifferenz wird mit Heatpipes überwunden. Für die Unterbringung und Verkabelung von el. Leiterplatten ist bereits Platz vorgesehen. Die Herstellungskosten pro Prototyp betragen \mbox{903\euro} bei Produktion von vier Einheiten. Ein Ausbau zur Wartung erfolgt alle drei Jahre. 
Aufgrund fehlender Kennwerte, ist der absolute Energieertrag nicht ermittelt worden. Darüber kann eine Messung am Prototypen im ersten Bodenversuch Aufschluss geben. Anknüpfend an diese Arbeit, soll die Sensorelektronik entwickelt werden. 