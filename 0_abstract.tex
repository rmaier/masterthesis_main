\chapter*{Zusammenfassung}
Serienanläufe stellen aufgrund ihrer zunehmenden Komplexität immer größere Herausforderungen an produzierende Unternehmen. Dies gilt insbesondere für Lean Start-ups (\gls{lsu}), die oft nicht über ein systematisches Anlaufmanagement verfügen.
Gegenstand dieser Arbeit ist die Entwicklung eines Leitfadens zur optimalen Umsetzung einer Best-Practice für den Serienanlauf im \gls{lsu}.

% % Lösung: 
% GG
% Da bisher in der Literatur keine einheitliche Auffassung über Anlaufmanagement im Allgemeinen herrscht, wurden zunächst fünf wichtige Quellen identifiziert. 
Da sich bisher in der einschlägigen Literatur keine gängige Definition des Anlaufmanagements durchgesetzt hat, wurden fünf Quellen identifiziert. 
% Im weiteren Verlauf der Arbeit stellen wir dar, welche Kriterien der Auswahl dieser Quellen zugrunde liegt. 
% 
Mit Hilfe des Tools \textit{\gls{atlas}} wurden die Themenkomplexe strukturiert und konsolidiert in einem Grundgerüst abgebildet. Dieses setzt sich aus strategisch konzeptionellen und operativen Aspekten zusammen und bildet die Struktur für die weitere Arbeit. 
%  Literaturrecherche
Anhand des Grundgerüsts erfolgte eine Literaturrecherche, die den Stand der Wissenschaft bzgl. der Themenkomplexe wiedergibt. Geeignete Methoden und Gestaltungsempfehlungen wurden herausgearbeitet und in den Kontext eingeordnet. 
% Leitfaden
Schließlich wurde ein Leitfaden nach Vorbild des Business Model Canvas erstellt. Dieses soll einem Unternehmer anhand von spezifisch zu beantwortenden Fragen helfen, zielgerichtet fundierte Entscheidungen hinsichtlich der Gestaltung eines Serienanlaufs zu treffen. 
% 

% % Impact
% Wissenschaft
Das hier entwickelte Grundgerüst sowie der Leitfaden können als Grundlage für weitere Forschung dienen. Diese kann sowohl im Hinblick auf Quantität (\gls{bspw} weitere Literaturrecherche) als auch auf Qualität (\gls{bspw} Erweiterung des Grundgerüsts) variiert werden. Denkbar ist auch eine Änderung der Betrachtungsebene. Für die wissenschaftliche Validierung der Erkenntnisse bietet sich empirische Forschung oder Action-Research an. 
% Bemerkung zu allg. Handlungsempfehlungen. Ggf. zu viel für das Abstract??
Weiterhin ist zu beachten, dass der Leitfaden lediglich zu gestaltende Aspekte abbildet. Im Hauptteil sind jedoch einige allgemeine Handlungsempfehlungen identifiziert worden, welche im Leitfaden nicht darstellbar sind. Diese Ergebnisse erscheinen fruchtbar, sodass sie in einem \textit{Handbuch Anlaufmanagement für das \gls{lsu}} zusammengefasst werden könnten.
% 
% Wirtschaft
In der Praxis ermöglicht der Leitfaden kaufmännisch geprägten Gründern wissenschaftlich fundierte Entscheidungen zu treffen. Voraussetzungen dafür sind eine hohe Motivation sowie vorhandene Ressourcen. Ebenfalls ist ein Verständnis der Grundbegriffe aus Produktion und Qualitätsmanagement erforderlich. 
%  2061 characters

\chapter*{Abstract}
Manufacturing ramp-ups challenge the manufacturing industry due to growing complexity. This affects especially the Lean Startup (\gls{lsu}), since it doesn't dispose of a systematically set up ramp-up management yet. The purpose of this thesis is to develop a guideline which supports the setup of a ramp-up management in \gls{lsu}s. 

Since no uniform interpretation of ramp-up management could be found in the literature of relevant research fields, five principal sources have been identified.
% Their choice out of a pertinence and relevance of criteria is further outlined in the text.
Supported with the tool \textit{\gls{atlas}}, these five interpretations were structured and unified into a basic framework. It consists of strategic-conceptual and operational elements and provides the structure for the further work of the thesis. The literature research was based on this framework and is meant to represent the current state of research in science and technology. Suitable methods and design recommendations were identified and put into the broader context. Finally, the results were formulated into a guideline, which follows the design of the \textit{Business Model Canvas}. This guideline contains specific questions, which in their resolution by the manager may lead to well-grounded decisions regarding the design of the ramp-up.

Both the developed framework and the guideline may also provide a framework for further research. % As well...as
The latter may be continued in quantitative (e.g. by adding further theoretical depth) as well as qualitative (e.g. by expanding the framework) aspects. In addition to that, the level of abstraction may be changed. Both empirical research and action research are applicable for the validation of results proposed in this thesis. 
Finally, the guideline allows commercially oriented managers and business leaders to take well-grounded decisions in their industrial practice. For this purpose, a high level of motivation and available resources are requirements that need to be met. Moreover, the entrepreneur needs to have a basic knowledge of quality management and manufacturing terms. 

% 1870 characters