\chapter{Fazit \& Ausblick}\label{sec:fazit}

\section{Kritische Würdigung}

\section{Ausblick}

In diesem Abschnitt sollen die Konsequenzen bzw. Implikationen für die Wissenschaft und Wirtschaft sowie ein Ausblick gegeben werden. 

\subsection*{Implikationen für die Wissenschaft}
Bereits in der Einführung wurde darauf hingewiesen, dass eine Validierung der Ergebnisse im Rahmen dieser Arbeit nicht stattfinden wird. Daher sind die hier gewonnenen Erkenntnisse (AM-Modell, Umsetzungsleitfaden) zur Zeit als ein Vorschlag für eine Best Practice zu betrachten. Eine Validierung kann durch eine zweite Person % TODO Kann ich das so schreiben -> Robert
oder aber durch empirische Bestätigung in der Industrie erfolgen. 

Für die weitere Forschung wurden vier verschiedene Ansätze identifiziert: 

\begin{enumerate}
 \item Die zu untersuchenden Aspekte des Grundgerüsts können erweitert werden. Zunächst können die in dieser Arbeit vorgeschlagenen aber nicht untersuchten Aspekte Kooperationen, Lieferanten und Logistik %TODO check
 hinzugezogen werden. Möglich ist auch, dass neue Aspekte identifiziert oder hier erarbeitete Aspekte weg gelassen werden. 
\item Die Forschung kann quantitativ durch eine umfangreichere Literaturrecherche und -auswertung ergänzt werden. Dabei wird die Anzahl relevanter Quellen und Lösungsvorschläge erhöht. 
\item Die Forschung kann in der Abstraktionsebene variiert werden. Es können detailliertere Handlungsempfehlungen entwickelt werden. Denkbar ist auch die Erstellung konkreter Umsetzungsvorschläge für diverse Anwendungs- bzw. Unternehmensszenarien. Dadurch kann der Implementierungsaufwand im Unternehmen erheblich reduziert werden. Dazu müssen zunächst Zieltypen identifiziert werden, die mögliche Anwendungsszenarien beschreiben. Anschließend werden für jeden Zieltyp konkrete Handlungsempfehlungen entwickelt. 
\item Erkenntnisse aus der Industrie können hinzugezogen werden. Denkbar ist eine Erhebung von Erfahrung aus (Lean) Start-ups und \gls{kmu}, die sich bereits mit Serienanläufen beschäftigt haben. Für die Erhebung eignen sich \gls{bspw} Fragebögen, Interviews oder Veröffentlichungen (Whitepaper, Präsentationen). 
\end{enumerate}



\subsection*{Implikationen für die Wirtschaft}

Für eine erfolgreiche Umsetzung der erarbeiteten Vorschäge im Unternehmen sind folgende Voraussetzungen identifiziert worden: 
\begin{enumerate}
 \item \textbf{Kompetenzen: } Die verantwortlichen Mitarbeiter müssen ein Grundverständnis für die Denkweise, Begriffe und Methoden (z.B.: QFD, Ishikawa-Diagramm) des Qualitätsmanagements vorweisen. 
 \item \textbf{Ressourcen: } Für die Umsetzung müssen genügend Ressourcen (Personal, Zeit, Geld) zum richtigen Zeitpunkt vorhanden sein. Zu Beginn ist eine strategische Planung und Implementierung der Vorschäge erforderlich. Im weiteren Verlauf muss der operative Betrieb gewährleistet sein. 
 \item \textbf{Motivation: } Für den Einsatz der notwendigen Ressourcen insbesondere zu Beginn einer Produktplanung ist starke Motivation erforderlich. Die Motivation hängt u.a. vom Vermögen der Mitarbeiter ab, das Anlaufmanagement als kritisches Handlungsfeld zu identifizieren. 
\end{enumerate}

Die Bedeutung der aufgeführten Punkte wird deutlich, wenn der Fall eintritt, dass das Lean Start-up über keine eigene QM-Abteilung oder QM-Mitarbeiter verfügt. 