\chapter{Fazit \& Ausblick}\label{sec:fazit}

\section{Kritische Würdigung}

In diesem Abschnitt wird auf die Beantwortung der Forschungsfragen eingegangen. Hauptaugenmerk ist die kritische Reflektion der angewandten Methoden. 

\subsection*{FF 1.1: Was zeichnet das LSU mit Hinblick auf das Anlaufmanagement aus? Welche Anforderungen werden gestellt?}
Zunächst erfolgte eine Literaturrecherche zu Lean Start-up (\gls{lsu}) im Allgemeinen. Der Stand der Wissenschaft wurde zusammengefasst und dargestellt. Um die Anforderungen des \gls{lsu} an ein Anlaufmanagement-Modell zu identifizieren, wurde gezielt Literatur gesucht. Da nur wenig Literatur zu beiden Themenkomplexen existiert, wurde die Recherche auf die Anforderungen von \gls{kmu} ausgeweitet. Für zukünftige Forschung ist auch eine Erhebung von Anforderungen durch Umfragen oder Interviews mit (Lean-)Start-ups denkbar. Schließlich wurden Anforderungen identifiziert und in methodische und technische Anforderungen unterteilt. Dies erzeugt eine differenzierte Sichtweise des Anforderungsprofils. Das Anforderungsprofil (siehe Abschnitt \ref{sec:anforderungen}) fasst die Randbedingungen für die weitere Untersuchung zusammen. 

\subsection*{FF 1.2: Welche Aspekte des Anlaufmanagements sind für das LSU von Bedeutung?}
Die Auswahlkriterien für die weitere Analyse ergeben sich aus dem in der FF1.1 erarbeiteten Anforderungsprofil. 

\begin{enumerate}
 \item \textbf{Systematische Literaturrecherche zu Anlaufmanagement allgemein: } Nach kurzer Recherche ist festzustellen, dass bisher keine einheitliche Auffassung des Anlaufmanagements existiert. Da die Herangehensweise in Abschnitt \ref{sec:herangehensweise_gg} ausführlich beschrieben ist, sind die Ergebnisse nachvollziehbar. Eine Konsolidierung der wichtigsten Auffassungen zum Thema Anlaufmanagement ist erforderlich, um die Grundlage der Arbeit zu bilden. 
 \item \textbf{Konsolidierung anhand einschlägiger Literatur: } Für die Konsolidierung wurden nach zuvor definierten Kriterien vier einschlägige Quellen identifiziert. 
 % Der Auswahlprozess ist im Anhang dokumentiert
 Verschiedene Perspektiven der Autoren und unterschiedliche Begriffe für ähnliche Sachverhalte stellten hohe Anforderungen an die Analyse. Daher wurde die Software \gls{atlas} für die qualitative Datenanalyse verwendet. Mittels sog. \textit{Coding} erfolgte eine Systematisierung von Begrifflichkeiten und es konnten geeignete Textpassagen identifiziert und miteinander verglichen werden. 
 \item \textbf{Entwicklung eines Grundgerüsts: } Das Grundgerüst ergibt sich aus der Systematisierung der Begrifflichkeiten und bildet zusammen mit dem Anforderungsprofil die Grundlage für die Arbeit. 
 \item \textbf{Systematische Literaturrecherche anhand des Grundgerüsts und des Anforderungsprofils: } Mit Hilfe des Grundgerüsts und des Anforderungsprofils konnte eine zielgerichtete Literaturrecherche erfolgen. Die Qualität der Quellen wurde durch zuvor definierte Kriterien sichergestellt. 
 Die aufgrund des begrenzten Umfangs der Masterarbeit erforderliche Limitierung der Quellenanzahl führte dazu, dass einige Aspekte des Grundgerüsts nicht analysiert werden konnten. % DONE Achtung mit der Formulierung!! 
\end{enumerate}

\subsection*{FF 1.3: Wie könnte ein Anlaufmanagement-Ansatz für das LSU auf Basis des Stands der Wissenschaft zum Thema Anlaufmanagement aussehen?}
Zur Beantwortung der FF1.3 wurden die Erkenntnisse aus der Analyse zusammenfassend dargestellt. Daraus wurden Fragen abgeleitet, welche in ein Umsetzungsleitfaden fließen. Der Umsetzungsleitfaden unterstützt Unternehmer, die für einen optimalen Serienanlauf wichtigen Entscheidungen zu treffen und eine geeignete Strategie zu entwickeln. Der Leitfaden ist bzgl. der Darbietung an das Business Model Canvas (\gls{bmc}) %DONE glossary
angelehnt. Dadurch zielt der Leitfaden hauptsächlich auf individuell zu gestaltende Elemente ab.

Im Hauptteil sind jedoch viele allgemeine Handlungsempfehlungen identifiziert worden, welche nicht im Leitfaden darstellbar sind. Diese Handlungsempfehlungen sind von großem Interesse und sollten \gls{bspw} in einem \textit{Handbuch Anlaufmanagement für das \gls{lsu}} zusammengefasst werden. 

\section{Ausblick}

In diesem Abschnitt sollen die Konsequenzen bzw. Implikationen für die Wissenschaft und Wirtschaft sowie ein Ausblick gegeben werden. 

\subsection*{Implikationen für die Wissenschaft}
Bereits in der Einführung wurde darauf hingewiesen, dass eine Validierung der Ergebnisse im Rahmen dieser Arbeit nicht stattfinden wird. Daher sind die hier gewonnenen Erkenntnisse (AM-Modell, Umsetzungsleitfaden) zur Zeit als ein Vorschlag für eine Best Practice zu betrachten. Eine Validierung kann durch eine zweite Person % TODO Kann ich das so schreiben -> Robert
oder aber durch empirische Bestätigung in der Industrie erfolgen. 

Für die weitere Forschung wurden vier verschiedene Ansätze identifiziert: 

\begin{enumerate}
 \item Die zu untersuchenden Aspekte des Grundgerüsts können erweitert werden. Zunächst können die in dieser Arbeit vorgeschlagenen aber nicht untersuchten Aspekte Kooperationen, Lieferanten und Logistik %TODO check
 hinzugezogen werden. Möglich ist auch, dass neue Aspekte identifiziert oder hier erarbeitete Aspekte weg gelassen werden. % TODO Formulierung: Entfallen?  
\item Die Forschung kann quantitativ durch eine umfangreichere Literaturrecherche und -auswertung ergänzt werden. Dabei wird die Anzahl relevanter Quellen und Lösungsvorschläge erhöht. 
\item Die Forschung kann in der Abstraktionsebene variiert werden. Es können detailliertere Handlungsempfehlungen entwickelt werden. Denkbar ist auch die Erstellung konkreter Umsetzungsvorschläge für diverse Anwendungs- bzw. Unternehmensszenarien. Dadurch kann der Implementierungsaufwand im Unternehmen erheblich reduziert werden. Dazu müssen zunächst Zieltypen identifiziert werden, die mögliche Anwendungsszenarien beschreiben. Anschließend werden für jeden Zieltyp konkrete Handlungsempfehlungen entwickelt. 
\item Erkenntnisse aus der Industrie können hinzugezogen werden. Denkbar ist eine Erhebung von Erfahrung aus (Lean) Start-ups und \gls{kmu}, die sich bereits mit Serienanläufen beschäftigt haben. Für die Erhebung eignen sich \gls{bspw} Fragebögen, Interviews oder Veröffentlichungen (Whitepaper, Präsentationen). 
\end{enumerate}



\subsection*{Implikationen für die Wirtschaft}

Für eine erfolgreiche Umsetzung der erarbeiteten Vorschläge im Unternehmen sind folgende Voraussetzungen identifiziert worden: 
\begin{enumerate}
 \item \textbf{Kompetenzen: } Die verantwortlichen Mitarbeiter müssen ein Grundverständnis für die Denkweise, Begriffe und Methoden (z.B.: QFD, Ishikawa-Diagramm) des Qualitätsmanagements vorweisen. 
 \item \textbf{Ressourcen: } Für die Umsetzung müssen genügend Ressourcen (Personal, Zeit, Geld) zum richtigen Zeitpunkt vorhanden sein. Zu Beginn ist eine strategische Planung und Implementierung der Vorschläge erforderlich. Im weiteren Verlauf muss der operative Betrieb gewährleistet sein. 
 \item \textbf{Motivation: } Für den Einsatz der notwendigen Ressourcen insbesondere zu Beginn einer Produktplanung ist starke Motivation erforderlich. Die Motivation hängt u.a. vom Vermögen der Führungskräfte ab, das Anlaufmanagement als kritisches Handlungsfeld zu identifizieren. 
\end{enumerate}

Die Bedeutung der aufgeführten Punkte wird deutlich, wenn der Fall eintritt, dass das Lean Start-up über keine eigene QM-Abteilung oder QM-Mitarbeiter verfügt. 
Eine Möglichkeit die Umsetzung im Unternehmen zu vereinfachen, ist die weiter oben genannte Entwicklung von Handlungsempfehlungen für konkrete Anwendungsszenarien. Die einfachere Umsetzung wird allerdings mit erhöhtem Forschungs- und Entwicklungsaufwand, sowie der Nutzung von nicht 100\% auf den Anwendungsfall zugeschnittenen Lösungen erkauft. 