%	Dependencies: texlive texlive-lang-german texlive-latex-recommended texlive-xetex cm-super kile biber 
%
%
%	main.tex
%
%	main for Master Thesis @ QW TU Berlin 
%
%	author: Rudolph R. Maier	
%
%	TU Berlin - 2018
%
%	Zur Benutzung von Bibtex und Biblatex: http://www.ub.uni-konstanz.de/serviceangebote/literaturverwaltung/bibtex/bibtex-und-biblatex-benutzen.html
%				Bibtex:		\renewcommand{\bibname}{Literaturverzeichnis} 
%
%	für Biber: tex.stackexchange.com/questions/26516/how-to-use-biber
% 		   pdflatex biber pdflatex
%
%% 	definiert Dokumenttyp und Grundlegende Einstellungen:
%
%
\documentclass[a4paper,oneside,11pt,titlepage,	%, captions=nooneline	% captions=nooneline --> flushleft
% twocolumn
% bibtotoc, liststotoc, % obsolete options 
% longbibliography % deprecated: bibtex option
% bibtotoc - Lit.VZ im Inha1ltsVZ
]{scrreprt}
				
\usepackage[
% 	colorlinks=true,
% 	urlcolor=blue,
% 	linkcolor=red,
	pdfauthor={Rudolph Ribeiro Maier},
	pdftitle={Titel der Masterarbeit},
% 	pdfsubject={The Subject},
% 	pdfkeywords={Some Keywords},
	pdfproducer={Latex with hyperref},
 	pdfcreator={pdflatex}
	bookmarks,				% PDF index
]{hyperref}
						%	KOMA Script (als Europäische  Anpassung vorzuziehen) 
% 						%	scrartcl, scrreprt, scrbook, scrlttr2

%% Codierung
\usepackage[utf8]{inputenc}			%	Codierung im Editor fuer direkte Eingabe der Sonderzeichen (WIN: latin1 oder ansinew, MAC: applemac, alt.: utf8)
\usepackage[T1]{fontenc}			%	U.A. damit Umlaute in PDF Dokumenten gefunden werden,  (T1: PostScript Type 1)
\usepackage[ngerman]{babel}			%	Anpassung der Überschriften und Silbentrennung (ngerman. english,...)
%		
%% Seitenränder
% 
\usepackage{geometry}
\geometry{left=30mm, right=30mm, top=20mm, bottom=20mm}
% Disable single lines at the start of a paragraph (Schusterjungen)
\clubpenalty = 10000
% Disable single lines at the end of a paragraph (Hurenkinder)
\widowpenalty = 10000
\displaywidowpenalty = 10000

%    Kein Einrücken der Absätze
\setlength{\parindent}{0pt}

% Zeilenabstand & Font
\usepackage{setspace}			% singlespacing, onehalfspacing, doublespacing. Oder \setstretch{1.25} https://texblog.org/2011/09/30/quick-note-on-line-spacing/
							% wird später für den Fließtext mit \linespread{x} definiert

% % % % % % % % % % % % % % % % % % % % % % % % % % % % % % % % % % % % % % % % % % % % % % % % % % % % % % % 
% 
%	FONT SELECTION with Math support from The LATEX Font Catalogue: www.tug.dk/FontCatalogue
% 	Test with pdffonts main.pdf
% 
% \newcommand{\changefont}[3]{
% \fontfamily{#1} \fontseries{#2} \fontshape{#3} \selectfont}
% \renewcommand{\rmdefault}{lmr}
% 
% \addtokomafont{sectioning}{\rmfamily} 		% 	Überschriften mit Serifen!
% 
% 	Latin Modern - Enhanced Computer Modern
% 
% \usepackage{lmodern}	% works: LMRoman12-Regular
% 
%
% 	EB Garamond (T1)	The default is oldstyle numbers. I have set the numbers to be lining to display lining numbers as well as oldstyle numbers
% 
% \usepackage[lining,scaled=.95]{ebgaramond} % works: EBGaramond12-Regular 		
% 
% 
% 	Garamond
% 
% \usepackage[urw-garamond]{mathdesign}	% works: GaramondNo8-Reg	 (~/texmf/fonts/type1/...)
% 
% 
% 	Nimbus Roman, is a clone of Times
% 
% \usepackage{nimbus} % ?? SFRM1200
% 
% 
% 	TIMES
% 
% \usepackage{mathptmx}	% works: NimbusRomNo9L-Regu - das hatte ich in der BA für den Fließtext
% 
% 
% 	TEX Gyre Termes, an Enhanced Time font
% 
% \usepackage{tgtermes} %	works: TeXGyreTermes-Regular 
% 
% 
% 	Palatino
% 
% \usepackage[sc]{mathpazo} % works: URWPalladioL-Roma
% \linespread{1.05}         % Palatino needs more leading (space between lines)
% 
% 
% 	KP Serif
% 
% \usepackage{kpfonts} % works: Kp-Regular
% 
% 
% 	Utopia Regular with Fourier, nur eine von beiden verwenden
% 
% \usepackage{fourier} % works: Utopia-Regular
% \usepackage[adobe-utopia]{mathdesign} % works: Utopia-Regular
% 
% 	
% 	Helvetica SANS SERIF
% 
\usepackage[scaled=0.9]{helvet}
% 
% 

\renewcommand*{\rmdefault}{\sfdefault}			% definiert serifenlos für serifenschrift (Grundtext). http://texwelt.de/wissen/fragen/785/wie-stelle-ich-alle-schrift-in-meinem-dokument-auf-serifenlos


% New header
\usepackage{fancyhdr}
\fancypagestyle{text}{%
  % flush all default styles
  \fancyhf{} 
  % Left part of the Header
  \fancyhead[LO]{\nouppercase{\leftmark}}
  % Center Part of the header
  \fancyhead[C]{}
  % Right part of the header
  \fancyhead[RO]{\thepage}
  % upper ruler
  \renewcommand{\headrulewidth}{0.4pt}
}
\fancypagestyle{plain}{%
  % flush all default styles
  \fancyhf{} 
  % Left part of the Header
  \fancyhead[LO]{\nouppercase{\leftmark}}
  % Center Part of the header
  \fancyhead[C]{}
  % Right part of the header
  \fancyhead[RO]{\thepage}
  % upper ruler
  \renewcommand{\headrulewidth}{0.4pt}
}
\fancypagestyle{fzvz}{%
  % flush all default styles
  \fancyhf{} 
  % Left part of the Header
  \fancyhead[LO]{\nouppercase{\leftmark}}
%   \fancyhead[LO]{FOOOO}
  % Center Part of the header
  \fancyhead[C]{}
  % Right part of the header
  \fancyhead[RO]{\thepage}
  % upper ruler
  \renewcommand{\headrulewidth}{0.4pt}
%   \renewcommand{\chaptermark}[1]{\markboth{#1}{}}
}


% 
%% Mathematik
\usepackage{amssymb}				%	Mathematische Symbole (Pfeile etc...)
\usepackage{amsfonts}
\usepackage{amsmath}				%	Fuer Mathematische Gleichungen

\usepackage[right]{eurosym}			% 	Eurozeichen 
\usepackage{amsopn}				%	für \grad
\DeclareMathOperator{\grad}{grad}		%	für \grad

% 
%Setzt den equation-Zaehler nach jeder Section zurueck
% \numberwithin{equation}{section}	
%
%% Content Management
\usepackage{lipsum}
\usepackage{subfigure} 				%	Grafiken nebeneinander : http://www.golatex.de/zwei-bilder-nebeneinander-t1915.html
\renewcommand{\floatpagefraction}{.8}		% 	Figure Objekte erst ab x % alleine auf einer Seite ohne Text
% \begin{figure} 
%     \subfigure[Bezeichnung der linken Grafik]{\includegraphics[width=0.49\textwidth]{ordner/name1.jpg}} 
%     \subfigure[Bezeichnung der rechten Grafik]{\includegraphics[width=0.49\textwidth]{ordner/name2.jpg}} 
% \caption{Titel unterm gesamten Bild} 
% \end{figure}
\usepackage{pstricks}				% 	PSTRICKS .tex Grafiken von DIA 
\usepackage{tikz}				% 	einbinden von DIA Grafiken (PGF?)
\usepackage{graphicx}				%	einbinden von Graphiken :	\includegraphics{schachbrett.eps}
\usepackage{colortbl}				% 	Für \rowcolor[gray]{0.9} zum Einfärben von Tabellenzeilen
% \graphicspath{{img/}}
% 
\usepackage{pdfpages} 				%	PDF include : 			\includepdf[pages={5,8,10-14}]{internal_rate_of_return.pdf}
\usepackage{listings}				%	Wie \begin{verbatim} : 		\begin{lstlisting}
						%	add hypertext capabilities
% 
% disable fucking ugly boxes
\hypersetup{pdfborder = 0 0 0}
% \booktabs					%	Für die Tabellen
\usepackage{tabularx}
%\usepackage{pdflscape}				%	Querformat
%\usepackage{enumitem}				%	Für bessere nummerierungen
%
% VON MAX
%\usepackage{subfigure}                         % mehrere Graphiken in einer Abbildung
%\usepackage{float}                             % erweiterte floating Befehle
%\usepackage[section]{placeins}                 % definiert \FloatBarrier
\usepackage[locale=DE]{siunitx}	
% 
% 
\usepackage[
    backend=biber,
    style=alphabetic, 	%numeric,
    bibstyle=alphabetic,
    sortlocale=de_DE,
%     natbib=true,
%     url=false, 
    isbn=false,
    doi=true,
%     defernumbers=true, % 
%     ibidtracker=context, %damit ebd. funktioniert 
%     eprint=false
]{biblatex} % Biber 
\usepackage{csquotes}				% When using babel or polyglossia with biblatex, loading csquotes is recommended to ensure that quoted texts are typeset according to the rules of your main language.
\addbibresource{main.bib}
% 	Name, Vorname: 
\DeclareNameAlias{default}{last-first}
%	Doppelpunkt nach Autor ( Anstatt Punkt)
\renewcommand*{\labelnamepunct}{\addcolon\addspace}
%
% \usepackage{german,longtable}
%

%------------------------------------------------------
% Add Glossary Functionality
\usepackage[
nonumberlist, 	% don't display page location where Term is used
acronym,      % create acronym list
% toc,           % Add GLossary location to Table of Contents..
% section
]      % ..as a section/chapter (but without number!)
{glossaries}
%
% Make \gls not fragile (useful if used within \caption{})
\robustify{\gls}
\robustify{\glspl}
%
% deactivate the default . after descriptions in the Glossary
\renewcommand*{\glspostdescription}{}
 
% Let the makefile build a glossary
\makeglossaries
%
% Include glossary.tex file with all the definitions
% usage: \gls{id}
%
\newglossaryentry{xxx}
{
  name=xx, 
  description={xxx}
}
%
\newglossaryentry{iot}
{
  name=IoT, 
  description={Internet of Things}
}
%
%
\newglossaryentry{erp}
{
  name=ERP, 
  description={Enterprise-Resource-Planning}
}
%
\newglossaryentry{ipdm}
{
  name=IPDM, 
  description={Integrierte Produktdatenmodelle}
}
%
\newglossaryentry{lsu}
{
  name=LSU, 
  description={Lean Start-up}
}
\newglossaryentry{sop}
%
{
  name=SOP, 
  description={Beginn der Serienproduktion}
}
%
\newglossaryentry{smed}
{
  name=SMED, 
  description={Single Minute Exchange of Dies, Methode zur Senkung von Rüstzeiten}
}
%
\newglossaryentry{heijunka}
{
  name=Heijunka, 
  description={
%(平準化),
Harmonisierung des Produktionsflusses}
}

\newglossaryentry{bspw}
{
  name=bspw.,
  description={Beispielsweise}
}
%
\newglossaryentry{mvp}
{  
  name=MVP,
  description={minimal überlebensfähiges Produkt, engl.: Minimum Viable
Product}
}

%
%%====================================================================================================
%
%
%------------------------------------------------------
% Add Blank page Functionality
\usepackage{afterpage}
\newcommand\blankpage{%
    \null
    \thispagestyle{empty}%
    \addtocounter{page}{-1}%
    \newpage}
%     
%%====================================================================================================


% % % % % % % % % % % % % % % % % % % % % % % % % % % % % % % % % % % % % % % % % % % % % % % % % % % % % % % 
\begin{document}
%
% \changefont{ptm}{m}{n}
% 
\setcounter{page}{-1}
\pagenumbering{roman}
% \includepdf{./img/deckblatt.pdf}

\titlehead
{Technische Universität Berlin\\
Fakultät V  Verkehrs- und Maschinensysteme\\
Institut für Werkzeugmaschinen und Fabrikbetrieb IWF\\
Fachgebiet Qualitätswissenschaft\\
Prof. Dr.-Ing. Roland Jochem\\
Dipl.-Ing. Robert Mies
\\
}
\subject{Masterarbeit}
% 						%
\title{Entwicklung eines Anlaufmodells für das Lean Start-up}
% 						%
\subtitle{\author{Rudolph Ribeiro Maier (330466)} 
}
						
\date{19.07.2018}
 % This is the titlepage
\maketitle
% \includepdf{latex_settings/AUF1_70.pdf}			% Aufgabenstellung
% \includepdf{latex_settings/AUF2_70.pdf}
% \blankpage
%
%
%=====================================================
% Load Declaration of Authorship
\newpage
\pagestyle{text}
\thispagestyle{empty}
\section*{Eidesstattliche Erklärung}
\begin{verbatim}

\end{verbatim}
Hiermit erkl\"{a}re ich, % Rudolph Manuel Ribeiro Maier, 
dass ich die vorliegende Arbeit %, betitelt  \textit{\enquote{Design eines Energie Harvesting Moduls für autonome Energieversorgung von Bodensensoren}} 
selbstst\"{a}ndig und eigenh\"{a}ndig sowie ohne unerlaubte fremde Hilfe und ausschließlich unter Verwendung der aufgef\"{u}hrten Quellen und Hilfsmittel angefertigt habe. \\

Die selbstständige und eigenständige Anfertigung versichert an Eides statt:

\begin{verbatim}

\end{verbatim}
\hrulefill\\
\hspace*{2cm}Unterschrift
\hfill
Berlin, 19. Juli 2018

%=====================================================
%
% \include{acknowledgment}
% \include{summary}
% \newpage
% 
% \begin{abstract}
\chapter*{Kurzzusammenfassung}

\lipsum[1-5]
\addcontentsline{toc}{chapter}{Kurzzusammenfassung}
% \end{abstract}

\tableofcontents
%
% %=====================================================
% Print Glossary
\newpage
% \pagestyle{glossary}
% \glsaddall
\printglossary[%style=altlist,
title=Abkürzungsverzeichnis]%,toctitle=Abkürzungsverzeichnis]
%  Damit erscheint es im InhaltsVZ als chapter. Die eingebaute Funktion beim Laden von glossaries unterstützt nur section, nicht chapter
\addcontentsline{toc}{chapter}{Abkürzungsverzeichnis}
% %
% %=====================================================
% \pagestyle{fzvz}
% % \renewcommand{\chaptermark}[1]{\markboth{#1}{}}
% \newpage
% \chaptermark{Verzeichnis der Formelzeichen}	% für chaptername im fancyheader, nur hier nötig
% \addcontentsline{toc}{chapter}{Verzeichnis der Formelzeichen}
% \input{./tab/fvz1}
% \newpage
% \input{./tab/fvzg1}
% \newpage
% % \addcontentsline{toc}{chapter}{\listfigurename}
% \listoffigures
% \newpage
% \listoftables
% 
%%====================================================================================================
%             				 TEXTANFANG
%%====================================================================================================
%
% \linespread{}

\pagestyle{text}
% Add Left Header
% 
\setstretch{1.5}		% is equivalent to MS Word Zeilenabstand 1.5
%
% \include{0_work}
\newpage
\setcounter{page}{0}
\pagenumbering{arabic}
% \section{Einführung}
\chapter{Einführung}
\section{Motivation \& Problemstellung}
Die produzierende Industrie findet sich heutzutage in einem zunehmend dynamischen Wettbewerbsumfeld wieder, welches vielschichtige Herausforderungen mit sich bringt \cite{Renner2012}. Die hauptsächlichen Herausforderungen liegen in steigenden Innovationsgeschwindigkeiten, kürzeren Produktlebenszyklen und einer höheren Variantenvielfalt \cite{Kuhn2002,Stauder2016}. Um dem durch die Globalisierung verstärkten Wettbewerb standzuhalten, müssen produzierende Unternehmen innovative Produkte und Dienstleistungen anbieten und sich zunehmend kundenorientiert aufstellen \cite{Surbier2014}. 
Eine zentrale Rolle wird hier dem Anlauf von Serienprodukten zugeschrieben. Aufgrund immer kürzer werdender Produktlebenszyklen rücken Kosten und Zeitaufwand in den Vordergrund \cite{Winkler2007}. So hat der Anlauf einen signifikanten Einfluss auf den wirtschaftlichen Erfolg des Produkts und die Time-to-Volume \cite{Klocke16}. Selbst ein um wenige Monate verschobener Verkaufsstart kann über Erfolg oder Misserfolg des Produkts entscheidend sein \cite{Schuh2008a}. Die Bedeutung der Serienanläufe findet bisher in der Wissenschaft keine angemessene Aufarbeitung \cite{Dyckhoff2012}. 

\section{Fokus der Arbeit}
Der Trend zur Konzentration auf Kernkompetenzen sorgt dafür, dass in großen Unternehmen immer mehr Wertschöpfungsanteile an Zulieferer abgegeben werden  \cite{Hilmola2015, Wildemann2008}. Der Gesamtanlauf setzt sich fortan aus vielen lokalen Einzelanläufen zusammen \cite{Zimolong2006}. Daraus resultieren höhere Abhängigkeiten zwischen größeren Unternehmen und den Zulieferern, die meist mittelständische Unternehmen sind. 

Die Abschlussarbeit soll sich im Speziellen mit dem Serienanlauf im KMU und SME als Zulieferer für größere Unternehmen beschäftigen, da hier erhebliches Verbesserungspotential erkennbar ist \cite[S.18]{Dombrowski2009a}. So gibt es in KMU meist keine Anlaufprozesse. Da es in KMU oft keine Stabsstellen gibt, werden Anläufe von den Mitarbeitern oft zusätzlich zum Tagesgeschäft gesteuert \cite{Dombrowski2009}. %TODO kein Zugriff auf Primärquelle D.Spath!! 
Mangelnde finanzielle und zeitliche Kapazitäten sowie fehlendes Know-how verhindern eine nachvollziehbare Dokumentation sowie proaktive Maßnahmen \cite{Zimolong2006,Dombrowski2009a}. 

Weiterhin soll untersucht werden, wie der Auftraggeber den Anlaufprozess des Lieferanten unterstützen kann. Größere Unternehmen verfügen in der Regel über mehr Ressourcen und teilweise eigene Anlaufprozesse. Im Zuge der Verlagerung der Wertschöpfungsanteile, gewinnt die Innovationskraft von Modul- und Systemlieferanten zunehmend an Bedeutung für den Erfolg eines Produktes \cite{Kuhn2002}. Ein erfolgreiches und effizientes Anlaufmanagement in KMU ist im Sinne der Entwicklung einer nachhaltigen Partnerschaft für Auftraggeber und Lieferant von großer Bedeutung. \textit{Wildemann} erkennt hier das Potenzial von Einspareffekten sowie Nutzung erheblicher Wettbewerbsvorteile auf beiden Seiten \cite{Wildemann2008}.

\textit{Dyckhoff} und \textit{Scholz} sind zu der Erkenntniss gekommen, dass das Thema weder in Industrie noch in der Wissenschaft hinreichend Beachtung findet \cite{Dyckhoff2012, Scholz2010}, weshalb hier keine zufriedenstellenden Ergebnisse zu erwarten sind.
Ziel der Arbeit ist, einen Überblick über den Stand der Forschung zu geben und einen Entwurf für ein Anlaufmodell zu entwickeln. 

\section{Herangehensweise}
Die Abschlussarbeit wird eine Literaturarbeit. In der Einführung erfolgt eine knappe Darstellung der zu behandelnden Themen Lean Startup / KMU und Anlaufmanagement. Im Hauptteil wird zunächst der Stand der Wissenschaft zum Thema Lean Startup skizziert. Den größeren Teil bildet eine umfassende Literaturanalyse zum Stand der Wissenschaft des Anlaufmanagements. Die Literaturrecherche erfolgt nach fest definierten Kriterien. Für die Literaturanalyse werden mit Hilfe des Tools \textit{Atlas.ti} alle relevanten Textstellen gecoded, d.h. identifiziert und nachvollziehbar dokumentiert. Anhand der  Ergebnisse wird anhand von möglichst vielen Quellen der Stand der Wissenschaft dargestellt. Im nächsten Abschnitt werden für das Lean Startup nicht berücksichtigte Anforderungen an das Anlaufmanagement ermittelt und daraus eine Art Anlaufmodell abgeleitet. 

Die Validierung der Ergebnisse erfolgt durch Zitierung der Quellen. Auf eine Validierung durch Experten, Fragebögen oder empirische Untersuchungen wird aufgrund des großen Umfangs verzichtet.
%
\section{Kontext}

\subsection{Lean Start-up}
\subsubsection*{Einführung}
Das Lean Start-up ist eine Businessmethode für dynamische Unternehmen oder Projekte, die hohen Risiken und Unsicherheiten ausgesetzt sind. 
Hauptziele der Methode sind kürzere Entwicklungszeiten, Einsparung von Kosten in der Entwicklungsphase und frühzeitiges Erkennen der Kundenbedürfnisse. 
Sie ist eine Antwort auf hoch dynamische Märkte, unbekannte Problemstellungen und Lösungen und hohen Risiken. Die Ursprünge liegen in den Denkweisen von Taiichi Ōno, W. Edwards Deming und Peter Drucker. 
2008 übertrug Eric Ries Lean Produktions Methoden auf hochtechnologie Startups und veröffentlichte 2011 die erstmals "Lean Startup" genannte Methode in seinem Buch. %TODO 2011 or 2008? 

%\subsection*{Definitionen}

\subsubsection*{Bestandteile}

\textit{1. Entwickeln einer Vision}. Die Vision dient als Grundlage für alle weiteren Handlungen. Aus ihr werden im nächsten Schritt Hypothesen abgeleitet. Anstatt einen aufwändigen Businessplan zu erstellen wird die Vision in einem Business Model Canvas definiert \cite{Blank2013}. Die Vision eines Lean Start-up zeichnet sich durch viele Freiheitsgrade und Unsicherheiten aus. 

\textit{2. Überführen der Vision hin zu Hypothesen}. Für jedes Element der im Business Model Canvas beschriebenen Vision werden Hypothesen abgeleitet. Die Hypothesen bilden die Freiheitsgrade und Unsicherheiten des BMC ab. Ziel ist, die Risiken durch spätere Beantwortung der Hypothesen zu minimieren. Nach Möglichkeit sollen die Hypothesn so formuliert werden, dass sie quantitativ beurteilt werden können. Die Hypothesen müssen wiederlegbar sein, um neue Erkenntnisse gewinnen zu können. 

\textit{3. Entwickeln von MVP Tests}. Ein minimal überlebensfähiges Produkt (\gls{mvp}, engl.: Minimum Viable Product) ist ein Werkzeug, mit dem man schnellstmöglich die Hypothesen am Kunden überprüfen kann \cite[93]{Ries2011}. Ziel ist zum einen den Build-Measure-Learn Zyklus zu beschleunigen, zum anderen die Lernrate in Bezug auf den Aufwand zu maximieren. So können frühzeitig nicht benötigte Funktionen und Produkteigenschaften erkannt und Zeit und Kosten gespart werden. Wenn die Entwicklung eines realen MVP zu aufwändig ist, kann ein Smoke Test eingesetzt werden. In einem Smoke Test wird das zukünftige Produkt in einem Video oder über eine Webseite vorgestellt.

\textit{4. Planung der Tests}. Bei der Durchführung der Tests kommt es darauf an, Kosten und Zeit zu minimieren. Daher werden zuerst Tests durchgeführt, die wenig kosten und hohe Risiken untersuchen. Beispielsweise ist eine Patentrecherche kostengünstig und kann frühzeitig sehr hohe Risiken aufdecken. Tests können nacheinander (seriell) oder gleichzeitig (parallel) durchgeführt werden. Bei parallelen Tests riskiert man, dass einzelne Tests überflüssig werden, profitiert jedoch von einem Zeitvorsprung gegenüber der seriellen Vorgehensweise. 

\textit{5. Interpretation der Ergebnisse}. Bei der Interpretation der Ergebnisse gibt es einige Fehlerquellen. Zum einen gibt es teilweise große Differenzen zwischen den geäußerten und reellen Kundenrückmeldungen. Zum anderen kann die Interpretation des Unternehmers durch eigene Wünsche oder Erwartungen verzerrt sein.

\textit{6. Reaktion}. Nach Auswertung der Ergebnisse sieht die LSU Methode eine Entscheidung zwischen drei Handlungsalternativen vor. \textit{Preserve}: Wenn die Tests die Hypothesen bestätigen wird die Strategie beibehalten. \textit{Pivot}: Wenn die Tests die Hypothesen wiederlegen oder neue Chancen aufzeigen, wird die Strategie angepasst. \textit{Perish}: Wenn die Tests die Hypothesn wiederlegen und der Unternehmer keine geeignete Strategie entwickeln kann, wird die Strategie verworfen. 

\textit{7. Skalierung und kontinuierliche Verbesserung}. Sobald alle relevanten Hypothesen bestätigt wurden, ist das Produkt auf den Markt abgestimmt. Jetzt kann massiv in Kundenakquise und Produktentwicklung investiert werden. Wichtig ist weiterhin, dass die Strategie fortwährend überprüft wird. Ein \textit{Pivot} ist auch nach der Skalierung bei größeren Änderungen sinnvoll. 

% \subsection*{Grenzen der Methodik}

\subsection{Anlaufmanagement}
Immer kürzere Produktlebenszyklen bei gleichzeitig höher werdenden Kundenwünschen und größerer Variantenvielfalt erhöhen die Komplexität und somit die Bedeutung des Serienanlaufs \cite{Kuhn2002,Schuh2004}. Die Risiken im Zusammenhang mit der Anlaufphase sind vielfältig. KUHN %TODO Biblatex Cite command for cap letter author in maintext
stellt fest, dass der Aufwand bis zum Erreichen einer stabilen Produktion oft unterschätzt wird. Infolgedessen kann es zum verspäteten Markteintritt sowie unzureichenden Kapazitäten und Qualitätsmängeln kommen \cite{Kuhn2002}. Um diesen Risiken entgegen zu wirken werden als übergeordnete Hauptziele für das Anlaufmanagement Beherrschung und zeitliche Verkürzung der Anlaufphasae genannt \cite{Kuhn2002, Schmitt2015}. 

Produktionsanläufe stellen auch deshalb eine große Herausforderung für Unternehmen dar, da sie hochkomplex sind und sich durch viele parallele und sequenzielle Teilprozesse auszeichnen. Sie sorgen zudem für eine starke Vernetzung der beteiligten Abteilungen innerhalb und außerhalb des Unternehmens \cite{Schuh2004}.


\subsubsection*{Definition}
In der Literatur existiert keine Einheitliche Definition des Begriffs Anlaufmanagement \cite[4]{Bischoff2007}. Selbst SCHMITT %TODO 
bemängelte 2015 ein fehlendes einheitliches Verständis der grundlegenden Begriffe des Produktionsanlaufs \cite[1]{Schmitt2015}. Vielmehr existieren unternehmensintern und teilweise auch projektspezifisch unterschiedliche Auffassungen über die Definition der Anlaufphase \cite[11]{Grosshenning2005}. KUHN %TODO
definierte das Anlaufmanagement wie folgt \cite[8]{Kuhn2002}: 
\begin{quotation}
Das Anlaufmanagement eines Serienproduktes umfasst alle Tätigkeiten und Maßnahmen zur Planung, Steuerung und Durchführung des Anlaufes mit den dazugehörigen Produktionssystemen, ab der Freigabe der Vorserie bis zum Erreichen einer geplanten Produktionsmenge, unter Einbeziehung vorgelagerter Prozesse und der nachgelagerten Prozesse im Sinne einer messbaren Eignung der Produkt- und Prozessreife.
\end{quotation}
SCHUH übernahm diese Auffassung \cite{Schuh08a} während RISSE und BISCHOFF den Beginn bereits nach der abgeschlossenen Produktentwicklung sehen \cite{Risse2002, Bischoff2007} (Freigabe Pflichtenheft).

Der Anwendungsbereich beschränkt sich nicht nur auf den Anlauf von neuen Produkten. Auch Modellderivate (Modellpflege), Varianten; neue Produktionssysteme, Fertigungsverfahren und Logistikprozesse stellen aus Perspektive des Managements ein Anlauf dar \cite[6]{Bischoff2007}. %TODO cite primara source LAICK/Warnecke/Aurich 2003

\subsubsection*{Lieferanten}

Ziele Werte Verhaltensnormen für Zusammenarbeit mit Lieferanten werden gemäß der Vision definiert Schmitt2015

Harmonisierung der Schnittstellen innerhalb der SC mit transparenten unternehmensübergreifende Strukturen Bischoff2007

Gemeinsame Informationsstrategie  Kuhn02

Frühe Einbindung und Integration der Lieferanten bedeutend für reibungslosen Anlauf Bischoff2007 S.28, Kuhn2002 S. 26

Einheitliche Datenbasis für den austausch von Informationen und Planungsdaten Kuhn02

Werkzeuge: 
  Lieferanten-Audits, KVP, PDCA Schuh08
  FMEA, QFD, Ishikawa, FTA Bischoff2007

\subsubsection*{Logistik}

Die Logistik beinhaltet die Koordinierung aller Material- und Informationsflüsse und Prozesse von Auftrag bis Auslieferung des Endprodukts. Die strategische Ebene beinhaltet die Entwicklung und Gestaltung der Wertschöpfungsnetzwerde und Prozesse nach logitsischen Prinzipien. Die operative Ebene beinhaltet die Lenkung und Kontrolle der Material- und Informationsflüsse und der dazugehörigen Prozesse. 
Hauptziele der Logistik ist, durch Gestaltung und Lenkung der logistischen Prozesse die Kundenbedürfnisse in den ökologischen, ökonomischen und sozialen Dimensionen optimal zu erfüllen \cite[28]{Schmitt2015}. 

Die Bedeutung der Logistik für die Anlaufphase ist durch die Globalisierung der Märkte, JiT-Konzepte und Reduzierung der Wertschöpfungstiefe gestiegen. Die Logistik hat zwei spezielle Funktionen in der Anlaufphase. Zum einen muss sie den Materialfluss der ersten Produkte bewerkstelligen. Zum anderen erprobt sie bereits Logistikprozesse für die Serie.
Durch den Querschnittscharakter der Logistik ist eine Abstimmung mit anderen Funktionsbereichen und der Logistik anderer Unterhehmen erforderlich \cite[1189]{Pfohl2000}.

\subsubsection*{Kooperationen}

\subsubsection*{Änderungen}

Definition: 
Technische Änderungen sind notwendige nachträgliche Anpassungen an bereits freigegebenen Entwicklungsständen \cite{Zanner2002}. Sie beinhalten immer eine Änderung der Dokumentation bzw. Datebnasis \cite[215]{Schuh2008}.
  %TODO cite Primärquelle Niemerg1997 - ZB Grimm-Zentrum 
  %Geschlossenes Außenmagazin 03a 
  %98 HA 8754 vorbestellt ins Campus Nord. 
Produktänderungen können in der Entwicklungs- und Konstruktionsphase bis zu 40\% der Gesamtressourcen beanspruchen \cite{Lindemann1998}
%TODO cite Lindemann1998 -> TUB QP624 77
Änderungsmanagement soll die Termintreue der Prozesse im Serienanlauf sicherstellen und die Durchlaufzeiten reduzieren \cite[216]{Schuh2008}. 

Ursachen: 
Auslöser für Änderungen können Gesetzesänderungen, interne Fehler, Qualitäts- und Sicherheitsprobleme, veränderte Kundenwünsche sowie eine veränderte Markt- und Wettbewerbssituation sein \cite{Zanner2002}. Auch treten Probleme oft erst dann in Erscheinung, wenn sie im Kontext der benachbarten Komponenten stehen \cite[24]{Kuhn2002}.

Konsequenzen: 
Änderungen bringen Konsequenzen mit sich. So führen sie zu steigendem Zeitdruck, einem erhöhten Personalaufwand in planerischen Abteilungen sowie können Kosten und Zeitverzögerungen aufgrund von Werkzeugänderungen entstehen \cite[24]{Kuhn2002}. 

Lösungsansatz und Bestandteile: 
Um den zeitlichen und finanziellen Aufwand gering zu halten, sollten Änderungen vermieden oder möglichst vorverlagert werden \cite{Schuh2008, Jania2004, Ass98}. 
% Schuh2008:215, Jania2004:69f, Ass98:107--131
%TODO Primärquelle Ass98 - TUB QP624 77
SCHUH teilt das Änderungsmanagement in Änderungsplanung, -ausführung und -absicherung ein \cite[217]{Schuh08}. 
%TODO cite original author Florian Rösch et. al.
LINDEMANN hingegen unterteilt das Thema detaillierter in vermeidung, Früherkennung, Problemanalyse, Lösungsfindung, Bewertung und Entscheidung. Die Erkenntisse werden mit Hilfe einer sog. Lernorientierten Auswertung im Sinne eines KVP ausgewertet \cite{Lindemann1998}. 
%TODO cite Primärquelle
%TODO KVP glossary, ausgewertet synonym

Enabler: 
Als Schlüsselrolle für erfolgreiches Änderungsmanagement wird oft die Kommunikation von Problemen und Änderungen innerhalb und über Unternehmensgrenzen hinweg genannt \cite{Kuhn2002, Schuh2008}.
% Kuhn2002:28+24, Schuh08:219
ZANNER betont die Bedeutung der Vertrauensverhältnisses für den Informationsaustausch und schlägt informelle standortübergreifende Treffen der Entwickler vor. Die Zuordnung eines Verantwortlichen Mitarbeiters für die Abwicklung einer Änderung soll helfen, die Schnittstellenprobleme bei der Arbeitsteiligen Arbeitsweise zu überwinden  \cite[42]{Zanner2002}.
Weiterhin werden eine einheitliche Terminologie \cite{Zanner2002} und Datenbasis sowie ein durchgängiges Versionsmanagement \cite{Kuhn2002} als Erfolgsfaktoren genannt. 
% Kuhn: Datenbasis S.25, Versionsmanagement S. 25

\subsection{Wissen}


% \cite*[prenote][postnote]{Schmitt2015}[extra] cite
% 
% \Cite*[prenote][postnote]{Schmitt2015}[extra] Cite
%  
% \footcite[prenote][postnote]{Schmitt2015}[extra] cite footcite
% 
% \Footcite[prenote][postnote]{Schmitt2015}[extra] cite Footcite
% 
% \textcite[prenote][postnote]{Schmitt2015}[extra] cite textcite
% 
% \citeauthor[prenote][postnote]{Schmitt2015}[extra] citeauthor
% 
% \citeauthor{Schmitt2015} Citeauthor
% 
% \citet{Schmitt2015} citet
% 
% \citep{Schmitt2015} citep
% 
% \autocites{Schmitt2015} autocites
% 
% \parencite{Schmitt2015} parencite		% \input{file} includes the commands and references
\chapter{Methodik}
\section{Anforderungen an das Anlaufmanagementmodell für das \gls{lsu}}
Damit das zu entwickelnde Modell den Serienanlauf im \gls{lsu} effektiv unterstützt, müssen zunächst einige Anforderungen formuliert werden. 

\textbf{Methodische Anforderungen}

Das zu entwickelnde Anlaufmanagementmodell muss sowohl horizontal als auch vertikal sinnvoll mit dem \gls{lsu} Ansatz kooperieren bzw. integriert werden. 
Die Beseitigung von Verschwendung sowie Anreize zur kontinuierlichen Verbesserung müssen strukturell im Modell verankert sein. 

Besondere Eigenschaften von (Lean-)Start-ups müssen berücksichtigt werden. Dazu zählen \gls{bspw} eine flache Hierarchie, eine kleine Anzahl an Mitarbeitern, Vorhandensein von Generalisten anstatt Spezialisten und Interdisziplinarität der Mitarbeiter und Aufgaben. Daraus werden folgende Forderungen abgeleitet: Eine kleine Anzahl an einfach anzuwendenden Methoden. Die Gestaltung von Ablauforganisation und Prozessen erfolgt mit geringem Detaillierungsgrad. An anderer Stelle soll jedoch mithilfe von Standardisierung die Komplexität der Lösungsalternativen beschränkt werden. Daraus folgt eine hohe Abstraktionsebene des Modells einerseits, andererseits jedoch ein hoher Detaillierungsgrad. 

Des weiteren wird eine Flexibilität des Modells gefordert. So muss das Modell, welches bereits in der Anfangsphase implementiert wird, bei schnellem Wachstum und stark veränderten Bedingungen weiterhin effektiv sein. Dazu zählen \gls{bspw} eine Skalierbarkeit der Methoden hinsichtlich Anzahl der Mitarbeiter sowie Mitarbeiterzuordnung von Kompetenzen und Aufgaben. 

\textbf{Technische Anforderungen}

Auch auf technischer Seite ist Flexibilität gefordert. Die Produktion bzw. der Anlauf müssen agil auf Stückzahlschwankungen reagieren können. Änderungen am Produkt oder die Einführung neuer Varianten müssen einfach und schnell mit hoher Qualität realisiert werden können. Analog dazu müssen Änderungen am Logistiksystem und Produktionslinie effizient durchgeführt werden können. 
Große Unsicherheiten sind ein inhärentes Merkmal des Serienanlaufs. Daher muss ein umfassendes Risikomanagement im Modell verankert sein. 


\section{Entwicklung des Grundgerüsts}

\section{Grundsätzliche Herangehensweise der Arbeit}



% \input{2_}}
\singlespacing
\newpage
\
\printbibliography %[title={Literaturverzeichnis}]
%
%
\newpage
\appendix
\chapter{Anlagenverzeichnis}

\section{Dombrowski-2011a - Lean Ramp-up. Handlungs- und Gestaltungsfelder}
\subsection{Die Handlungsfelder im Lean Ramp-up}\label{appendix:dom11a:hf}

\textbf{Produktentwicklung und Konstruktion}
umfassen alle Aufgaben, die sich mit
dem Konzipieren, Entwerfen, Ausar
beiten und Erproben eines Produkts
beschäftigen. Als Ergebnis resultie
ren Zeichnungen, Stücklisten und an
dere Produktdokumentationen.

\textbf{Fertigungs- und Montagemittel} umfas
sen alle Aufgaben, die sich mit der Be
darfsplanung, Auswahl, Beschaffung,
Herstellung, Einrichtung, Program
mierung und Inbetriebnahme von Ma
schinen, Anlagen, Werkzeugen und
Vorrichtungen beschäftigen. Dazu ge
hören auch die Wartung und Instand
haltung. Als Ergebnis resultiert ein
Fertigungs- und Montagekonzept.

\textbf{Fertigungs- und Montageprozesse} um
fassen alle Aufgaben, die sich mit der
Festlegung der Arbeitsabläufe zur
Herstellung eines Produkts in der
Fertigung und Montage beschäftigen.
Es werden u. a. Reihenfolgen und
Vorgabezeiten bestimmt sowie Pro
duktionsmittel zugeordnet. Als Er
gebnis resultieren Arbeitspläne, in
denen alle Informationen dokumen
tiert sind.

\textbf{Personal- und Arbeitsorganisation} um
fasst alle Aufgaben, die sich mit der
Bedarfs- und Einsatzplanung, Be
schaffung, Entwicklung und Freiset
zung von Personal sowie mit der Ge
staltung einer arbeitsgerechten und
bestmöglichen Zusammenarbeit von
Mensch und Technik beschäftigen.
Als Ergebnis resultieren zum Beispiel
Personaleinsatzpläne,
 Arbeitszeitund Entgeltsysteme.

\textbf{Produktionsplanung und -steuerung}
(PPS) umfasst alle Aufgaben, die sich
mit der Festlegung, Veranlassung,
Überwachung und Sicherung des Pro
duktionsprogramms nach Art und
Menge unter Berücksichtigung von
Terminen und Kapazitäten beschäfti
gen. Als Ergebnis resultiert ein Pro
duktionsplan mit Bedarfsmengen und
-terminen für Zukauf- und Eigenferti
gungsteile.

 \textbf{Einkaufs- und Dispositionsprozesse}
umfassen alle Aufgaben, die sich mit
der kostenoptimalen strategischen
und operativen Beschaffung von Zu
kaufteilen, Handelswaren, Betriebs
mitteln und Dienstleistungen von ei
nem Lieferanten beschäftigen. Als Er
gebnis resultieren zum Beispiel Sour
cingstrategien, Verträge mit Lieferan
ten und verfügbare Lagerbestände.


 \textbf{Logistikprozesse und Logistikmittel}
umfassen alle Aufgaben, die sich mit
der Festlegung von effizienten inner
und außerbetrieblichen Transporten
bzw. Materialflüssen und der Bereit
stellung von Gütern beschäftigen.
Außerdem werden Logistikmittel, wie
z. B. Lager- und Transportmittel be
stimmt. Als Ergebnis resultieren sog.
Logistiksysteme.

 \textbf{Gebäude, Layout und Arbeitsplätze}
umfassen alle fabrikplanerischen Auf
gaben, die sich mit der Festlegung, op
timalen Auslegung und Realisierung
der Produktionsstätten beschäftigen.
Der Umfang reicht dabei von der Um
gestaltung einzelner Arbeitsplätze bis
hin zur Errichtung neuer Gebäude.
Als Ergebnis resultieren eingerichtete
Arbeitsplätze, Flächen und Gebäude.


 \textbf{Qualitätsmanagement und Qualitäts
mittel} umfassen alle Aufgaben, die
sich mit der Planung, Lenkung, Prü
fung, Sicherung und Verbesserung
der Qualitätsmerkmale von Produk
ten, Prozessen und Leistungen be
schäftigen. Außerdem werden Quali
tätsmittel, wie z. B. Prüf- und Mess
mittel bestimmt. Als Ergebnis resul
tieren beispielsweise Arbeits- und
Prüfanweisungen.

\textbf{Informationsprozesse und -systeme}
umfassen alle Aufgaben, die sich mit
der Beschaffung, Verarbeitung, Über
tragung und Speicherung von Infor
mationen zur Integration und ziel
orientierten Steuerung aller operati
ven Prozesse beschäftigen. Als Ergeb
nis resultieren zum Beispiel Systeme
zur Betriebsdatenerfassung (BDE).


\subsection{Die Gestaltungsfelder im Lean Ramp-up}\label{appendix:dom11a:gf}
\textbf{Integration und Kooperation} umfassen
alle Methoden und Werkzeuge, die
fachbereichs-, phasen-, technologieund unternehmensübergreifend zur
Synchronisierung von Produkt- und
Produktionsentwicklung beitragen.
Dazu wird eine simultane, interdisziplinäre und partnerschaftliche Zusammenarbeit angestrebt. Ziel ist es,
zum Beispiel Schnittstellen und Änderungen zu reduzieren.
\textbf{Partizipation und Veränderung} umfassen alle Methoden und Werkzeuge,
die zur Motivation der Mitarbeiter
und zum Abbau bzw. zur Vermeidung
von Widerständen und Konflikten beitragen. Dazu werden alle betroffenen
Organisationseinheiten am ProdukBild 4. Gestaltungsfelder im Lean Ramp-up
tionsanlauf beteiligt. Ziel ist es, die
Potenziale der Mitarbeiter zu nutzen
und einen reibungslosen Anlauf zu erreichen.

 \textbf{Wertschöpfung und Just-in-Time (JIT)}
umfassen alle Methoden und Werkzeuge, die zur produktiven, schnellen
und termingerechten Herstellung
bzw. Lieferung der Produkte beitragen. Dazu werden alle Verluste in den
Produktions- und Logistikprozessen
eliminiert und eine fließende und
kundenorientierte Produktion aufgebaut. Ziel ist ein schlankes Produktionssystem.

 \textbf{Pilotierung und Qualifizierung} umfasst
alle Methoden und Werkzeuge, die
zur Absicherung von Produkt- und
Prozessreifegrad sowie zur Steigerung der Leistungsfähigkeit des Produktionssystems beitragen. Dazu
werden sog. Pilotbereiche eingerichtet in denen Produktionstests sowie
Mitarbeiterschulungen erfolgen. Ziel
ist eine steile Lern- bzw. Anlaufkurve.

 \textbf{Priorisierung und Standardisierung}
umfassen alle Methoden und Werkzeuge, die zur Reduzierung, Beherrschung und Vermeidung der technologischen, prozessualen und organisatorischen Komplexität im Produktionsanlauf beitragen. Dazu werden
Schwerpunkte gebildet und Referenzunterlagen erstellt. Ziel ist es, den
Aufwand im Produktionsanlauf zu reduzieren.

 \textbf{Reaktionsfähigkeit und Flexibilität} umfassen alle Methoden und Werkzeuge,
die zum zeitnahen Erkennen veränderter Randbedingungen und Störungen sowie zur kontinuierlichen Anpassung des Anlaufmanagements beitragen. Dazu werden Frühwarnsysteme etabliert und Handlungsoptionen
bestimmt. Ziel ist es, schnell auf Veränderungen und Störungen zu reagieren.

 \textbf{Fehler- und Risikovermeidung} umfasst
alle Methoden und Werkzeuge, die
zur präventiven Qualitätssicherung
und -verbesserung beitragen. Dazu
werden frühzeitig die Ergebnisse der
Produkt- und Produktionsentwicklung veranschaulicht und Fehler- bzw.
Risikopotentiale eliminiert. Ziel sind
eine hohe Produkt- und Prozessqualität sowie geringe Änderungs- und
Prüfkosten.

 \textbf{Problemlösung und Stabilisierung} umfassen alle Methoden und Werkzeuge,
die zur reaktiven Qualitätssicherung
und -verbesserung beitragen. Dazu
werden die Produkte und Prozesse
kontinuierlich überprüft und überwacht sowie systematisch Problemursachen beseitigt. Ziel ist eine Stabilisierung des Anlaufs und Vermeidung
von Folge- und Wiederholungsfehlern.

 \textbf{Wissenstransfer und KVP} umfassen
alle Methoden und Werkzeuge, die
zum Transfer von Erfahrungswissen
und zur Erhöhung der Mitarbeiterkompetenzen beitragen. Dazu wird
explizites und – soweit möglich – implizites Wissen identifiziert, gesammelt, aufbereitet und vermittelt. Ziel
ist es, mit dessen Nutzung und
Weiterentwicklung aktuelle und zukünftige Anläufe zu verbessern.

 \textbf{Transparenz und Visualisierung} umfassen alle Methoden und Werkzeuge,
die zur Verfügbarkeit und leicht verständlichen Darstellung von Informationen und Daten beitragen. Dazu
werden sowohl informations- und
kommunikationstechnische Systeme
als auch optische Hilfsmittel und Signale eingesetzt. Ziel ist die Regelung,
Steuerung und Verbesserung des Produktionsanlaufs.
  
\newpage
\blankpage
\end{document}
