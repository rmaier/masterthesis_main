%	Dependencies: texlive texlive-lang-german texlive-latex-recommended texlive-xetex cm-super kile biber 
%
%
%	main.tex
%
%	main for Master Thesis @ QW TU Berlin 
%
%	author: Rudolph R. Maier	
%
%	TU Berlin - 2018
%
%	Zur Benutzung von Bibtex und Biblatex: http://www.ub.uni-konstanz.de/serviceangebote/literaturverwaltung/bibtex/bibtex-und-biblatex-benutzen.html
%				Bibtex:		\renewcommand{\bibname}{Literaturverzeichnis} 
%
%	für Biber: tex.stackexchange.com/questions/26516/how-to-use-biber
% 		   pdflatex biber pdflatex
%
%% 	definiert Dokumenttyp und Grundlegende Einstellungen:
%
%
\documentclass[a4paper,oneside,11pt,titlepage,	%, captions=nooneline	% captions=nooneline --> flushleft
% pointlessnumbers,	% Kein Punkt bei Kapitelnummern, nur für scrreprt. Veraltet? (https://golatex.de/pointlessnumbers-pagestyle-t2337.html)
% draft, 		% Ggf. microtype benutzen, siehe scrguide.pdf (KOMA Script v. 20.08.2017) S.59
% twocolumn
% bibtotoc, liststotoc, % obsolete options 
% longbibliography % deprecated: bibtex option
% bibtotoc - Lit.VZ im Inha1ltsVZ % deprecated, use toc=bib
toc=bib, 	% bibliography in toc. see scrguide p.77ff
% toc=flat,	
toc=listof,	% listoffigures and listoftables in toc
]{scrreprt}
				
\usepackage[
 	colorlinks=true,
% 	urlcolor=blue,
 	linkcolor=red,
	pdfauthor={Rudolph Ribeiro Maier},
	pdftitle={Entwicklung eines Anlaufmodells fuer das Lean Start-up},
% 	pdfsubject={The Subject},
% 	pdfkeywords={Some Keywords},
	pdfproducer={Latex with hyperref},
 	pdfcreator={pdflatex}
	bookmarks,				% PDF index
]{hyperref}
						%	KOMA Script (als Europäische  Anpassung vorzuziehen) 
% 						%	scrartcl, scrreprt, scrbook, scrlttr2

%% Codierung
\usepackage[utf8]{inputenc}			%	Codierung im Editor fuer direkte Eingabe der Sonderzeichen (WIN: latin1 oder ansinew, MAC: applemac, alt.: utf8)
\usepackage[T1]{fontenc}			%	U.A. damit Umlaute in PDF Dokumenten gefunden werden,  (T1: PostScript Type 1)
\usepackage[ngerman]{babel}			%	Anpassung der Überschriften und Silbentrennung (ngerman. english,...)
%		
%% Seitenränder
% 
\usepackage{geometry}
\geometry{
  papersize={210mm,297mm},
  left=30mm, right=30mm, top=20mm, bottom=20mm, % margins
  headsep = \baselineskip,			% Abstand zwischen Kopf und Haupttext
  footskip = \dimexpr2\baselineskip+3mm\relax,	% Abstand zwischen Haupttext und der Grundlinie des Fußes
%   showframe,					% for debugging
%   includeheadfoot,				% include head/foot in margins
  }
%   
\clubpenalty = 10000				% Disable single lines at the start of a paragraph (Schusterjungen)
\widowpenalty = 10000				% Disable single lines at the end of a paragraph (Hurenkinder)
\displaywidowpenalty = 10000


\setlength{\parindent}{0pt}			% Kein Einrücken der Absätze

% Zeilenabstand & Font
\usepackage{setspace}				% singlespacing, onehalfspacing, doublespacing. Oder \setstretch{1.25} https://texblog.org/2011/09/30/quick-note-on-line-spacing/
						% wird später für den Fließtext mit \linespread{x} definiert

% % % % % % % % % % % % % % % % % % % % % % % % % % % % % % % % % % % % % % % % % % % % % % % % % % % % % % % 
% 
%	FONT SELECTION with Math support from The LATEX Font Catalogue: www.tug.dk/FontCatalogue
% 	Test with pdffonts main.pdf
% 
% \newcommand{\changefont}[3]{
% \fontfamily{#1} \fontseries{#2} \fontshape{#3} \selectfont}
% \renewcommand{\rmdefault}{lmr}
% 
% \addtokomafont{sectioning}{\rmfamily} 		% 	Überschriften mit Serifen!
% 
% 	Latin Modern - Enhanced Computer Modern
% 
% \usepackage{lmodern}	% works: LMRoman12-Regular
% 
%
% 	EB Garamond (T1)	The default is oldstyle numbers. I have set the numbers to be lining to display lining numbers as well as oldstyle numbers
% 
% \usepackage[lining,scaled=.95]{ebgaramond} % works: EBGaramond12-Regular 		
% 
% 
% 	Garamond
% 
% \usepackage[urw-garamond]{mathdesign}	% works: GaramondNo8-Reg	 (~/texmf/fonts/type1/...)
% 
% 
% 	Nimbus Roman, is a clone of Times
% 
% \usepackage{nimbus} % ?? SFRM1200
% 
% 
% 	TIMES
% 
% \usepackage{mathptmx}	% works: NimbusRomNo9L-Regu - das hatte ich in der BA für den Fließtext
% 
% 
% 	TEX Gyre Termes, an Enhanced Time font
% 
% \usepackage{tgtermes} %	works: TeXGyreTermes-Regular 
% 
% 
% 	Palatino
% 
% \usepackage[sc]{mathpazo} % works: URWPalladioL-Roma
% \linespread{1.05}         % Palatino needs more leading (space between lines)
% 
% 
% 	KP Serif
% 
% \usepackage{kpfonts} % works: Kp-Regular
% 
% 
% 	Utopia Regular with Fourier, nur eine von beiden verwenden
% 
% \usepackage{fourier} % works: Utopia-Regular
% \usepackage[adobe-utopia]{mathdesign} % works: Utopia-Regular
% 
% 	
% 	Helvetica SANS SERIF
% 
\usepackage[scaled=0.9]{helvet}
% 
% 

\renewcommand*{\rmdefault}{\sfdefault}		% definiert serifenlos für serifenschrift (Grundtext). http://texwelt.de/wissen/fragen/785/wie-stelle-ich-alle-schrift-in-meinem-dokument-auf-serifenlos


% New header
\usepackage{scrlayer-scrpage}			% replaced fancyhdr, see: https://tobiw.de/tbdm/layout-2
\clearpairofpagestyles
 
\setkomafont{pageheadfoot}{\sffamily\footnotesize}
\setkomafont{pagehead}{\normalsize}		% {\bfseries}
\setkomafont{pagination}{}

\KOMAoptions{
   headsepline = true,
   footsepline = false,
   plainheadsepline = true, 
   plainfootsepline = false,			% no function? 
}
\automark[chapter]{chapter}			% syntax: [plainoption]{normaloption}; *{forboth}
\ihead*{\headmark}
\ohead*{\pagemark}
% \ofoot[\pagemark]{}				% Seitenzahlen unten für plain
% \ohead*{Seite~\pagemark}

% \usepackage{fancyhdr}
% \fancypagestyle{text}{%
  % flush all default styles
  \fancyhf{} 
  % Left part of the Header
  \fancyhead[LO]{\nouppercase{\leftmark}}
  % Center Part of the header
  \fancyhead[C]{}
  % Right part of the header
  \fancyhead[RO]{\thepage}
  % upper ruler
  \renewcommand{\headrulewidth}{0.4pt}
}
\fancypagestyle{plain}{%
  % flush all default styles
  \fancyhf{} 
  % Left part of the Header
  \fancyhead[LO]{\nouppercase{\leftmark}}
  % Center Part of the header
  \fancyhead[C]{}
  % Right part of the header
  \fancyhead[RO]{\thepage}
  % upper ruler
  \renewcommand{\headrulewidth}{0.4pt}
}
\fancypagestyle{fzvz}{%
  % flush all default styles
  \fancyhf{} 
  % Left part of the Header
  \fancyhead[LO]{\nouppercase{\leftmark}}
%   \fancyhead[LO]{FOOOO}
  % Center Part of the header
  \fancyhead[C]{}
  % Right part of the header
  \fancyhead[RO]{\thepage}
  % upper ruler
  \renewcommand{\headrulewidth}{0.4pt}
%   \renewcommand{\chaptermark}[1]{\markboth{#1}{}}
}

% \usepackage{titleps}				%	https://tex.stackexchange.com/questions/127972/package-fancyhdr-no-dot-behind-chapter-number-in-header
% \input{latex_settings/titleps}
% 
%% Mathematik
\usepackage{amssymb}				%	Mathematische Symbole (Pfeile etc...)
\usepackage{amsfonts}
\usepackage{amsmath}				%	Fuer Mathematische Gleichungen

\usepackage[right]{eurosym}			% 	Eurozeichen 
\usepackage{amsopn}				%	für \grad
\DeclareMathOperator{\grad}{grad}		%	für \grad

% 
%Setzt den equation-Zaehler nach jeder Section zurueck
% \numberwithin{equation}{section}	
%
%% Content Management
\usepackage{lipsum}
\usepackage{subfigure} 				%	Grafiken nebeneinander : http://www.golatex.de/zwei-bilder-nebeneinander-t1915.html
\renewcommand{\floatpagefraction}{.8}		% 	Figure Objekte erst ab x % alleine auf einer Seite ohne Text
% \begin{figure} 
%     \subfigure[Bezeichnung der linken Grafik]{\includegraphics[width=0.49\textwidth]{ordner/name1.jpg}} 
%     \subfigure[Bezeichnung der rechten Grafik]{\includegraphics[width=0.49\textwidth]{ordner/name2.jpg}} 
% \caption{Titel unterm gesamten Bild} 
% \end{figure}
\usepackage{pstricks}				% 	PSTRICKS .tex Grafiken von DIA 
\usepackage{tikz}				% 	einbinden von DIA Grafiken (PGF?)
\usepackage{graphicx}				%	einbinden von Graphiken :	\includegraphics{schachbrett.eps}
\usepackage{colortbl}				% 	Für \rowcolor[gray]{0.9} zum Einfärben von Tabellenzeilen
% \graphicspath{{img/}}
% 
\usepackage{pdfpages} 				%	PDF include : 			\includepdf[pages={5,8,10-14}]{internal_rate_of_return.pdf}
\usepackage{listings}				%	Wie \begin{verbatim} : 		\begin{lstlisting}
						%	add hypertext capabilities
% 
% disable fucking ugly boxes
\hypersetup{pdfborder = 0 0 0}
% \booktabs					%	Für die Tabellen
\usepackage{tabularx}
%\usepackage{pdflscape}				%	Querformat
%\usepackage{enumitem}				%	Für bessere nummerierungen
%
% VON MAX
%\usepackage{subfigure}                         % mehrere Graphiken in einer Abbildung
%\usepackage{float}                             % erweiterte floating Befehle
%\usepackage[section]{placeins}                 % definiert \FloatBarrier
\usepackage[locale=DE]{siunitx}	
% 
% 
\usepackage[
    backend=biber,
    style=alphabetic, 	%numeric,
    bibstyle=alphabetic,
    sortlocale=de_DE,
%     natbib=true,
    url=false, 
    isbn=false,
    doi=true,
%     defernumbers=true, % 
%     ibidtracker=context, %damit ebd. funktioniert 
%     eprint=false
]{biblatex} % Biber 
\usepackage{csquotes}				% When using babel or polyglossia with biblatex, loading csquotes is recommended to ensure that quoted texts are typeset according to the rules of your main language.
\addbibresource{main.bib}
% 	Name, Vorname: 
\DeclareNameAlias{default}{last-first}
%	Doppelpunkt nach Autor ( Anstatt Punkt)
\renewcommand*{\labelnamepunct}{\addcolon\addspace}
%
% \usepackage{german,longtable}
%

%------------------------------------------------------
% Add Glossary Functionality
\usepackage[
nonumberlist, 	% don't display page location where Term is used
acronym,      % create acronym list
toc,           % Add GLossary location to Table of Contents..
% section
]      % ..as a section/chapter (but without number!)
{glossaries}
%
% Make \gls not fragile (useful if used within \caption{})
\robustify{\gls}
\robustify{\glspl}
%
% deactivate the default . after descriptions in the Glossary
\renewcommand*{\glspostdescription}{}

\addto\captionsngerman{%						% Translation for glossaries
 \renewcommand\glossaryname{Abkürzungsverzeichnis}}			% see: https://www.mrunix.de/forums/showthread.php?65522-Glossaries-Verzeichnisbezeichnungen-umbenennen
\deftranslation[to=ngerman]{Acronyms}{Abkürzungsverzeichnis (trans)}
\deftranslation[to=ngerman]{Glossary}{Stichwortverzeichnis (trans)} 
% Let the makefile build a glossary
\makenoidxglossaries
%
% Include glossary.tex file with all the definitions
% usage: \gls{id}
%
\newglossaryentry{xxx}
{
  name=xx, 
  description={xxx}
}
%
\newglossaryentry{iot}
{
  name=IoT, 
  description={Internet of Things}
}
%
%
\newglossaryentry{erp}
{
  name=ERP, 
  description={Enterprise-Resource-Planning}
}
%
\newglossaryentry{ipdm}
{
  name=IPDM, 
  description={Integrierte Produktdatenmodelle}
}
%
\newglossaryentry{lsu}
{
  name=LSU, 
  description={Lean Start-up}
}
\newglossaryentry{sop}
%
{
  name=SOP, 
  description={Beginn der Serienproduktion}
}
%
\newglossaryentry{smed}
{
  name=SMED, 
  description={Single Minute Exchange of Dies, Methode zur Senkung von Rüstzeiten}
}
%
\newglossaryentry{heijunka}
{
  name=Heijunka, 
  description={
%(平準化),
Harmonisierung des Produktionsflusses}
}

\newglossaryentry{bspw}
{
  name=bspw.,
  description={Beispielsweise}
}
%
\newglossaryentry{mvp}
{  
  name=MVP,
  description={minimal überlebensfähiges Produkt, engl.: Minimum Viable
Product}
}

%
%%====================================================================================================
%
%
%------------------------------------------------------
% Add Blank page Functionality
\usepackage{afterpage}
\newcommand\blankpage{%
    \null
    \thispagestyle{empty}%
    \addtocounter{page}{-1}%
    \newpage}
%     
%%====================================================================================================


% % % % % % % % % % % % % % % % % % % % % % % % % % % % % % % % % % % % % % % % % % % % % % % % % % % % % % % 
\begin{document}
%
% \changefont{ptm}{m}{n}
% 
\setcounter{page}{-1}
\pagenumbering{roman}
% \includepdf{./img/deckblatt.pdf}

\titlehead
{Technische Universität Berlin\\
Fakultät V  Verkehrs- und Maschinensysteme\\
Institut für Werkzeugmaschinen und Fabrikbetrieb IWF\\
Fachgebiet Qualitätswissenschaft\\
Prof. Dr.-Ing. Roland Jochem\\
Dipl.-Ing. Robert Mies
\\
}
\subject{Masterarbeit}
% 						%
\title{Entwicklung eines Anlaufmodells für das Lean Start-up}
% 						%
\subtitle{\author{Rudolph Ribeiro Maier (330466)} 
}
						
\date{19.07.2018}
 % This is the titlepage
\maketitle
\includepdf[pages={1,2}]{latex_settings/onepager.pdf}			% Aufgabenstellung
% \includepdf[pages=2]{latex_settings/onepager.pdf}			% Aufgabenstellung
% \blankpage
%
%
%=====================================================
% Load Declaration of Authorship
\newpage
% \pagestyle{text}
\pagestyle{headings}

\thispagestyle{empty}
\section*{Eidesstattliche Erklärung}
\begin{verbatim}

\end{verbatim}
Hiermit erkl\"{a}re ich, % Rudolph Manuel Ribeiro Maier, 
dass ich die vorliegende Arbeit %, betitelt  \textit{\enquote{Design eines Energie Harvesting Moduls für autonome Energieversorgung von Bodensensoren}} 
selbstst\"{a}ndig und eigenh\"{a}ndig sowie ohne unerlaubte fremde Hilfe und ausschließlich unter Verwendung der aufgef\"{u}hrten Quellen und Hilfsmittel angefertigt habe. \\

Die selbstständige und eigenständige Anfertigung versichert an Eides statt:

\begin{verbatim}

\end{verbatim}
\hrulefill\\
\hspace*{2cm}Unterschrift
\hfill
Berlin, 19. Juli 2018

%=====================================================
%
% \include{acknowledgment}
% \include{summary}
% \newpage
% 
% \begin{abstract}
\chapter*{Kurzzusammenfassung}

\lipsum[1-5]
% \addcontentsline{toc}{chapter}{Kurzzusammenfassung}
% \end{abstract}

\tableofcontents
%
% %=====================================================
% Print Glossary
\newpage
% \pagestyle{glossary}
% \glsaddall
\printnoidxglossaries 							% see [glossariesbegin.pdf, p.16]
\listoffigures
\listoftables
% \addcontentsline{toc}{chapter}{Abkürzungsverzeichnis}
% %
% %=====================================================
% \pagestyle{fzvz}
% % \renewcommand{\chaptermark}[1]{\markboth{#1}{}}
% \newpage
% \chaptermark{Verzeichnis der Formelzeichen}	% für chaptername im fancyheader, nur hier nötig
% \addcontentsline{toc}{chapter}{Verzeichnis der Formelzeichen}
% \input{./tab/fvz1}
% \newpage
% \input{./tab/fvzg1}
% \newpage
% % \addcontentsline{toc}{chapter}{\listfigurename}
% \listoffigures
% \newpage
% \listoftables
% 
%%====================================================================================================
%             				 TEXTANFANG
%%====================================================================================================
%
% \linespread{}

% \pagestyle{text}
% Add Left Header
%
\setstretch{1.5}		% is equivalent to MS Word Zeilenabstand 1.5
%
% \include{0_work}
\newpage
\setcounter{page}{0}
\pagenumbering{arabic}
 % \section{Einführung}
\chapter{Einführung}
\section{Motivation \& Problemstellung}
Die produzierende Industrie findet sich heutzutage in einem zunehmend dynamischen Wettbewerbsumfeld wieder, welches vielschichtige Herausforderungen mit sich bringt \cite{Renner2012}. Die hauptsächlichen Herausforderungen liegen in steigenden Innovationsgeschwindigkeiten, kürzeren Produktlebenszyklen und einer höheren Variantenvielfalt \cite{Kuhn2002,Stauder2016}. Um dem durch die Globalisierung verstärkten Wettbewerb standzuhalten, müssen produzierende Unternehmen innovative Produkte und Dienstleistungen anbieten und sich zunehmend kundenorientiert aufstellen \cite{Surbier2014}. 
Eine zentrale Rolle wird hier dem Anlauf von Serienprodukten zugeschrieben. Aufgrund immer kürzer werdender Produktlebenszyklen rücken Kosten und Zeitaufwand in den Vordergrund \cite{Winkler2007}. So hat der Anlauf einen signifikanten Einfluss auf den wirtschaftlichen Erfolg des Produkts und die Time-to-Volume \cite{Klocke16}. Selbst ein um wenige Monate verschobener Verkaufsstart kann über Erfolg oder Misserfolg des Produkts entscheidend sein \cite{Schuh2008a}. Die Bedeutung der Serienanläufe findet bisher in der Wissenschaft keine angemessene Aufarbeitung \cite{Dyckhoff2012}. 

\section{Fokus der Arbeit}
Der Trend zur Konzentration auf Kernkompetenzen sorgt dafür, dass in großen Unternehmen immer mehr Wertschöpfungsanteile an Zulieferer abgegeben werden  \cite{Hilmola2015, Wildemann2008}. Der Gesamtanlauf setzt sich fortan aus vielen lokalen Einzelanläufen zusammen \cite{Zimolong2006}. Daraus resultieren höhere Abhängigkeiten zwischen größeren Unternehmen und den Zulieferern, die meist mittelständische Unternehmen sind. 

Die Abschlussarbeit soll sich im Speziellen mit dem Serienanlauf im KMU und SME als Zulieferer für größere Unternehmen beschäftigen, da hier erhebliches Verbesserungspotential erkennbar ist \cite[S.18]{Dombrowski2009a}. So gibt es in KMU meist keine Anlaufprozesse. Da es in KMU oft keine Stabsstellen gibt, werden Anläufe von den Mitarbeitern oft zusätzlich zum Tagesgeschäft gesteuert \cite{Dombrowski2009}. %TODO kein Zugriff auf Primärquelle D.Spath!! 
Mangelnde finanzielle und zeitliche Kapazitäten sowie fehlendes Know-how verhindern eine nachvollziehbare Dokumentation sowie proaktive Maßnahmen \cite{Zimolong2006,Dombrowski2009a}. 

Weiterhin soll untersucht werden, wie der Auftraggeber den Anlaufprozess des Lieferanten unterstützen kann. Größere Unternehmen verfügen in der Regel über mehr Ressourcen und teilweise eigene Anlaufprozesse. Im Zuge der Verlagerung der Wertschöpfungsanteile, gewinnt die Innovationskraft von Modul- und Systemlieferanten zunehmend an Bedeutung für den Erfolg eines Produktes \cite{Kuhn2002}. Ein erfolgreiches und effizientes Anlaufmanagement in KMU ist im Sinne der Entwicklung einer nachhaltigen Partnerschaft für Auftraggeber und Lieferant von großer Bedeutung. \textit{Wildemann} erkennt hier das Potenzial von Einspareffekten sowie Nutzung erheblicher Wettbewerbsvorteile auf beiden Seiten \cite{Wildemann2008}.

\textit{Dyckhoff} und \textit{Scholz} sind zu der Erkenntniss gekommen, dass das Thema weder in Industrie noch in der Wissenschaft hinreichend Beachtung findet \cite{Dyckhoff2012, Scholz2010}, weshalb hier keine zufriedenstellenden Ergebnisse zu erwarten sind.
Ziel der Arbeit ist, einen Überblick über den Stand der Forschung zu geben und einen Entwurf für ein Anlaufmodell zu entwickeln. 

\section{Herangehensweise}
Die Abschlussarbeit wird eine Literaturarbeit. In der Einführung erfolgt eine knappe Darstellung der zu behandelnden Themen Lean Startup / KMU und Anlaufmanagement. Im Hauptteil wird zunächst der Stand der Wissenschaft zum Thema Lean Startup skizziert. Den größeren Teil bildet eine umfassende Literaturanalyse zum Stand der Wissenschaft des Anlaufmanagements. Die Literaturrecherche erfolgt nach fest definierten Kriterien. Für die Literaturanalyse werden mit Hilfe des Tools \textit{Atlas.ti} alle relevanten Textstellen gecoded, d.h. identifiziert und nachvollziehbar dokumentiert. Anhand der  Ergebnisse wird anhand von möglichst vielen Quellen der Stand der Wissenschaft dargestellt. Im nächsten Abschnitt werden für das Lean Startup nicht berücksichtigte Anforderungen an das Anlaufmanagement ermittelt und daraus eine Art Anlaufmodell abgeleitet. 

Die Validierung der Ergebnisse erfolgt durch Zitierung der Quellen. Auf eine Validierung durch Experten, Fragebögen oder empirische Untersuchungen wird aufgrund des großen Umfangs verzichtet.
%
\section{Kontext}

\subsection{Lean Start-up}
\subsubsection*{Einführung}
Das Lean Start-up ist eine Businessmethode für dynamische Unternehmen oder Projekte, die hohen Risiken und Unsicherheiten ausgesetzt sind. 
Hauptziele der Methode sind kürzere Entwicklungszeiten, Einsparung von Kosten in der Entwicklungsphase und frühzeitiges Erkennen der Kundenbedürfnisse. 
Sie ist eine Antwort auf hoch dynamische Märkte, unbekannte Problemstellungen und Lösungen und hohen Risiken. Die Ursprünge liegen in den Denkweisen von Taiichi Ōno, W. Edwards Deming und Peter Drucker. 
2008 übertrug Eric Ries Lean Produktions Methoden auf hochtechnologie Startups und veröffentlichte 2011 die erstmals "Lean Startup" genannte Methode in seinem Buch. %TODO 2011 or 2008? 

%\subsection*{Definitionen}

\subsubsection*{Bestandteile}

\textit{1. Entwickeln einer Vision}. Die Vision dient als Grundlage für alle weiteren Handlungen. Aus ihr werden im nächsten Schritt Hypothesen abgeleitet. Anstatt einen aufwändigen Businessplan zu erstellen wird die Vision in einem Business Model Canvas definiert \cite{Blank2013}. Die Vision eines Lean Start-up zeichnet sich durch viele Freiheitsgrade und Unsicherheiten aus. 

\textit{2. Überführen der Vision hin zu Hypothesen}. Für jedes Element der im Business Model Canvas beschriebenen Vision werden Hypothesen abgeleitet. Die Hypothesen bilden die Freiheitsgrade und Unsicherheiten des BMC ab. Ziel ist, die Risiken durch spätere Beantwortung der Hypothesen zu minimieren. Nach Möglichkeit sollen die Hypothesn so formuliert werden, dass sie quantitativ beurteilt werden können. Die Hypothesen müssen wiederlegbar sein, um neue Erkenntnisse gewinnen zu können. 

\textit{3. Entwickeln von MVP Tests}. Ein minimal überlebensfähiges Produkt (\gls{mvp}, engl.: Minimum Viable Product) ist ein Werkzeug, mit dem man schnellstmöglich die Hypothesen am Kunden überprüfen kann \cite[93]{Ries2011}. Ziel ist zum einen den Build-Measure-Learn Zyklus zu beschleunigen, zum anderen die Lernrate in Bezug auf den Aufwand zu maximieren. So können frühzeitig nicht benötigte Funktionen und Produkteigenschaften erkannt und Zeit und Kosten gespart werden. Wenn die Entwicklung eines realen MVP zu aufwändig ist, kann ein Smoke Test eingesetzt werden. In einem Smoke Test wird das zukünftige Produkt in einem Video oder über eine Webseite vorgestellt.

\textit{4. Planung der Tests}. Bei der Durchführung der Tests kommt es darauf an, Kosten und Zeit zu minimieren. Daher werden zuerst Tests durchgeführt, die wenig kosten und hohe Risiken untersuchen. Beispielsweise ist eine Patentrecherche kostengünstig und kann frühzeitig sehr hohe Risiken aufdecken. Tests können nacheinander (seriell) oder gleichzeitig (parallel) durchgeführt werden. Bei parallelen Tests riskiert man, dass einzelne Tests überflüssig werden, profitiert jedoch von einem Zeitvorsprung gegenüber der seriellen Vorgehensweise. 

\textit{5. Interpretation der Ergebnisse}. Bei der Interpretation der Ergebnisse gibt es einige Fehlerquellen. Zum einen gibt es teilweise große Differenzen zwischen den geäußerten und reellen Kundenrückmeldungen. Zum anderen kann die Interpretation des Unternehmers durch eigene Wünsche oder Erwartungen verzerrt sein.

\textit{6. Reaktion}. Nach Auswertung der Ergebnisse sieht die LSU Methode eine Entscheidung zwischen drei Handlungsalternativen vor. \textit{Preserve}: Wenn die Tests die Hypothesen bestätigen wird die Strategie beibehalten. \textit{Pivot}: Wenn die Tests die Hypothesen wiederlegen oder neue Chancen aufzeigen, wird die Strategie angepasst. \textit{Perish}: Wenn die Tests die Hypothesn wiederlegen und der Unternehmer keine geeignete Strategie entwickeln kann, wird die Strategie verworfen. 

\textit{7. Skalierung und kontinuierliche Verbesserung}. Sobald alle relevanten Hypothesen bestätigt wurden, ist das Produkt auf den Markt abgestimmt. Jetzt kann massiv in Kundenakquise und Produktentwicklung investiert werden. Wichtig ist weiterhin, dass die Strategie fortwährend überprüft wird. Ein \textit{Pivot} ist auch nach der Skalierung bei größeren Änderungen sinnvoll. 

% \subsection*{Grenzen der Methodik}

\subsection{Anlaufmanagement}
Immer kürzere Produktlebenszyklen bei gleichzeitig höher werdenden Kundenwünschen und größerer Variantenvielfalt erhöhen die Komplexität und somit die Bedeutung des Serienanlaufs \cite{Kuhn2002,Schuh2004}. Die Risiken im Zusammenhang mit der Anlaufphase sind vielfältig. KUHN %TODO Biblatex Cite command for cap letter author in maintext
stellt fest, dass der Aufwand bis zum Erreichen einer stabilen Produktion oft unterschätzt wird. Infolgedessen kann es zum verspäteten Markteintritt sowie unzureichenden Kapazitäten und Qualitätsmängeln kommen \cite{Kuhn2002}. Um diesen Risiken entgegen zu wirken werden als übergeordnete Hauptziele für das Anlaufmanagement Beherrschung und zeitliche Verkürzung der Anlaufphasae genannt \cite{Kuhn2002, Schmitt2015}. 

Produktionsanläufe stellen auch deshalb eine große Herausforderung für Unternehmen dar, da sie hochkomplex sind und sich durch viele parallele und sequenzielle Teilprozesse auszeichnen. Sie sorgen zudem für eine starke Vernetzung der beteiligten Abteilungen innerhalb und außerhalb des Unternehmens \cite{Schuh2004}.


\subsubsection*{Definition}
In der Literatur existiert keine Einheitliche Definition des Begriffs Anlaufmanagement \cite[4]{Bischoff2007}. Selbst SCHMITT %TODO 
bemängelte 2015 ein fehlendes einheitliches Verständis der grundlegenden Begriffe des Produktionsanlaufs \cite[1]{Schmitt2015}. Vielmehr existieren unternehmensintern und teilweise auch projektspezifisch unterschiedliche Auffassungen über die Definition der Anlaufphase \cite[11]{Grosshenning2005}. KUHN %TODO
definierte das Anlaufmanagement wie folgt \cite[8]{Kuhn2002}: 
\begin{quotation}
Das Anlaufmanagement eines Serienproduktes umfasst alle Tätigkeiten und Maßnahmen zur Planung, Steuerung und Durchführung des Anlaufes mit den dazugehörigen Produktionssystemen, ab der Freigabe der Vorserie bis zum Erreichen einer geplanten Produktionsmenge, unter Einbeziehung vorgelagerter Prozesse und der nachgelagerten Prozesse im Sinne einer messbaren Eignung der Produkt- und Prozessreife.
\end{quotation}
SCHUH übernahm diese Auffassung \cite{Schuh08a} während RISSE und BISCHOFF den Beginn bereits nach der abgeschlossenen Produktentwicklung sehen \cite{Risse2002, Bischoff2007} (Freigabe Pflichtenheft).

Der Anwendungsbereich beschränkt sich nicht nur auf den Anlauf von neuen Produkten. Auch Modellderivate (Modellpflege), Varianten; neue Produktionssysteme, Fertigungsverfahren und Logistikprozesse stellen aus Perspektive des Managements ein Anlauf dar \cite[6]{Bischoff2007}. %TODO cite primara source LAICK/Warnecke/Aurich 2003

\subsubsection*{Lieferanten}

Ziele Werte Verhaltensnormen für Zusammenarbeit mit Lieferanten werden gemäß der Vision definiert Schmitt2015

Harmonisierung der Schnittstellen innerhalb der SC mit transparenten unternehmensübergreifende Strukturen Bischoff2007

Gemeinsame Informationsstrategie  Kuhn02

Frühe Einbindung und Integration der Lieferanten bedeutend für reibungslosen Anlauf Bischoff2007 S.28, Kuhn2002 S. 26

Einheitliche Datenbasis für den austausch von Informationen und Planungsdaten Kuhn02

Werkzeuge: 
  Lieferanten-Audits, KVP, PDCA Schuh08
  FMEA, QFD, Ishikawa, FTA Bischoff2007

\subsubsection*{Logistik}

Die Logistik beinhaltet die Koordinierung aller Material- und Informationsflüsse und Prozesse von Auftrag bis Auslieferung des Endprodukts. Die strategische Ebene beinhaltet die Entwicklung und Gestaltung der Wertschöpfungsnetzwerde und Prozesse nach logitsischen Prinzipien. Die operative Ebene beinhaltet die Lenkung und Kontrolle der Material- und Informationsflüsse und der dazugehörigen Prozesse. 
Hauptziele der Logistik ist, durch Gestaltung und Lenkung der logistischen Prozesse die Kundenbedürfnisse in den ökologischen, ökonomischen und sozialen Dimensionen optimal zu erfüllen \cite[28]{Schmitt2015}. 

Die Bedeutung der Logistik für die Anlaufphase ist durch die Globalisierung der Märkte, JiT-Konzepte und Reduzierung der Wertschöpfungstiefe gestiegen. Die Logistik hat zwei spezielle Funktionen in der Anlaufphase. Zum einen muss sie den Materialfluss der ersten Produkte bewerkstelligen. Zum anderen erprobt sie bereits Logistikprozesse für die Serie.
Durch den Querschnittscharakter der Logistik ist eine Abstimmung mit anderen Funktionsbereichen und der Logistik anderer Unterhehmen erforderlich \cite[1189]{Pfohl2000}.

\subsubsection*{Kooperationen}

\subsubsection*{Änderungen}

Definition: 
Technische Änderungen sind notwendige nachträgliche Anpassungen an bereits freigegebenen Entwicklungsständen \cite{Zanner2002}. Sie beinhalten immer eine Änderung der Dokumentation bzw. Datebnasis \cite[215]{Schuh2008}.
  %TODO cite Primärquelle Niemerg1997 - ZB Grimm-Zentrum 
  %Geschlossenes Außenmagazin 03a 
  %98 HA 8754 vorbestellt ins Campus Nord. 
Produktänderungen können in der Entwicklungs- und Konstruktionsphase bis zu 40\% der Gesamtressourcen beanspruchen \cite{Lindemann1998}
%TODO cite Lindemann1998 -> TUB QP624 77
Änderungsmanagement soll die Termintreue der Prozesse im Serienanlauf sicherstellen und die Durchlaufzeiten reduzieren \cite[216]{Schuh2008}. 

Ursachen: 
Auslöser für Änderungen können Gesetzesänderungen, interne Fehler, Qualitäts- und Sicherheitsprobleme, veränderte Kundenwünsche sowie eine veränderte Markt- und Wettbewerbssituation sein \cite{Zanner2002}. Auch treten Probleme oft erst dann in Erscheinung, wenn sie im Kontext der benachbarten Komponenten stehen \cite[24]{Kuhn2002}.

Konsequenzen: 
Änderungen bringen Konsequenzen mit sich. So führen sie zu steigendem Zeitdruck, einem erhöhten Personalaufwand in planerischen Abteilungen sowie können Kosten und Zeitverzögerungen aufgrund von Werkzeugänderungen entstehen \cite[24]{Kuhn2002}. 

Lösungsansatz und Bestandteile: 
Um den zeitlichen und finanziellen Aufwand gering zu halten, sollten Änderungen vermieden oder möglichst vorverlagert werden \cite{Schuh2008, Jania2004, Ass98}. 
% Schuh2008:215, Jania2004:69f, Ass98:107--131
%TODO Primärquelle Ass98 - TUB QP624 77
SCHUH teilt das Änderungsmanagement in Änderungsplanung, -ausführung und -absicherung ein \cite[217]{Schuh08}. 
%TODO cite original author Florian Rösch et. al.
LINDEMANN hingegen unterteilt das Thema detaillierter in vermeidung, Früherkennung, Problemanalyse, Lösungsfindung, Bewertung und Entscheidung. Die Erkenntisse werden mit Hilfe einer sog. Lernorientierten Auswertung im Sinne eines KVP ausgewertet \cite{Lindemann1998}. 
%TODO cite Primärquelle
%TODO KVP glossary, ausgewertet synonym

Enabler: 
Als Schlüsselrolle für erfolgreiches Änderungsmanagement wird oft die Kommunikation von Problemen und Änderungen innerhalb und über Unternehmensgrenzen hinweg genannt \cite{Kuhn2002, Schuh2008}.
% Kuhn2002:28+24, Schuh08:219
ZANNER betont die Bedeutung der Vertrauensverhältnisses für den Informationsaustausch und schlägt informelle standortübergreifende Treffen der Entwickler vor. Die Zuordnung eines Verantwortlichen Mitarbeiters für die Abwicklung einer Änderung soll helfen, die Schnittstellenprobleme bei der Arbeitsteiligen Arbeitsweise zu überwinden  \cite[42]{Zanner2002}.
Weiterhin werden eine einheitliche Terminologie \cite{Zanner2002} und Datenbasis sowie ein durchgängiges Versionsmanagement \cite{Kuhn2002} als Erfolgsfaktoren genannt. 
% Kuhn: Datenbasis S.25, Versionsmanagement S. 25

\subsection{Wissen}


% \cite*[prenote][postnote]{Schmitt2015}[extra] cite
% 
% \Cite*[prenote][postnote]{Schmitt2015}[extra] Cite
%  
% \footcite[prenote][postnote]{Schmitt2015}[extra] cite footcite
% 
% \Footcite[prenote][postnote]{Schmitt2015}[extra] cite Footcite
% 
% \textcite[prenote][postnote]{Schmitt2015}[extra] cite textcite
% 
% \citeauthor[prenote][postnote]{Schmitt2015}[extra] citeauthor
% 
% \citeauthor{Schmitt2015} Citeauthor
% 
% \citet{Schmitt2015} citet
% 
% \citep{Schmitt2015} citep
% 
% \autocites{Schmitt2015} autocites
% 
% \parencite{Schmitt2015} parencite		% \input{file} includes the commands and references
\chapter{Methodik}
\section{Anforderungen an das Anlaufmanagementmodell für das \gls{lsu}}
Damit das zu entwickelnde Modell den Serienanlauf im \gls{lsu} effektiv unterstützt, müssen zunächst einige Anforderungen formuliert werden. 

\textbf{Methodische Anforderungen}

Das zu entwickelnde Anlaufmanagementmodell muss sowohl horizontal als auch vertikal sinnvoll mit dem \gls{lsu} Ansatz kooperieren bzw. integriert werden. 
Die Beseitigung von Verschwendung sowie Anreize zur kontinuierlichen Verbesserung müssen strukturell im Modell verankert sein. 

Besondere Eigenschaften von (Lean-)Start-ups müssen berücksichtigt werden. Dazu zählen \gls{bspw} eine flache Hierarchie, eine kleine Anzahl an Mitarbeitern, Vorhandensein von Generalisten anstatt Spezialisten und Interdisziplinarität der Mitarbeiter und Aufgaben. Daraus werden folgende Forderungen abgeleitet: Eine kleine Anzahl an einfach anzuwendenden Methoden. Die Gestaltung von Ablauforganisation und Prozessen erfolgt mit geringem Detaillierungsgrad. An anderer Stelle soll jedoch mithilfe von Standardisierung die Komplexität der Lösungsalternativen beschränkt werden. Daraus folgt eine hohe Abstraktionsebene des Modells einerseits, andererseits jedoch ein hoher Detaillierungsgrad. 

Des weiteren wird eine Flexibilität des Modells gefordert. So muss das Modell, welches bereits in der Anfangsphase implementiert wird, bei schnellem Wachstum und stark veränderten Bedingungen weiterhin effektiv sein. Dazu zählen \gls{bspw} eine Skalierbarkeit der Methoden hinsichtlich Anzahl der Mitarbeiter sowie Mitarbeiterzuordnung von Kompetenzen und Aufgaben. 

\textbf{Technische Anforderungen}

Auch auf technischer Seite ist Flexibilität gefordert. Die Produktion bzw. der Anlauf müssen agil auf Stückzahlschwankungen reagieren können. Änderungen am Produkt oder die Einführung neuer Varianten müssen einfach und schnell mit hoher Qualität realisiert werden können. Analog dazu müssen Änderungen am Logistiksystem und Produktionslinie effizient durchgeführt werden können. 
Große Unsicherheiten sind ein inhärentes Merkmal des Serienanlaufs. Daher muss ein umfassendes Risikomanagement im Modell verankert sein. 


\section{Entwicklung des Grundgerüsts}

\section{Grundsätzliche Herangehensweise der Arbeit}



\chapter{Durchführung}

\section{Reichwald-2004}
% Kommentar: 
% Der Ansatz des Feedback-Loops passt schon gut zum LSU Konzept. Hervorzuheben ist z.B. der zielgerichtete Einsatz von Software, auch wenn nicht Abteilungen oder Prozesse zu überwinden sind. Betont werden soll jedenfalls, dass der Einsatz von Software hier nur unterstützend ist und den Menschen nicht von seiner Verantwortung entbindet. 
REICHWALD et al. untersucht das Projektmanagement im Feldanlauf und fokussiert sich dabei auf ein nachhaltiges Wissensmanagement \cite{Reichwald2004}. Ein durchgängiges und effizientes Wissensmanagement ist ein wesentlicher Bestandteil erfolgreicher Anläufe \cite{Kuhn2002}. % TODO check primary source Kuhn2002

REICHWALD et al. unterteilt den Wissensmanagementprozess in folgende Bestandteile: Identifizierung der Wissenslücken, Wissenserwerb und Wissensentwicklung, Wissensverteilung und Wissensbewahrung. Die ersten vier Bestandteile werden in Folgendem kurz vorgestellt. 

\subsection{Identifizierung der Wissenslücken}
Bereits vor Markteinführung muss bekannt sein, inwieweit das Produkt den Kundenerwartungen entspricht. Besonders die Produktmerkmale, die unterhalb der Kundenerwartungen liegen müssen identifiziert werden. Dazu eignen sich realitätsnahe Produkttests mit einer dem zukünftigem Kundenkreis entsprechenden Gruppe. Es sind geeignete Erhebungsinstrumente auszuwählen und weiterzuentwickeln. Um möglichst viele Qualitätsaspekte zu berücksichtigen müssen die Erfahrungen der Testpersonen über ein breites Spektrum erfasst werden. Darüber hinaus müssen die Ergebnisse einfach auszuwerten sein um eine schnelle Berücksichtigung zu gewährleisten. Dazu eignen sich z.B. standardisierte Fragebögen oder kurze mündliche Befragungen. 

\subsection{Wissenserwerb, -entwicklung und -verteilung}
Nach erfolgter Produkttests werden die Ergebnisse ausgewertet. Die einzelnen Ergebnisse werden in ein sogenanntes E-Workflowsystem eingepflegt, welches eine Art \gls{erp} System darstellt.
Dieses System ist in der Lage das gesammelte Faktenwissen entlang der gesamten Prozesskette bereitzustellen. Die bei der Auswertung der Tests gewonnenen Schwerpunkte bilden die Handlungsfelder des Anlaufteams. Mit Hilfe des E-Workflowsystems werden den jeweiligen Handlungsfeldern Maßnahmen und Zuständigkeiten sowie Umsetzungstermine zugeteilt. 
Nach der Umsetzung der Maßnahmen muss die Wirksamkeit möglichst durch die gleichen Personen bestätigt werden, die im Vorfeld die Handlungsfelder aufgezeigt haben. Wird die Wirksamkeit bestätigt, ist das Handlungsfeld erfolgreich abgeschlossen. 
Sind Erkenntnisse des laufenden Projekts auch für zukünftige Projekte von Bedeutung, so sollten sie im E-Workflowsystem gesondert gekennzeichnet und in zukünftigen Entwicklungsprozessen eingegliedert werden. 


\subsection{Kurzzusammenfassung}

Ein durchgängiges und effizientes Wissensmanagement ist ein wesentlicher Bestandteil erfolgreicher Anläufe. Dabei werden frühzeitig Kundenrückmeldungen zur Produktverbesserung ausgewertet und die Arbeit mit Softwaresystemen unterstützt. 

\section{Harjes-2004}

HARJES et al. untersucht das Anlaufmanagement mit besonderer Berücksichtigung des Produktentstehungprozesses \cite{Harjes2004}. 

\subsection{Robuste Produktionssysteme}

Höhere Variantenvielfalt und Individualisierungswünsche der Kunden stellen hohe Anforderungen an Fertigungs- und Montagelinien. Zunächst ist eine Standardisierung erforderlich. Produktionssysteme sollten einfach und übertragbar gestaltet werden. Daraus erfolgt eine erhöhte Flexibilität bei Integration neuer Baureihen und Komponenten, Änderungen können somit reibungsloser implementiert werden. Um Auswirkungen vom Prozess oder Produkt auf das Produktionssystem frühzeitig bewerten zu können, sind Prozess- und Produktdaten standardisiert zu verknüpfen und stets aktuell zu halten. 

\subsection{Produktdatenmodell}
Die stetige Reduzierung der eigenen Wertschöpfungstiefe erfordert eine hohe Transparenz bzgl. der Produktdaten. Dies erfolgt mit dem Aufbau digitaler Produktdatenmodelle, welche stets einen echten, plausiblen und aktuellen Datenstand aufweisen müssen. 
Integrierte Produktdatenmodelle (\gls{ipdm})
verknüpfen Produkt- und Prozessdatenmodelle. Damit bekommen Änderungen mehrdimensionalen Charakter, betroffene Komponenten können identifiziert werden und die Folgen lassen sich simulieren und bewerten. 
Weiterführend wird die digitale Fabrik genannt, die die digitale Planung einer Fertigungsfabrik mit Integration aller Produkt- und Prozessdaten beschreibt. 

\subsection{Kurzzusammenfassung}
Robuste Profuktionssysteme reagieren agil auf Änderungen und können flexibel erweitert werden. Digitale Produktdatenmodelle sorgen Unternehmensübergreifend und -intern für erhöhte Transparenz und bessere Folgenabschätzung von Änderungen. 

\section{Straub-2006}

STRAUB et al. verfolgt die Vision der Umstellung der Produktion von \gls{sop} 
auf Kammlinie an einem Wochenende. Im Fokus seiner Arbeit steht die schnelle und richtige Reaktion auf ungeplante Störungen im Anlauf \cite{Straub2006}. Die bisher eingesetzte präventive Methode der digitalen Fabrik erhöht zwar signifikant den Reifegrad der Planung, bietet jedoch keine Antwort aud verbleibende ungeplante Störungen. STRAUB beschreibt drei Säulen zur Realisierung kürzerer Anläufe: Einsatz von Anlaufteams, die organisatorische Einbindung der Teams in die Organisation und der Einsatz eines Methodenbaukastens. Letzterer ist für das \gls{lsu} von Beteudung. 

\subsection{Methodenbaukasten}

Grundgedanke des Methodenbaukastens ist der Einsatz moderner Methoden, Werkzeuge und Standards.
Zum einen wird eine erhöhte Effizienz und Transparenz bewirkt. Zum anderen wird eine objektive Bewertung von Situationen und damit ein einheitliches Verständnis erreicht, was insbesondere die Zusammenarbeit mit jüngeren und unerfahrenen Mitarbeitern erleichert. 
Des weiteren wird der Einsatz einer Scorecard empfohlen. %TODO Bild 6 in Anhang oder hier. 
Zunächst werden quantifizierbare Anlaufindikatoren definiert. Mit Hilfe der Scorecard werden die wichtigsten Anlaufindikatoren kontinuierlich überwacht und Abweichungen vom Soll Wert werden schnell erkannt. Es folgt eine systematische Ursachenanalyse. So kann eine schnelle und zielgerichtete Reaktion gewährleistet werden. 

\section{Berg-2006}

\section{Quasdorff-2016 - Lean Management und Digitale Fabrik}

QUASDORFF et al. behandelt die Schnittmengen von Lean Management und der Digitalen Fabrik \cite{Quasdorff2016}. 

Die Digitale Fabrik umfasst die Abbildung und Simulation von Produkt, Prozess und Ressourcen in einem Informationssystem. Während für die Digitale Fabrik die Datenbasis für den Erfolg ausschlaggebend ist, muss die Lean Philosophie aktiv im Unternehmen gelebt werden. Beim gleichzeitigen Einsatz beider Methoden sind große Synergieeffekte zu erwarten. 

Die Digitale Fabrik unterstützt die Vermeidung von Muda (Verschwendung), Mura (Unausgeglichenheit) und Muri (Überbeanspruchung). Durch die zunehmende Digitalisierung (Industrie 4.0 bzw. \gls{iot}) wächst die Bedeutung von Quellen der Verschwendung im Bereich der Informationstechnik und der Datenverarbeitung. 

Bei der Gestaltung der Digitalen Fabrik müssen einige Aspekte beachtet werden. So ist die konsequente Anwendung von Lean Prinzipien Voraussetzung für die Digitale Fabrik. Schlanke Prozesse und deren Standardisierung sorgen dafür, dass die Komplexität der Modelle der Digitalen Fabrik beherrschbar wird. 
Verbesserungen an ineffizienten Prozessen sollten am Prozess als solchen ansetzen anstatt verbesserte Technologie einzusetzen. % TODO cite [8]

Erfolgsfaktoren sind eine hohe Detailtreue und Datenqualität. Das Modell sollte zu jedem Zeitpunkt der Realität entsprechen. Dennoch sollte vor einer Änderung der Ist-Zustand mit dem Dokumentationszustand verglichen werden. 

Abschließend ist zu bemerken, dass der Einsatz der Digitalen Fabrik den Gang in den Shop Floor nicht ersetzen sondern nur unterstützen kann. 

\textbf{Einordnung:} Der Artikel liefert Ansätze für die Gestaltung der DF im LSU. Dabei muss stets auf den Angemessenen Einsatz der Methoden geachtet werden. Ggf. sollten nur einige kritische Elemente Einzug in die DF erhalten. Dennoch muss das Modell zu jedem Zeitpunkt aktuell und plausibel sein. Auch die Detailtreue muss im Anfangsstadium nicht zwingend maximiert werden sondern den Zweck erfüllen einen hohen Reifegrad in der frühen Planungsphase zu erreichen und viele Entscheidungen möglichst früh treffen zu können. 

\section{Schwarz-2017 - Reifegradmodell für Lean Production}

SCHWARZ et al. entwickelt ein Reifegradmodell zur Bewertung des Implementierungsfortschritts von Lean Production im Unternehmen \cite{Schwarz2017}. Einfache Befragungen eignen sich aufgrund der Komplexität nicht zur Bewertung. 

\subsection{Bestandteile}
Das von SCHWARZ entwickelte Reifegradmodell erfasst den Fortschritt in den zwei Dimensionen Methodenkompetenz und Unternehmenskultur. 
Methodenkompetenz beschreibt die Fähigkeit eines produzierenden Unternehmens die Prinzipien der Lean Philosophie durch Anwendung spezifischer Methoden systematisch und gezielt im Produktionssystem umzusetzen. 
Die folgenden fünf Lean Prinzipien dienen als Grundlage für das Modell: Kundennutzen, Wertstrom, Fluss, Pull und Perfektion. 
Es existieren zahlreiche Methoden die jeweils ein oder mehrere Prinzipien umsetzen. 

Voraussetzung für den nachhaltigen Einsatz von Lean Prinzipien ist das aktive Leben der Ideen sowie die Verankerung in der Unternehmenskultur. % TODO Cite [7] Baumgärtner (nicht verfügbar), alt. Quelle suchen
Die Unternehmenskultur wird mit Hilfe folgender Aspekte beschrieben: 
\begin{itemize}
 \item Grundlegende Annahmen und Überzeugungen
\item Implizite und explizite Werte 
\item Mittel zur Verwirklichung dieser Werte
\item die Außenwirkung.
\end{itemize}
Die Ausprägungen der Aspekte sind Voraussetzungen für eine nachhaltige und langfristige Anwendung der Methoden durch die Mitarbeiter und somit für die Implementierung der Lean Prinzipien. 

\textbf{Gestaltung}

Die Bewertung der Reifegrade in den zwei Dimensionen erfolgt in sechs Stufen. % TODO siehe Grafik. 


Für die Dimension Methodenkompetenz erfolgt die Bewertung mit Hilfe von 13 Fragen. Abgefragt werden Eigenschaften, die auf den Implementierungsgrad abzielen. \Gls{bspw} wird die Qualifizierung der Mitarbeiter und Führungskräfte oder der Einfluss der Kundenforderungen auf die Produktion abgefragt. 
Auch wird der Einsatz bestimmter Methoden Reifegraden zugeordnet. So wird der Einsatz von \gls{smed} der Stufe 2 (``Wissend'') und der Einsatz von \gls{heijunka} der Stufe 4 (``Etabliert/Gesichert'') zugeordnet. 

Für die Dimension Unternehmenskultur wird anhand von sieben Fragen ermittelt, inwieweit die Lean Production in der Unternehmenskultur verankert ist. 

\Gls{bspw} wird abgefragt, inwieweit die 5 Lean Prinzipien in der Unternehmensphilosophie verankert, kommuniziert und verstanden ist (``Annahme und Überzeugung'') oder inwieweit die Implementierung von den Führungskräften unterstützt wird (``Werte''). 

\begin{figure}[!ht] 
    \begin{minipage}{0.3\linewidth} 
    \begin{center}
      \includegraphics[scale=.27]{./img/schwarz2017:rg.png}
 % schwarz2017:rg.png: 0x0 pixel, 300dpi, 0.00x0.00 cm, bb=
    \end{center}
      \caption{Die sechs Reifegradstufen \cite{Schwarz2017}}\label{fig:links} 
    \end{minipage} 
    \hfill 
    \begin{minipage}{0.6\linewidth} 
 \includegraphics[scale=.3]{./img/schwarz2017:portfolio.png}
 % schwarz2017:portfolio.png: 0x0 pixel, 300dpi, 0.00x0.00 cm, bb=
    \caption{Portfoliodarstellung mit zwei Dimensionen \cite{Schwarz2017}}\label{fig:rechts} 
    \end{minipage} 
  \end{figure} 

\subsection{Durchführung}
Die tatsächliche Durchführung teilt sich auf in: Befragung, Detaildarstellung der Ergebnisse aller 20 Themen, Analyse und Ableitung von Verbesserungspotentialen und Ausarbeitung eines Maßnahmeplans für die Realisierung. 
%
% \textbf{Befragung}
Zunächst erfolgt die Befragung bei der eine Einschätzung zu jedem der 20 Themen stattfindet. Dazu werden \gls{bspw} Führungskräfte und ggf. externe Personen befragt. 

% \textbf{Auswertung}
Die Auswertung erfolgt in drei Schritten. Zunächst werden die arithmetischen Mittel der Antworten für jedes Thema ermittelt. Große Abweichungen untereinander deuten auf eine unausgewogene Entwicklung hin und es besteht punktueller Nachholbedarf. Im nächsten Schritt wird der Mittelwert über alle 20 Themen ermittelt. Dieser stellt den aktuellen Reifegrad des Unternehmens in Bezug auf Lean Production insgesamt dar. Im dritten Schritt werden die Mittelwerte der zwei Dimensionen miteinander in Bezug gesetzt. % TODO Grafik

Eine Abweichung größer als eine Reifegradstufe wird als kritisch bewertet und deutet auf eine einseitige Implementierung hin. Abhilfe schafft hier die Anpassung des Ressourceneinsatzes. 

\textbf{Einordnung:} 
Die Überprüfung der Dimension Methodenkompetenz kann das Lean Start-up dabei unterstützen den Einsatz geeigneter Methoden zu steuern. 
Die Überprüfung der Dimension Unternehmenskultur hingegen ist in den frühen Phasen des Lean Start-up wenig sinnvoll. Lean Start-ups bestehen üblicherweise aus kleinen Teams mit flachen Hierarchien, und Identifikation und Motivation ist bei allen Mitarbeitern sehr ausgeprägt.


\section{Christensen-2016 - Lean Application to Manufacturing Ramp-up}

CHRISTENSEN et al. untersucht, inwieweit Lean Prinzipien und Methoden auf den Produktionsanlauf übertragbar sind \cite{Christensen2016}. Schwerpunkte der Arbeit sind Qualität und Lernprozesse. Die Ergebnisse der Arbeit werden in einem Framework zusammengefasst, der in Folgendem skizziert wird. 

\subsection{Qualität}
Qualität ist ein wichtiger Indikator für die Marktreife des Produkts. Ein hohes Qualitätsniveau soll in kürzester Zeit erreicht werden, was bei immer kürzeren Produktlebenszyklen eine hohe Herausforderung darstellt. 

Die Mitarbeiter sollen dazu motiviert werden, mit Hilfe von Versuchen den kontinuierlichen Verbesserungsprozess zu unterstützen. Ferner sollen Qualitätsprobleme möglichst früh im Anlaufprozess identifiziert und beseitigt werden. 

\subsection{Zeit}
Schnellere Produktionsprozesse und kürzere Taktzeiten erhöhen den Einfluss menschlicher und technischer Fehler. 

Vermeidung verschwenderischer Aktivitäten und Fokussierung auf Wertschöpfende Tätigkeiten ermöglichen eine höhere Qualität bei gleichzeitiger Zeitersparnis. 

\subsection{Kommunikation}
Mangelnde Kommunikation stellt einen erheblichen Störfaktor im Serienanlauf dar. 

Standardisierte Kommunikation und Informationsflüsse in Kombination mit Lean Techniken wie z.B. \gls{obeya} Meetings überwinden das Abteilungsdenken. 

\subsection{Lieferanten}
Die Leistung einer Lieferkette zeigt sich erst im Zusammenspiel mit allen Komponenten und Lieferanten. 
Bevor Änderungen in der Lieferkette durchgeführt werden, müssen die verantwortlichen Mitarbeiter die konsequente Ausrichtung nach Lean Prinzipien gewährleisten. 

\subsection{Qualifizierung u. Personal}
Eine feste Zuordnung von Verantwortlichkeiten kann die Geschwindigkeit und Qualität von Entscheidungen erhöhen. 
Feste Zuordnung von Verantwortlichkeiten sollte bis in die unterste Ebene auf den Shopfloor reichen.

\subsection{Produktdaten}
Detaillierte und transparente Produktdaten wirken Fehlern entgegen, die aus mangelhafter Dokumentation resultieren. 
Für Produktdaten und Arbeitsanweisungen sollten Standards erarbeitet und umgesetzt werden. Ein verankerter Lernprozess unterstützt den kontinuierlichen Verbesserungsprozess und verringert die Anzahl unvorhersehbarer Störungen. 

\subsection{Engpässe}
Engpässe beeinträchtigen die Anlaufperformance und sind schwer vorherzusagen. 

Mit Hilfe systematischer Identifikation und Beseitigung verschwenderischer Aktivitäten können Engpässe vermieden und die Profuktionsleistung geglättet werden. 


\chapter{Ableitung Modell}\label{sec:ableitung}

\section{Regelung}

Die Regelung umfasst Maßnahmen während des Serienanlaufs. Dazu gehören zum einen die Beherrschung von Komplexität und Unsicherheit und zum anderen Strategien zum optimalen Umgang mit Störungen. 

Zur Beherrschung von Unsicherheiten eignet sich die Abbildung komplexer Systeme in abstrahierten Modellen. Die Heuristik hilft bei Zeitdruck und mangelhafter Datenbasis gute Entscheidungen zu treffen. In komplexen Systemen können Muster erkannt und Regeln abgeleitet werden, ohne die Funktion oder die Ursachen zu verstehen. 

Zum optimalen Umgang mit Störungen soll im Vorfeld ein Methodenbaukasten entwickelt werden. Dazu gehört eine Auswahl an modernen Methoden und Werkzeugen sowie die Definition von Standards. Zur objektiven Bewertung von Situationen sollen quantifizierbare Anlaufindikatoren definiert werden. Werden sie mit Hilfe einer Scorecard laufend überwacht, kann eine schnelle und zielgerichtete Reaktion stattfinden. 

\section{Produktentwicklung}

Die Produktentwicklung soll Fehler im Produkt und Prozess frühzeitig vermeiden. Möglichst früh soll eine hohe Produkt- und Prozessreife erreicht werden. 

Beim Aufbau digitaler Produktdatenmodelle soll stets mit aktuellen, echten und plausiblen Daten gearbeitet werden. Produkt- und Prozessdaten sollen mit Hilfe sog. \gls{ipdm}-Systeme verknüpft werden. Somit bekommen Änderungen mehrdimensionalen Charakter. Die Änderungsfolgen können besser simuliert und bewertet werden. 

Das Set-based Engineering ermöglicht frühzeitiges Erreichen hoher Reifegrade. Es werden parallel verschiedene Lösungsentwürfe entwickelt. Das zu Beginn kostenintensive Verfahren zahlt sich später durch schnellere Marktreife und Kostenersparnisse in Serienanlauf und -produktion aus. 

\section{Wissen}

Nachhaltiges Wissensmanagement ermöglicht bessere Nutzung firmeninternen Wissens sowie den gezielten Einsatz von Lernprozessen. 

Wissenslücken, insbesondere mit Hinblick auf die Erfüllung der Kundenwünsche, müssen frühzeitig identifiziert werden. Dazu eignen sich Produkttests an Testpersonen. Geeignete Erhebungsinstrumente sind hierfür zu entwickeln. Die gewonnenen Erkenntnisse müssen entlang der gesamten Prozesskette zur Verfügung gestellt werden. Dazu eignet sich ein sog. E-Workflowsystem, welches eine Art \gls{erp}-System darstellt. 

Wissen kann auch gezielt durch Lernprozesse generiert werden. Die Prototypen- und Nullserienphase sollen als Versuchsfeld betrachtet werden. Geplante Versuche in diesen Phasen führen zu Erkenntisgewinn und letztlich zu Verkürzung der Anlaufzeit. 

\section{Qualität}

Gezielter Methodeneinsatz ermöglicht frühzeitige und effektive Analyse von Ursache-Wirkungs %TODO groß oder kleinschreibung? Siehe auch 3.6 QM : Zink2010  
Zusammenhängen bei Qualitätsproblemen. 

Das weit verbreitete Ishikawa-Diagramm wird an die Bedürfnisse des Serienanlaufs angepasst. Dazu wurde der sog. Hypothesen-Suchraum entwickelt. Dieser spannt sich über die bekannten und ggf. anzupassenden Ishikawa-Dimensionen (Methode, Material, Mensch, Messung, Milieu, Maschine) und weiterhin durch die relevanten Phasen im Produktlebenszyklus auf. Nachdem potentielle Ursachen  von Qualitätsproblemen identifiziert wurden, werden sie einzelnen Phasen zugeordnet. Im zweiten Schritt werden mit Hilfe des sog. Wirkgefüges Zusammenhänge zwischen den Einflussfaktoren erarbeitet. Dabei bildet das Wirkgefüge die inhaltlichen Wechselwirkungen zwischen den Ursachen übersichtlich ab und lässt Rückschlüsse auf die Hauptursachen zu.

\section{Risiken}

Mit der Adaption von klassischen Risikomanagement-Methoden auf den Serienanlauf können anlaufrobuste Produktionssysteme ermöglicht werden. 

Zur systematischen Erfassung von Risiken werden strukturierte Risikobereiche gebildet. Risiken lassen sich zu Produktionssystemelementen zuordnen. Anschließend werden sie unter dem Aspekt der Auswirkungen klassifiziert. Es folgt eine getrennte Betrachtung von Leistungs- und Kostenrisiken. 
Während die Kostenrisiken direkt quantifiziert werden können, müssen Leistungsrisiken im Rahmen der Szenarioberechnung bewertet werden (siehe Abschnitt \ref{sec:wildemannszenarien}).
Mit Hilfe der Szenarioberechnung wird ein Risikoportfolio erarbeitet. Vier Handlungsstrategien welche in Abschnitt  \ref{sec:wildemannszenarien} beschrieben wurden, lassen sich differenziert den einzelnen Risiken im Portfolio zuordnen.

\section{Änderungen}

Ein aglies Änderungsmanagement ermöglicht eine zielgerichtete Analyse und Durchführung von technischen Änderungen an Produkt und Prozessen. Die Handlungsempfehlung untergliedert sich in Voraussetzungen für ein optimales Änderungsmanagement und in die Durchführung. % TODO Formulierung...

Als Voraussetzungen wird die Definition von Phasen (Prototyp, Übergang, Vorserie/Serie) genannt. Für die Phasen sollen jeweils Ausprägungen in den Dimensionen Agilität, Freiheitsgrade, Produktionsplanung und Dokumentation definiert werden. \Gls{bspw} könnte für die Prototyp-Phase eine hohe Agilität und viele Freiheitsgrade gering ausgeprägter Produktionsplanung und Dokumentation gegenüber stehen. Gestaltungsempfehlungen für die Designemelemte sind in Abschnitt \ref{sec:schuh2017} detailliert beschrieben. 
Die Durchführung soll eine effiziente Abwicklung der Änderungsbearbeitung unter Berücksichtigung der Wechselwirkungen und Interdependenzen ermöglichen. Zentrale Aspekte bilden die Gruppierung von Änderungen um Aufgaben zu parallelisieren sowie methodische Unterstützung zur Festlegung einer geeigneten Bearbeitungsreihenfolge. 

\section{Produktionssysteme}

Produktionssysteme sollen robust gegenüber Störungen und flexibel erweiterbar sein. 

Werden Produktionssysteme einfach und übertragbar gestaltet, kann eine erhöhte Flexibilität bei der Integration neuer Baureihen und Komponenten erreicht werden. Standardisierte Verknüpfung von Prozess- und Produktdaten ermöglicht eine frühzeitige Bewertung von Produkt oder Prozess auf das Produktionssystem. 

Um schnell auf Änderungen reagieren zu können, muss der Planungsaufwand reduziert werden. Zentrale Aspekte bilden hier die Einführung und Nutzung von Standards, welche die Komplexität (Auswahl) verringern. Für Maschinen und Anlagen könnte \gls{bspw} eine Standardisierung mit Hilfe von Bibliotheken erfolgen. Routinetätigkeiten wie z.B. Auswertungen sollten automatisiert erfolgen. 
\chapter{Fazit \& Ausblick}\label{sec:fazit}

\section{Kritische Würdigung}

\section{Ausblick}

In diesem Abschnitt sollen die Konsequenzen bzw. Implikationen für die Wissenschaft und Wirtschaft sowie ein Ausblick gegeben werden. 

\subsection*{Implikationen für die Wissenschaft}
Bereits in der Einführung wurde darauf hingewiesen, dass eine Validierung der Ergebnisse im Rahmen dieser Arbeit nicht stattfinden wird. Daher sind die hier gewonnenen Erkenntnisse (AM-Modell, Umsetzungsleitfaden) zur Zeit als ein Vorschlag für eine Best Practice zu betrachten. Eine Validierung kann durch eine zweite Person % TODO Kann ich das so schreiben -> Robert
oder aber durch empirische Bestätigung in der Industrie erfolgen. 

Für die weitere Forschung wurden vier verschiedene Ansätze identifiziert: 

\begin{enumerate}
 \item Die zu untersuchenden Aspekte des Grundgerüsts können erweitert werden. Zunächst können die in dieser Arbeit vorgeschlagenen aber nicht untersuchten Aspekte Kooperationen, Lieferanten und Logistik %TODO check
 hinzugezogen werden. Möglich ist auch, dass neue Aspekte identifiziert oder hier erarbeitete Aspekte weg gelassen werden. 
\item Die Forschung kann quantitativ durch eine umfangreichere Literaturrecherche und -auswertung ergänzt werden. Dabei wird die Anzahl relevanter Quellen und Lösungsvorschläge erhöht. 
\item Die Forschung kann in der Abstraktionsebene variiert werden. Es können detailliertere Handlungsempfehlungen entwickelt werden. Denkbar ist auch die Erstellung konkreter Umsetzungsvorschläge für diverse Anwendungs- bzw. Unternehmensszenarien. Dadurch kann der Implementierungsaufwand im Unternehmen erheblich reduziert werden. Dazu müssen zunächst Zieltypen identifiziert werden, die mögliche Anwendungsszenarien beschreiben. Anschließend werden für jeden Zieltyp konkrete Handlungsempfehlungen entwickelt. 
\item Erkenntnisse aus der Industrie können hinzugezogen werden. Denkbar ist eine Erhebung von Erfahrung aus (Lean) Start-ups und \gls{kmu}, die sich bereits mit Serienanläufen beschäftigt haben. Für die Erhebung eignen sich \gls{bspw} Fragebögen, Interviews oder Veröffentlichungen (Whitepaper, Präsentationen). 
\end{enumerate}



\subsection*{Implikationen für die Wirtschaft}

Für eine erfolgreiche Umsetzung der erarbeiteten Vorschäge im Unternehmen sind folgende Voraussetzungen identifiziert worden: 
\begin{enumerate}
 \item \textbf{Kompetenzen: } Die verantwortlichen Mitarbeiter müssen ein Grundverständnis für die Denkweise, Begriffe und Methoden (z.B.: QFD, Ishikawa-Diagramm) des Qualitätsmanagements vorweisen. 
 \item \textbf{Ressourcen: } Für die Umsetzung müssen genügend Ressourcen (Personal, Zeit, Geld) zum richtigen Zeitpunkt vorhanden sein. Zu Beginn ist eine strategische Planung und Implementierung der Vorschäge erforderlich. Im weiteren Verlauf muss der operative Betrieb gewährleistet sein. 
 \item \textbf{Motivation: } Für den Einsatz der notwendigen Ressourcen insbesondere zu Beginn einer Produktplanung ist starke Motivation erforderlich. Die Motivation hängt u.a. vom Vermögen der Mitarbeiter ab, das Anlaufmanagement als kritisches Handlungsfeld zu identifizieren. 
\end{enumerate}

Die Bedeutung der aufgeführten Punkte wird deutlich, wenn der Fall eintritt, dass das Lean Start-up über keine eigene QM-Abteilung oder QM-Mitarbeiter verfügt. 
\newpage
\singlespacing
% \newpage
% \
\printbibliography %[title={Literaturverzeichnis}]
%
%
\newpage
\appendix
% \chapter{Anlagenverzeichnis}

\section{Dombrowski-2011a - Lean Ramp-up. Handlungs- und Gestaltungsfelder}
\subsection{Die Handlungsfelder im Lean Ramp-up}\label{appendix:dom11a:hf}

\textbf{Produktentwicklung und Konstruktion}
umfassen alle Aufgaben, die sich mit
dem Konzipieren, Entwerfen, Ausar
beiten und Erproben eines Produkts
beschäftigen. Als Ergebnis resultie
ren Zeichnungen, Stücklisten und an
dere Produktdokumentationen.

\textbf{Fertigungs- und Montagemittel} umfas
sen alle Aufgaben, die sich mit der Be
darfsplanung, Auswahl, Beschaffung,
Herstellung, Einrichtung, Program
mierung und Inbetriebnahme von Ma
schinen, Anlagen, Werkzeugen und
Vorrichtungen beschäftigen. Dazu ge
hören auch die Wartung und Instand
haltung. Als Ergebnis resultiert ein
Fertigungs- und Montagekonzept.

\textbf{Fertigungs- und Montageprozesse} um
fassen alle Aufgaben, die sich mit der
Festlegung der Arbeitsabläufe zur
Herstellung eines Produkts in der
Fertigung und Montage beschäftigen.
Es werden u. a. Reihenfolgen und
Vorgabezeiten bestimmt sowie Pro
duktionsmittel zugeordnet. Als Er
gebnis resultieren Arbeitspläne, in
denen alle Informationen dokumen
tiert sind.

\textbf{Personal- und Arbeitsorganisation} um
fasst alle Aufgaben, die sich mit der
Bedarfs- und Einsatzplanung, Be
schaffung, Entwicklung und Freiset
zung von Personal sowie mit der Ge
staltung einer arbeitsgerechten und
bestmöglichen Zusammenarbeit von
Mensch und Technik beschäftigen.
Als Ergebnis resultieren zum Beispiel
Personaleinsatzpläne,
 Arbeitszeitund Entgeltsysteme.

\textbf{Produktionsplanung und -steuerung}
(PPS) umfasst alle Aufgaben, die sich
mit der Festlegung, Veranlassung,
Überwachung und Sicherung des Pro
duktionsprogramms nach Art und
Menge unter Berücksichtigung von
Terminen und Kapazitäten beschäfti
gen. Als Ergebnis resultiert ein Pro
duktionsplan mit Bedarfsmengen und
-terminen für Zukauf- und Eigenferti
gungsteile.

 \textbf{Einkaufs- und Dispositionsprozesse}
umfassen alle Aufgaben, die sich mit
der kostenoptimalen strategischen
und operativen Beschaffung von Zu
kaufteilen, Handelswaren, Betriebs
mitteln und Dienstleistungen von ei
nem Lieferanten beschäftigen. Als Er
gebnis resultieren zum Beispiel Sour
cingstrategien, Verträge mit Lieferan
ten und verfügbare Lagerbestände.


 \textbf{Logistikprozesse und Logistikmittel}
umfassen alle Aufgaben, die sich mit
der Festlegung von effizienten inner
und außerbetrieblichen Transporten
bzw. Materialflüssen und der Bereit
stellung von Gütern beschäftigen.
Außerdem werden Logistikmittel, wie
z. B. Lager- und Transportmittel be
stimmt. Als Ergebnis resultieren sog.
Logistiksysteme.

 \textbf{Gebäude, Layout und Arbeitsplätze}
umfassen alle fabrikplanerischen Auf
gaben, die sich mit der Festlegung, op
timalen Auslegung und Realisierung
der Produktionsstätten beschäftigen.
Der Umfang reicht dabei von der Um
gestaltung einzelner Arbeitsplätze bis
hin zur Errichtung neuer Gebäude.
Als Ergebnis resultieren eingerichtete
Arbeitsplätze, Flächen und Gebäude.


 \textbf{Qualitätsmanagement und Qualitäts
mittel} umfassen alle Aufgaben, die
sich mit der Planung, Lenkung, Prü
fung, Sicherung und Verbesserung
der Qualitätsmerkmale von Produk
ten, Prozessen und Leistungen be
schäftigen. Außerdem werden Quali
tätsmittel, wie z. B. Prüf- und Mess
mittel bestimmt. Als Ergebnis resul
tieren beispielsweise Arbeits- und
Prüfanweisungen.

\textbf{Informationsprozesse und -systeme}
umfassen alle Aufgaben, die sich mit
der Beschaffung, Verarbeitung, Über
tragung und Speicherung von Infor
mationen zur Integration und ziel
orientierten Steuerung aller operati
ven Prozesse beschäftigen. Als Ergeb
nis resultieren zum Beispiel Systeme
zur Betriebsdatenerfassung (BDE).


\subsection{Die Gestaltungsfelder im Lean Ramp-up}\label{appendix:dom11a:gf}
\textbf{Integration und Kooperation} umfassen
alle Methoden und Werkzeuge, die
fachbereichs-, phasen-, technologieund unternehmensübergreifend zur
Synchronisierung von Produkt- und
Produktionsentwicklung beitragen.
Dazu wird eine simultane, interdisziplinäre und partnerschaftliche Zusammenarbeit angestrebt. Ziel ist es,
zum Beispiel Schnittstellen und Änderungen zu reduzieren.
\textbf{Partizipation und Veränderung} umfassen alle Methoden und Werkzeuge,
die zur Motivation der Mitarbeiter
und zum Abbau bzw. zur Vermeidung
von Widerständen und Konflikten beitragen. Dazu werden alle betroffenen
Organisationseinheiten am ProdukBild 4. Gestaltungsfelder im Lean Ramp-up
tionsanlauf beteiligt. Ziel ist es, die
Potenziale der Mitarbeiter zu nutzen
und einen reibungslosen Anlauf zu erreichen.

 \textbf{Wertschöpfung und Just-in-Time (JIT)}
umfassen alle Methoden und Werkzeuge, die zur produktiven, schnellen
und termingerechten Herstellung
bzw. Lieferung der Produkte beitragen. Dazu werden alle Verluste in den
Produktions- und Logistikprozessen
eliminiert und eine fließende und
kundenorientierte Produktion aufgebaut. Ziel ist ein schlankes Produktionssystem.

 \textbf{Pilotierung und Qualifizierung} umfasst
alle Methoden und Werkzeuge, die
zur Absicherung von Produkt- und
Prozessreifegrad sowie zur Steigerung der Leistungsfähigkeit des Produktionssystems beitragen. Dazu
werden sog. Pilotbereiche eingerichtet in denen Produktionstests sowie
Mitarbeiterschulungen erfolgen. Ziel
ist eine steile Lern- bzw. Anlaufkurve.

 \textbf{Priorisierung und Standardisierung}
umfassen alle Methoden und Werkzeuge, die zur Reduzierung, Beherrschung und Vermeidung der technologischen, prozessualen und organisatorischen Komplexität im Produktionsanlauf beitragen. Dazu werden
Schwerpunkte gebildet und Referenzunterlagen erstellt. Ziel ist es, den
Aufwand im Produktionsanlauf zu reduzieren.

 \textbf{Reaktionsfähigkeit und Flexibilität} umfassen alle Methoden und Werkzeuge,
die zum zeitnahen Erkennen veränderter Randbedingungen und Störungen sowie zur kontinuierlichen Anpassung des Anlaufmanagements beitragen. Dazu werden Frühwarnsysteme etabliert und Handlungsoptionen
bestimmt. Ziel ist es, schnell auf Veränderungen und Störungen zu reagieren.

 \textbf{Fehler- und Risikovermeidung} umfasst
alle Methoden und Werkzeuge, die
zur präventiven Qualitätssicherung
und -verbesserung beitragen. Dazu
werden frühzeitig die Ergebnisse der
Produkt- und Produktionsentwicklung veranschaulicht und Fehler- bzw.
Risikopotentiale eliminiert. Ziel sind
eine hohe Produkt- und Prozessqualität sowie geringe Änderungs- und
Prüfkosten.

 \textbf{Problemlösung und Stabilisierung} umfassen alle Methoden und Werkzeuge,
die zur reaktiven Qualitätssicherung
und -verbesserung beitragen. Dazu
werden die Produkte und Prozesse
kontinuierlich überprüft und überwacht sowie systematisch Problemursachen beseitigt. Ziel ist eine Stabilisierung des Anlaufs und Vermeidung
von Folge- und Wiederholungsfehlern.

 \textbf{Wissenstransfer und KVP} umfassen
alle Methoden und Werkzeuge, die
zum Transfer von Erfahrungswissen
und zur Erhöhung der Mitarbeiterkompetenzen beitragen. Dazu wird
explizites und – soweit möglich – implizites Wissen identifiziert, gesammelt, aufbereitet und vermittelt. Ziel
ist es, mit dessen Nutzung und
Weiterentwicklung aktuelle und zukünftige Anläufe zu verbessern.

 \textbf{Transparenz und Visualisierung} umfassen alle Methoden und Werkzeuge,
die zur Verfügbarkeit und leicht verständlichen Darstellung von Informationen und Daten beitragen. Dazu
werden sowohl informations- und
kommunikationstechnische Systeme
als auch optische Hilfsmittel und Signale eingesetzt. Ziel ist die Regelung,
Steuerung und Verbesserung des Produktionsanlaufs.
  
\newpage
\blankpage
\end{document}
